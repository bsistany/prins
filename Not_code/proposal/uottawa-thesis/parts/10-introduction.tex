%======================================================================
\chapter{Introduction}
%======================================================================
\ac{drm} refers to the digital management of rights associated with the access or usage of digital assets. There are various aspects of rights management however. According to the authors of the whitepaper ``A digital rights management ecosystem model for the education community,'' digital rights management systems cover the following four areas: 1) defining rights 2) distributing/acquiring rights 3) enforcing rights and 4) tracking usage \cite{collier}. [REF]

\ac{rel}s, or more precisely \ac{drel}s when dealing with digital assets deal with the ``rights definition'' aspect of the \ac{drm} ecosystem. A \ac{drel}, allows the expression and definition of digital asset usage rights such that other areas of the \ac{drm} ecosystem, namely the enforcement mechanism and the usage tracking components can function correctly.

Currently the most popular \ac{rel}s are the \ac{xrml} [REF], and the \ac{odrl}[REF]. Both of these languages are XML based and are considered declarative languages. \ac{xrml} has been selected to be the REL for \emph{MPEG-21} which is an ISO standard for multimedia applications. \ac{odrl} is also a standards based \ac{rel} which has been accepted as part of the W3C community with the mandate of standardizing how rights and policies, related to the usage of digital content on the Open Web Platform, \emph{OWP}, are expressed [REF][wikipedia]. \ac{odrl} 2.0 supports expression of rights and also privacy rules for social media while \ac{odrl} 1.0 was only dealing with the mobile ecosystem -- \ac{odrl} 1.0 was adopted by the Open Mobile Alliance, \emph{OMA} in 2000.

As popular as both \ac{xrml} and \ac{odrl} are, their adoption and usage is still somewhat limited in practice. Both Apple and Microsoft for example have defined their own lightweight \ac{rel}s [REF][problem with \ac{rel}s paper] in \emph{Fair Play} (Apple) and in \emph{PlayReady} (Microsoft). The authors of [REF][the problem with \ac{rel}s] argue that both these \ac{rel}s and other ones are simply too complex to be used effectively since they try to cover much of the \ac{drm} ecosystem. 

Another issue with the current batch of \ac{rel}s are due to their semantics being expressed in a natural language (e.g. English). By necessity natural languages are ambiguous and open to interpretation. 

To formalize the semantics of \ac{rel}s several approaches have been attempted by various authors. The main categories are logic based [REF], operational semantics based interpreters [REF] and finally web ontology based (from the Knowledge Representation Field)[REF]. In this thesis we will focus on the logic based approach to formalizing semantics and will study a specific logic based language that is a translation from a subset of \ac{odrl}.










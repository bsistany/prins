%======================================================================
\chapter{ODRL0 Semantics}\label{chap:semantics}
%======================================================================

                  
%----------------------------------------------------------------------
\section{Introduction}\label{sec:introsemantics}


In this section, we describe the semantics of ODRL0 language by a translation from agreements to a subset of many-sorted first-order logic formulas with equality. Note that in the listings in this chapter we use $[\![]\!]$ (double square brackets) notation as a mapping of ODRL0 syntactic elements to their translations as many-sorted first-order logic formulas. We also use $\triangleq$ between a translation and its corresponding formula to mean the translation is ``mapped to'' the formula. The semantics will help answer queries of the form ``may subject \emph{s} perform action \emph{act} to asset \emph{a}?''. If the answer is yes, we say permission is granted. Otherwise permission is denied. 

We use superscripts to denote parameters to translation mappings (with some exceptions explained later). However we make a distinction on whether the translation notation is used on the \ac{rhs} of a $\triangleq$ or the \ac{lhs}. The right side occurrence is similar to a function call where we pass actual parameters to the function call - here to the mapping. The left side occurrence is similar to function declarations where we define formal parameters - here when we define the translation for a construct for the first time. Another distinction between \ac{lhs} and \ac{rhs} occurrences is in the naming of the occurrence. When the translation for a construct appears on the \ac{rhs} of a $\triangleq$ we use the top-level name of the construct as it appears on the \ac{lhs} of the corresponding production. For example, see the \ac{rhs} in the listing~\ref{lst:transAgreementast} where the translation of a policy set mentions $policySet$ (corresponding to the name used on the \ac{lhs} of~\ref{lst:policySetast}). When we define the semantics for a construct, the definition is in terms of cases, one for each clause of the grammar for the construct. For $policySet$, the descriptive names for the cases are 'Primitive Policy Set', 'Primitive Exclusive Policy Set' and 'Conjunction of Policies'. The descriptive names for the cases are seen as comments when we defined the syntax (see~\ref{lst:policySetast}). Note also that the number of parameters/superscripts may be different depending on whether the translation is a \ac{rhs} or a \ac{lhs}. 



At a high-level, an agreement is translated into a conjunction of formulas of the form $\forall x ( prerequisites(x) \rightarrow P(x))$ where $P(x)$ itself is a conjunction of formulas of the form $ prerequisites(x) \rightarrow (\lnot) Permitted (x, act, a)$, where $Permitted (x, act, a)$ means the subject $x$ is permitted to perform action $act$ on asset $a$.

\section{Agreement Translation}
The translation of an $agreement$ returns the translation for a $policySet$ with arguments $prin_{u}$, the agreement's user and $a$, the asset.


\lstset{mathescape, language=AST}  
\begin{lstlisting}[frame=single, caption={Agreement Translation},label={lst:transAgreementast}]
$[\![ agreement$ $for$ $prin_{u}$ $about$ $a$ $with$ $policySet]\!]$ $\triangleq$ $[\![policySet]\!]^{prin_{u}, a}$
\end{lstlisting}

\section{Policy Set Translation}
The translation for a $policySet$ ($[\![policySet]\!]^{prin_{u}, a}$) is defined by cases, one for each clause of the grammar in listing~\ref{lst:policySetast}. Recall that a $policySet$ is either a $PrimitivePolicySet$, a $\linebreak PrimitiveExclusivePolicySet$ or a $AndPolicySet$. Each of these has its own translation function, which will be defined in the next 3 subsections. See~\ref{sec:getIdtranslation} for the translation of $getId$ and~\ref{sec:acttranslation} for the translation of $act$ mentioned below.


\subsection{PrimitivePolicySet Translation}
Translation of a $PrimitivePolicySet$ ($preRequisite \rightarrow policy$) yields a formula that includes a test on whether the subject is in the set of agreements' users, the translation of the policy and the translation of the $prerequisite$. Basically if the subject in question is a user of the agreement and the policySet prerequisites hold, then the policy holds. Translation of the policy for a $PrimitivePolicySet$ is called a \emph{positive translation}. A positive translation is one where the actions described by the policies are permitted.   

\lstset{mathescape, language=AST}  
\begin{lstlisting}[frame=single, caption={Policy Set Translation {$\colon$} PrimitivePolicySet},label={lst:transpolicyformulaPrimitivePolicySet}]
$[\![ preRequisite \rightarrow policy]\!]^{prin_{u}, a}$ $\triangleq$ $\forall x$ $(\!( [\![prin_{u}]\!]_{x}$ $\land$ $[\![preRequisite]\!]^{getId (policy), prin_{u}, a}_{x}) \rightarrow [\![policy]\!]^{positive, prin_{u}, a}_{x}\!)$
\end{lstlisting}



\lstset{mathescape, language=AST} 
\begin{lstlisting}[frame=single, caption={Positive Policy Translation {$\colon$} Single policy},label={lst:transpolicypositiveSingle}]

$[\![ preRequisite \Rightarrow_{policyId} act ]\!]^{positive, prin_{u}, a}_{x}$ $\triangleq$ $([\![ preRequisite ]\!]^{\left\{{policyId}\right\}, prin_{u}, a}_{x}) \Rightarrow Permitted(x, [\![act]\!], a)$

\end{lstlisting}


If the policy is a $AndPolicy$, the translation yields a conjunction of positive translations of each policy in turn.

\lstset{mathescape, language=AST}  
\begin{lstlisting}[frame=single, caption={Positive Policy Translation {$\colon$} List of policies},label={lst:transpolicypositiveListOfPolicies}]

$[\![ and [policy_{1}, ..., policy_{m}]]\!]^{positive, prin_{u}, a}_{x}$ $\triangleq$ $[\![policy_{1}]\!]^{positive, prin_{u}, a}_{x}$ $\land$ $...$ $\land$ $[\![policy_{m}]\!]^{positive, prin_{u}, a}_{x}$

\end{lstlisting}


\subsection{PrimitiveExclusivePolicySet Translation}
$PrimitiveExclusivePolicySet$ ($preRequisite \mapsto policy$) yields the conjunction of two implications. The first implication, is the same as one found in the translation of $PrimitivePolicySet$. The second implication however restricts access (to make the policy set exclusive) to only those subjects that are in the agreement's user. Translation of the policy in the second implication is called a \emph{negative translation}. A negative translation is one where the actions described by the policies are not permitted. 


\lstset{mathescape, language=AST}  
\begin{lstlisting}[frame=single, caption={Policy Set Translation {$\colon$} PrimitiveExclusivePolicySet},label={lst:transpolicyformulaPrimitiveExclusivePolicySet}]
$[\![ preRequisite \mapsto policy]\!]^{prin_{u}, a}$ $\triangleq$ $\forall x$ $(\!( [\![prin_{u}]\!]_{x}$ $\land$ $[\![preRequisite]\!]^{getId (policy), prin_{u}, a}_{x}) \rightarrow [\![policy]\!]^{positive, prin_{u}, a}_{x}\!)$ $\land$ $\forall x$ $(\neg[\![prin_{u}]\!]_{x} \rightarrow [\![policy]\!]^{negative, a}_{x})$
\end{lstlisting}


\lstset{mathescape, language=AST}  
\begin{lstlisting}[frame=single, caption={Negative Policy Translation {$\colon$} Single policy},label={lst:transpolicynegativeSingle}]

$[\![ preRequisite \Rightarrow_{policyId} act ]\!]^{negative, a}_{x}$ $\triangleq$ $\lnot (Permitted(x, [\![act]\!], a))$
\end{lstlisting}

If the policy is a $AndPolicy$, the translation yields a conjunction of negative translations of each policy in turn.

\lstset{mathescape, language=AST}  
\begin{lstlisting}[frame=single, caption={Negative Policy Translation {$\colon$} List of policies},label={lst:transpolicynegativeListOfPolicies}]

$[\![ and [policy_{1}, ..., policy_{m}]]\!]^{negative, a}_{x}$ $\triangleq$ $[\![policy_{1}]\!]^{negative, a}_{x}$ $\land$ $...$ $\land$ $[\![policy_{m}]\!]^{negative, a}_{x}$

\end{lstlisting}

\subsection{AndPolicySet Translation}
$AndPolicySet$ translates to conjunctions of the corresponding policy set translations. 

\lstset{mathescape, language=AST}  
\begin{lstlisting}[frame=single, caption={Policy Set Translation {$\colon$} AndPolicySet},label={lst:transpolicyformulaAndPolicySet}]
$[\![ and [policySet_{1}, ..., policySet_{m}]]\!]^{prin_{u}, a}$ $\triangleq$ $[\![policySet_{1}]\!]^{prin_{u}, a}$ $\land$ $...$ $\land$ $[\![policySet_{m}]\!]^{prin_{u}, a}$

\end{lstlisting}

\subsection{getId Translation}\label{sec:getIdtranslation}

The $getId$ function when applied to a single $policy$(see~\ref{lst:policyast}) is defined to return the $policyId$ of the $policy$.

\lstset{mathescape, language=AST}  
\begin{lstlisting}[frame=single, caption={getId for a Single policy},label={lst:getIdSinglePolicyAST}]
$[\![ getId (policy) ]\!]$ $\triangleq$ $[\![ getId (preRequisite \Rightarrow_{policyId} act) ]\!]$ $\triangleq$ $policyId$
\end{lstlisting}

$getId$ function when applied to a set of policies is the set union of the translations for each individual policy(see~\ref{lst:getIdSinglePolicyAST}).

\lstset{mathescape, language=AST}  
\begin{lstlisting}[frame=single, caption={getId for Policies},label={lst:getIdAndPolicyAST}]

$[\![ getId (and [policy_{1}, ..., policy_{m}) ]\!]$ $\triangleq$ $getId (policy_{1})$ $\cup$ $...$ $\cup$ $getId (policy_{m})$
\end{lstlisting}

\subsection{action Translation}\label{sec:acttranslation}

Translation of actions, such as $Play$ and $Display$ is simply the constant corresponding to each action instance.

\lstset{mathescape, language=AST}  
\begin{lstlisting}[frame=single, caption={act Translation},label={lst:actiontranslationAST}]
$[\![act]\!]$ $\triangleq$ $N$
\end{lstlisting}


\section{Principal Translation}
Translation for a \emph{prin} ($[\![ prin ]\!]_{x}$) is a formula that is true if and only if the subject $x$ is in the prin set. A $prin$ is either a single subject or a list of subjects ($\{ subject_{1}, ..., subject_{m} \})$ so the translation covers both cases. Each of these has its own translation function, which will be defined in the following 2 subsections.


If the $prin$ is a single subject, the translation is a formula that is true if and only if the subject $x$ is the same as the single subject $subject$.

\subsection{Single Subject Translation}
\lstset{mathescape, language=AST}  
\begin{lstlisting}[frame=single, caption={Prin Translation {$\colon$} Single subject},label={lst:transprinSingle}]
$[\![ subject ]\!]_{x}$ $\triangleq$ $x=subject$
\end{lstlisting}

\subsection{List of Subjects Translation}
Translation of a list of subjects is the disjunction of the translations for each subject.

\lstset{mathescape, language=AST}  
\begin{lstlisting}[frame=single, caption={Prin Translation {$\colon$} List of subjects},label={lst:transprinListOfSubjects}]

$[\![ \{ subject_{1}, ..., subject_{m} \} ]\!]_{x}$ $\triangleq$ $[\![subject_{1}]\!]_{x}$ $\lor$ $...$ $\lor$ $[\![subject_{m}]\!]_{x}$

\end{lstlisting}




\section{Prerequisite Translation}

Translation for a prerequisite is a formula $[\![prerequisite]\!]^{\left\{ {id_{1}, ..., id_{m}}\right\}, prin, a}_{x}$, where the set of $id$s refer to identifiers for policies that are implied by the prerequisites, $prin$ is the set of users (typically the agreement's user(s) denoted by $prin_{u}$) to which the prerequisites apply, $a$ is the asset and $x$ is a variable of type $subject$. The translation for a $prerequisite$ is described by translation formulas for each type of $prerequisite$. A $prerequisite$ is either always $true$, a $Constraint$, a $ForEachMember$, a $NotCons$, a $AndPrqs$, a $OrPrqs$ or a $XorPrqs$. Each of these has its own translation function, which will be defined in the following subsections.

\subsection{True Prerequisite Translation}
The translation for a $TruePrq$ yields a formula that is always \emph{true}.

\lstset{mathescape, language=AST}  
\begin{lstlisting}[frame=single, caption={Prerequisite Translation {$\colon$} Always True Prerequisite},label={lst:transpreRequisiteTruePrq}]
	$[\![ true ]\!]^{\left\{ {id_{1}, ..., id_{m}}\right\}, prin, a}_{x}$ $\triangleq$ True
\end{lstlisting}

\subsection{Constraint Prerequisite Translation}
The translation for a $Constraint$ is handled by a specialized constraint translation function (coverage of which starts at listing~\ref{lst:transconstraintPrin}.

\lstset{mathescape, language=AST}  
\begin{lstlisting}[frame=single, caption={Prerequisite Translation {$\colon$} Constraint},label={lst:transpreRequisiteConstraint}]

$[\![ constraint ]\!]^{\left\{ {id_{1}, ..., id_{m}}\right\}, prin, a}_{x}$ $\triangleq$ $[\![ constraint ]\!]^{\left\{ {id_{1}, ..., id_{m}}\right\}, prin}_{x}$ 
\end{lstlisting}

\subsection{ForEachMember Prerequisite Translation}
The translation for a $ForEachMember$ is also is handled by a specialized translation function (listing~\ref{lst:transforEachMember}.

\lstset{mathescape, language=AST}  
\begin{lstlisting}[frame=single, caption={Prerequisite Translation {$\colon$} ForEachMember},label={lst:transpreRequisiteForEachMember}]

$[\![ forEachMember ]\!]^{\left\{ {id_{1}, ..., id_{m}}\right\}, prin, a}_{x}$ $\triangleq$ $[\![ forEachMember ]\!]^{\left\{ {subject_{1}, ..., subject_{k}}\right\}, \left\{ {constraint_{1}, ..., constraint_{n}}\right\}, \left\{ {id_{1}, ..., id_{m}}\right\}}_{x}$ 	
\end{lstlisting}

\subsection{NotCons Prerequisite Translation}
The translation for a $NotCons$ yields a formula that is simply the negation of the translation for a constraint.

\lstset{mathescape, language=AST}  
\begin{lstlisting}[frame=single, caption={Prerequisite Translation {$\colon$} Not Constraint},label={lst:transpreRequisiteNotConstraint}]

$[\![ not$ $constraint ]\!]^{\left\{ {id_{1}, ..., id_{m}}\right\}, prin, a}_{x}$ $\triangleq$ $\lnot[\![ constraint ]\!]^{\left\{ {id_{1}, ..., id_{m}}\right\}, prin}_{x}$ 
\end{lstlisting}

\subsection{AndPrqs Prerequisite Translation}
The translation for a $AndPrqs$ yields a formula that is the conjunction of the translation for each $preRequisite$.

\lstset{mathescape, language=AST}  
\begin{lstlisting}[frame=single, caption={Prerequisite Translation {$\colon$} Conjunction },label={lst:transpreRequisiteAndPrqs}]

$[\![and$ $[preRequisite_{1}, ..., preRequisite_{k}]]\!]^{\left\{ {id_{1}, ..., id_{m}}\right\}, prin, a}_{x}
$ $\triangleq$ $[\![preRequisite_{1}]\!]^{\left\{ {id_{1}, ..., id_{m}}\right\}, prin, a}_{x}$ $\land$ $...$ $\land$ $[\![preRequisite_{k}]\!]^{\left\{ {id_{1}, ..., id_{m}}\right\}, prin, a}_{x}$

\end{lstlisting}

\subsection{OrPrqs Prerequisite Translation}
The translation for a $OrPrqs$ yields a formula that is the inclusive disjunction of the translation for each $preRequisite$.

\lstset{mathescape, language=AST}  
\begin{lstlisting}[frame=single, caption={Prerequisite Translation {$\colon$} Inclusive Disjunction},label={lst:transpreRequisiteOrPrqs}]

$[\![or$ $[preRequisite_{1}, ..., preRequisite_{k}]]\!]^{\left\{ {id_{1}, ..., id_{m}}\right\}, prin, a}_{x}$ $\triangleq$ $[\![preRequisite_{1}]\!]^{\left\{ {id_{1}, ..., id_{m}}\right\}, prin, a}_{x}$ $\lor$ $...$ $\lor$ $[\![preRequisite_{k}]\!]^{\left\{ {id_{1}, ..., id_{m}}\right\}, prin, a}_{x}$

\end{lstlisting}

\subsection{XorPrqs Prerequisite Translation}
The translation for a $XorPrqs$ yields a formula that is the exclusive disjunction of the translation for each $preRequisite$.


\lstset{mathescape, language=AST}  
\begin{lstlisting}[frame=single, caption={Prerequisite Translation {$\colon$} Exclusive Disjunction},label={lst:transpreRequisiteXorPrqs}]

$[\![Xor$ $[preRequisite_{1}, ..., preRequisite_{k}]]\!]^{\left\{ {id_{1}, ..., id_{m}}\right\}, prin, a}_{x}$ $\triangleq$ $[\![preRequisite_{1}]\!]^{\left\{ {id_{1}, ..., id_{m}}\right\}, prin, a}_{x}$ $\oplus $ $...$ $\oplus$ $[\![preRequisite_{k}]\!]^{\left\{ {id_{1}, ..., id_{m}}\right\}, prin, a}_{x}$

\end{lstlisting}


\section{Constraint Translation}

Translation for a constraint is a formula $[\![constraint]\!]^{\left\{ {id_{1}, ..., id_{m}}\right\}, prin}_{x}$, where the set of $id$s refer to identifiers for policies that are implied by the constraint, $prin$ is the set of users (typically the agreement's user(s) denoted by $prin_{u}$ to which the constraint applies, $x$ is a variable of type $subject$. The translation for a $constraint$ is described by translation formulas for each type of $constraint$. A $constraint$ is either a $Principal$, a $Count$, or a $CountByPrin$. Each of these has its own translation function, which will be defined in the following subsections.

\subsection{Principal Constraint Translation}
The translation for a $Principal$ is handled by a specialized translation function (listing~\ref{lst:transprin}).   

\lstset{mathescape, language=AST}  
\begin{lstlisting}[frame=single, caption={Constraint Translation {$\colon$} Principal},label={lst:transconstraintPrin}]

$[\![ prin ]\!]^{\left\{ {id_{1}, ..., id_{m}}\right\}, prin}_{x}$ $\triangleq$ $[\![ prin ]\!]_{x}$ 
\end{lstlisting}

\subsection{Count Constraint Translation}
The translation for a $Count$ is handled by a specialized translation function in listing~\ref{lst:transconstraintCount}. 


\subsection{CountByPrin Constraint Translation}
The translation for a $CountByPrin$ is handled by the same specialized translation function as that for $Count$. The difference is that $CountByPrin$ overrides the subjects passed in $prin$ (typically agreement's user(s) or $prin_{u}$) by a different set of subjects (listing~\ref{lst:transconstraintCountbyPrin}).

\lstset{mathescape, language=AST}  
\begin{lstlisting}[frame=single, caption={Constraint Translation {$\colon$} Count by Principal},label={lst:transconstraintCountbyPrin}]

$[\![ prin(count [N]) ]\!]^{\left\{ {id_{1}, ..., id_{m}}\right\}, prin}_{x}$ $\triangleq$ $[\![ prin(count [N]) ]\!]^{\left\{ {subject_{1}, ..., subject_{m}}\right\}, \left\{ {id_{1}, ..., id_{n}}\right\}}_{x}$ 
\end{lstlisting}


\subsection{forEachMember Translation}

\lstset{mathescape, language=AST}  
\begin{lstlisting}[frame=single, caption={ForEachMember Translation {$\colon$} Count by Principal},label={lst:transforEachMember}]

$[\![ forEachMember ]\!]^{\left\{ {subject_{1}, ..., subject_{k}}\right\}, \left\{ {constraint_{1}, ..., constraint_{n}}\right\}, \left\{ {id_{1}, ..., id_{m}}\right\}}_{x}$ $\triangleq$ $\bigwedge_{(i, subject)\in (subjects \times \left\{ {1, ..., m}\right\})}\nolimits$ $[\![constraint_{i}]\!]^{\left\{ {id_{1}, ..., id_{m}}\right\}, \left\{subject\right\}}_{x}$


\end{lstlisting}

\subsection{''Not Constraint" Translation}

The translation for ``Not Constraint'' was listed in ~\ref{lst:transpreRequisiteNotConstraint} earlier but we repeat it here to go along the Coq version. 

\lstset{mathescape, language=AST}  
\begin{lstlisting}[frame=single, caption={Not Constraint Translation},label={lst:transnotCons}]

$[\![ not$ $constraint ]\!]^{\left\{ {id_{1}, ..., id_{m}}\right\}, prin_{u}}_{x}$ $\triangleq$ $\lnot[\![ constraint ]\!]^{\left\{ {id_{1}, ..., id_{m}}\right\}, prin_{u}}_{x}$ 

\end{lstlisting}

\section{Count Translation}
Translation for $Count$ or $CountByPrin$ is based on whether the translation is on a single pair or multiple pairs of subject/policyIds. Each of these two cases will be described in the following subsections.

\subsection{Count Translation For Subject/ID Pair}
The translation for $Count$ or $CountByPrin$ for a pair of subject and policy identifier is a formula that is true if the number of times $subject_{1}$ has invoked a policy with policy identifier $id_{1}$ is smaller than $N$.

\lstset{mathescape, language=AST}  
\begin{lstlisting}[frame=single, caption={Count Translation {$\colon$} subject and policyId Pair},label={lst:transcountSinglePair}]

$[\![ count [N] ]\!]^{subject_{1}, id_{1}}_{x}$ $\triangleq$ $getCount(subject_{1}, id_{1}) < N$
\end{lstlisting}

\subsection{Count Translation For Subject/ID Pairs}
The translation for $Count$ or $CountByPrin$ for subject and policy identifier pairs is a formula that is true when the sum of the number of times that $subject_{i}$ has invoked a policy with policy identifier $id_{j}$ is smaller than $N$.

\lstset{mathescape, language=AST}  
\begin{lstlisting}[frame=single, caption={Count Translation {$\colon$} subject and policyId Pairs},label={lst:transcountSinglePairs}]

$[\![ count [N] ]\!]^{\left\{ {subject_{1}, ..., subject_{m}}\right\}, \left\{ {id_{1}, ..., id_{n}}\right\}}_{x}$ $\triangleq$ $(\sum_{(id, subject)\in (ids \times subjects)}\nolimits$ $getCount(subject, id)) < N$

\end{lstlisting}



%$getCount(subject_{1}, id_{1})$ $+$ $...$ $getCount(subject_{m}, id_{1})$ $getCount(subject_{1}, id_{1})$ $+$$...$ $getCount(subject_{m}, id_{1})$ < N$



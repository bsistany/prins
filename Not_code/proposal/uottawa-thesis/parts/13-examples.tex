%======================================================================
\chapter{Examples}\label{chap:examples}

%======================================================================

%----------------------------------------------------------------------
\section{Introduction}
%----------------------------------------------------------------------

%%%%%%% Examples 

In this chapter we will take a tour of the syntax and semantics we have so far developed by examining some example agreements. In the following we will start by reviewing some of the examples used in~\cite{pucella2006}.

Ultimately the goal of specifying all the syntax and semantics is to declare some interesting theorems about policy expressions and proving them in Coq. However we will first start with some specific propositions/theorems about these examples to get a feel for how proofs are done in Coq.

\section{Agreement 2.1}

Consider example 2.1 (from~\cite{pucella2006}) where the $policySet$ is a $AndPolicySet$ with $p1$ and $p2$ as the individual $policySet$s. Let $p1$ be defined as $Count[5]$ $\rightarrow$ $print$ and $p2$ as $and[Alice, Count[2]]$ $\rightarrow$ $print$. 

The agreement is that the asset \emph{The Report} may be printed a total of five times by either \emph{Alice} or \emph{Bob}, and twice more by Alice. So if Alice and Bob have used policy $p1$ to justify their printing of the asset $m_{p1}$ and $n_{p1}$ times, respectively, then either may do so again if $m_{p1} + n_{p1} < 5$. If they have used $p2$ to justify their printing of the asset $m_{p2}$ and $n_{p2}$ times, respectively, then only Alice may do so again if $m_{p2} + n_{p2} < 2$. Note that since Bob doesn't meet the prerequisite of being Alice, $n_{p2}$ is effectively $0$, so we have $m_{p2} < 2$ as the condition for Alice being able to print again (Alice does meet the prerequisite of being Alice).

\lstset{language=Pucella2006}
\begin{lstlisting}[frame=single, caption={Agreement 2.1 (as used in~\cite{pucella2006})},label={lst:example21pucella2006}]
agreement
 for {Alice, Bob} 
 about TheReport 
 with and [p1, p2].
\end{lstlisting}

The Coq version of the agreement 2.1 (listing~\ref{lst:example21pucella2006}) and its sub-parts is listed below. It is best to start with the agreement itself called $A2.1$ in the listing and compare to the agreement 2.1 listed in ~\ref{lst:example21pucella2006}.

\lstset{language=Coq}
\begin{minipage}[c]{0.95\textwidth}
\begin{lstlisting}[frame=single, caption={Agreement 2.1 in Coq},label={lst:example21}]


Definition p1A1:policySet :=
  PrimitivePolicySet
    TruePrq
    (PrimitivePolicy (Constraint (Count  5)) id1 Print).

Definition p2A1prq1:preRequisite := (Constraint (Principal (Single Alice))).
Definition p2A1prq2:preRequisite := (Constraint (Count 2)).

Definition p2A1:policySet :=
  PrimitivePolicySet
    TruePrq
    (PrimitivePolicy (AndPrqs (NewList p2A1prq1 (Single p2A1prq2))) id2 Print).

Definition A2.1 := Agreement (NewList Alice (Single Bob)) TheReport
                  (AndPolicySet (NewList p1A1 (Single p2A1))).

\end{lstlisting}                  
\end{minipage}

\section{Agreement 2.5}

Consider example 2.5 (from~\cite{pucella2006}) where the $policySet$ is a $PrimitivePolicySet$ with a $Count$ constraint as prerequisite and a $AndPolicy$ as the policy. The $AndPolicy$ is the conjunction of two $PrimitivePolicy$s. Both policies have prerequisites of type $ForEachMember$ with actions $display$ and $print$ respectively. The $prin$ component for both $ForEachMember$s is ${Alice, Bob}$, whereas the constraint for the first $ForEachMember$ is $Count[5]$ and for the second is $Count[2]$.

\lstset{language=Pucella2006}
\begin{lstlisting}[frame=single, caption={Agreement 2.5 (as used in~\cite{pucella2006})},label={lst:example25pucella2006}]
agreement
 for {Alice, Bob} 
 about ebook 
 with Count [10] $\rightarrow$ and [forEachMember[{Alice, Bob}; Count[5]] $\Rightarrow_{id1}$ display,
                                 forEachMember[{Alice, Bob}; Count[1]] $\Rightarrow_{id2}$ print].
\end{lstlisting}

The Coq version of the agreement 2.5 (listing~\ref{lst:example25pucella2006}) and its sub-parts is listed below. See agreement 2.5 listed in ~\ref{lst:example25pucella2006} for comparison.

The agreement is that the asset \emph{ebook} may be displayed up to five times by Alice and Bob each, and printed once by each. However the total number of actions (either $display$ or $print$) justified by the two policies by either Alice and Bob is at most 10.


\lstset{language=Coq}
\begin{lstlisting}[frame=single, caption={Example 2.5},label={lst:example25}]

Definition tenCount:preRequisite := (Constraint (Count 10)).
Definition fiveCount:constraint := (Count 5).
Definition oneCount:constraint := (Count 1).

Definition prins2_5 := (NewList Alice (Single Bob)).
Definition forEach_display:preRequisite := ForEachMember prins2_5 (Single fiveCount).
Definition forEach_print:preRequisite := ForEachMember prins2_5 (Single oneCount).

Definition primPolicy1:policy := PrimitivePolicy forEach_display id1 Display.
Definition primPolicy2:policy := PrimitivePolicy forEach_print id2 Print.

Definition policySet2_5:policySet :=
  PrimitivePolicySet tenCount (AndPolicy (NewList primPolicy1 (Single primPolicy2))).
                     
Definition A2.5 := Agreement prins2_5 ebook policySet2_5.

\end{lstlisting}








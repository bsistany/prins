%======================================================================
\chapter{Proposed Future Work}
%======================================================================

%----------------------------------------------------------------------
\section{Introduction}
%----------------------------------------------------------------------

We first mentioned queries in chapter~\ref{chap:semantics} on page~\pageref{chap:semantics}. Ultimately policy statements describing an agreement will be used to enforce those agreements. To enforce policy agreements, access queries are asked from the policy engine and access is granted or denied based on the answer.

By defining formal semantics for \ac{odrl} the authors of ~\cite{pucella2006} were able to prove that answering a query on whether access should be granted or not, is decidable and NP-hard for the full \ac{odrl}. 

In this chapter we will review our encoding of queries in Coq and Coq representations of other definitions used in ~\cite{pucella2006} which we will use to prove decidability results of our own. 


\section{Queries}

Queries are tuples of the form $(A, s, action, a, e)$ in ~\cite{pucella2006}. The tuple corresponds to the question of determining whether a set $A$ of agreements imply that a subject $s$ may perform action $action$ on an asset $a$ given the environment $e$. The Coq representation is listed in listing~\ref{lst:querycoq}. We distinguish single agreement queries from multiple agreement queries by defining two separate types: $single\_query$ and $general\_query$.

\lstset{language=Coq, frame=single, caption={Queries},label={lst:querycoq}}
\begin{minipage}[c]{0.95\textwidth}
\begin{lstlisting}

Inductive single_query : Set := 
   | SingletonQuery : agreement -> subject -> act -> asset -> environment -> single_query.
   

Inductive general_query : Set := 
   | GeneralQuery : nonemptylist agreement -> subject -> act -> asset -> environment -> general_query.
\end{lstlisting}
\end{minipage}

\section{Answering Queries}\label{sec:answerqueriesodrl}

Answering a query as defined earlier can lead to one of four outcomes: error (~\ref{lst:errordecision}), permitted (~\ref{lst:permitdecision}), denied (~\ref{lst:denydecision}) and ``not applicable'' (~\ref{lst:notapplicabledecision}) defined in ~\cite{Tschantz}.

\lstset{mathescape, language=AST} 
\begin{lstlisting}[frame=single, caption={Answerable Queries: Error},label={lst:errordecision}]

$(\bigwedge [\![ agreement]\!]) \land E \implies Permitted(s, act, a)$ and $(\bigwedge [\![ agreement]\!]) \land E \implies \lnot Permitted(s, act, a)$

\end{lstlisting}

\lstset{mathescape, language=AST} 
\begin{lstlisting}[frame=single, caption={Answerable Queries: Permit},label={lst:permitdecision}]

$(\bigwedge [\![ agreement]\!]) \land E \implies Permitted(s, act, a)$ and $(\bigwedge [\![ agreement]\!]) \land E \notimplies \lnot Permitted(s, act, a)$

\end{lstlisting}

\lstset{mathescape, language=AST} 
\begin{lstlisting}[frame=single, caption={Answerable Queries: Deny},label={lst:denydecision}]

$(\bigwedge [\![ agreement]\!]) \land E \notimplies Permitted(s, act, a)$ and $(\bigwedge [\![ agreement]\!]) \land E \implies \lnot Permitted(s, act, a)$

\end{lstlisting}

\lstset{mathescape, language=AST} 
\begin{lstlisting}[frame=single, caption={Answerable Queries: Not Applicable},label={lst:notapplicabledecision}]

$(\bigwedge [\![ agreement]\!]) \land E \notimplies Permitted(s, act, a)$ and $(\bigwedge [\![ agreement]\!]) \land E \notimplies \lnot Permitted(s, act, a)$

\end{lstlisting}


In ~\cite{pucella2006} a slightly different formulation is used to denote the same four decision types. ``Query Inconsistent'',``Permission Granted'', ``Permission Denied'' and ``Permission Unregulated''. The main difference is the fact that environments are explicitly taken into account in ~\cite{Tschantz} but are implicit in ~\cite{pucella2006} where the algorithm of ``holds'' takes into account the environment (notion of \emph{E-validity}).

\lstset{mathescape, language=AST} 
\begin{lstlisting}[frame=single, caption={$f^{+}_q$},label={lst:fplusformula}]

$f^{+}_q \triangleq (\bigwedge [\![ agreement]\!]) \implies Permitted(s, act, a)$ 

\end{lstlisting}

\lstset{mathescape, language=AST} 
\begin{lstlisting}[frame=single, caption={$f^{-}_q$},label={lst:fminusformula}]

$f^{-}_q \triangleq (\bigwedge [\![ agreement]\!]) \implies \lnot Permitted(s, act, a)$ 

\end{lstlisting}


\lstset{mathescape, language=AST} 
\begin{lstlisting}[frame=single, caption={Answerable Queries: Query Inconsistent},label={lst:queryinconsistentdecision}]

$f^{+}_q$ and $f^{-}_q$ both hold

\end{lstlisting}

\lstset{mathescape, language=AST} 
\begin{lstlisting}[frame=single, caption={Answerable Queries: Permission Granted},label={lst:permission granteddecision}]

$f^{+}_q$ holds and $f^{-}_q$ does not hold

\end{lstlisting}

\lstset{mathescape, language=AST} 
\begin{lstlisting}[frame=single, caption={Answerable Queries: Permission Denied},label={lst:permission denieddecision}]

$f^{+}_q$ does not hold and $f^{-}_q$ holds

\end{lstlisting}

\lstset{mathescape, language=AST} 
\begin{lstlisting}[frame=single, caption={Answerable Queries: Permission Unregulated},label={lst:permission unregulateddecision}]

$f^{+}_q$ does not hold and $f^{-}_q$ does not hold

\end{lstlisting}

\section{Machine-Checked Proof of Decidability of Queries}

By defining formal semantics for \ac{odrl} authors of ~\cite{pucella2006} were able to show some important results. First result is that answering the question of whether a set of ODRL statements imply a permission, denial or other possibilities is decidable and also that its complexity is NP-hard (see Theorem ~\ref{customthm41} from ~\cite{pucella2006} re-printed here below).

\begin{customthm}{4.1}\label{customthm41}
The problem of deciding, for a query $q = (A, s, act, a, E)$, whether $f^{+}_q$ is E-valid is decidable but NP-hard. Similarly for $f^{-}_q$.
\end{customthm}

The authors of ~\cite{pucella2006} then prove that by removing the construct $not[policySet]$ from \ac{odrl}'s syntax answering the same query remains decidable and efficient (polynomial time complexity). 

We will prove equivalent results as above starting with the decidability result of answering a query in ODRL0 (which does not include $not[policySet]$). The theorem in listing ~\ref{lst:decidabilityodrl0coq} states that for all environments, all single agreements, all subjects, all actions and all assets, either permission is granted, permission is denied, permission is unregulated or query is inconsistent. 

\lstset{language=Coq, frame=single, caption={Environments and Counts},label={lst:decidabilityodrl0coq}}
\begin{minipage}[c]{0.95\textwidth}
\begin{lstlisting}
Theorem queriesAreDecidable: forall (e:environment), 
                forall (agr: agreement),
                forall (s:subject),
                forall (action:act),
                forall (a:asset),

(permissionGranted e [agr] s action a) \/
(permissionDenied e [agr] s action a)  \/
(queryInconsistent e [agr] s action a) \/
(permissionUnregulated e [agr] s action a).

\end{lstlisting}
\end{minipage}

We will then augment ODRL0 with the constructs we omitted from the full ODRL (resulting in what we have earlier called ODRL1 or ODRL2) including the troublesome construct $not[policySet]$ and attempt to prove that the decidability results remain intact. There is a chance that a proof is not possible due to particulars of the Coq encoding we have used, in which case the we will detail the specifics.

\section{SELinux}
 
We started out by looking at \ac{drel}s and specifically \ac{odrl} where rights expressions are used to arbitrate access to assets under conditions. Recall that the main construct in \ac{odrl} is the agreement which specifies users, asset(s) and policies (as part of policy sets) whereby controls on users' access to the assets are described. This is reminiscent of how access control conditions are expressed in access control policy languages such as \ac{xacml} and \ac{selinux}.

While \ac{xacml} is a high-level and platform independent access control system \ac{selinux} is platform dependent (e.g. Linux based) and low-level. \ac{selinux} enhances the \ac{DAC} that most unix based systems employ by \ac{MAC} where designed access control policies are applied throughout the system possibly overriding whatever \ac{DAC} is in place by the system users. 

\ac{selinux} uses Linux's extended file attributes to attach a \emph{security context} to passive entities (e.g. files, directories, sockets) and also to each active entity typically a Linux user space process. Security context is a data structure that is composed of a user, a role and a domain (or type). While users can map directly to ordinary user names they can also be defined separately. Roles are meant to group users and add flexibility when assigning permissions and are the basis for \ac{RBAC}. Finally domains or types are the basis for defining common access control requirements for both passive and active entities. 


The enforcement of \ac{selinux} policies are performed by the \emph{security server}. Whenever a security operation is requested from user space by a system call, the security server is invoked to arbitrate the operation and either allow the operation or to deny it. Each operation is identified by two pieces of information: an object class (e.g. file) and a permission (e.g. read, write). When an operation is requested to be performed on an object, the class and the permissions associated with the object along with security contexts of the source (typically the source entity is a process) and the object are passed to the security server. The security server consults the loaded policy (loaded at 
boot time) and allows or denies the access request [sarna-starosta]

\section{SELinux Policy Language}

The \ac{selinux} policy has four different kinds of statements: declarations, rules, constraints and assertions [archer]. Assertions are compile time checks that the \emph{checkpolicy} tool performs at compile time. The other three kinds of statements however are evaluated at run-time. 

Declaration statements declare a user, a role and a type. 

\lstset{language=selinux}
\begin{lstlisting}[frame=single, caption={Declarations},label={lst:declsselinux}]

user u types Ru;

role r types Tr;

type t, attrib_{1}, ..., attrib_{n};

\end{lstlisting}

Rule statements define access vector rules and type transition rules. Access vector (AV) rules (see ~\ref{lst:avruleselinux}) specify which operations are allowed and whether to audit (log). Any operation not covered by AV rules are denied and all denied operations are logged. The semantics of the AV rule with avkind \emph{allow} is: processes with type \emph{sourcetype} are allowed to perform operations in \emph{perm} on objects with class \emph{obj-class} and type \emph{targettype}. \emph{auditallow} means to allow and audit, \emph{dontaudit} means to never audit and finally \emph{neverallow} provides a mechanism to override allow rules. When a process changes security context, the role may change, assuming a role transition rule exists relating the old and the new roles (~\ref{lst:roletransitionselinux}).

\lstset{language=selinux}
\begin{lstlisting}[frame=single, caption={AV rule},label={lst:avruleselinux}]

avkind sourcetype targettype:object-class perm

avkind=allow, auditallow, dontaudit, neverallow
\end{lstlisting}

Constraints are additional conditions that must hold for an attempted operation to be allowed. Constrains relate all of their arguments (the security contexts) to the server (see ~\ref{lst:constrainselinux}. Whenever a permission is requested on an obj-class, the security server checks that the two security contexts are related by a constrain statement.

\lstset{language=selinux}
\begin{lstlisting}[frame=single, caption={Constrain rule},label={lst:constrainselinux}]

constrain classes, perms, sourcetype, sourcerole, sourceuser, targettype, targetrole, targetuser
\end{lstlisting}


\section{Agreements in \ac{selinux}}

As with \ac{odrl} we will start by limiting the policy language to only allow AV rules. As mentioned earlier an operation not covered by a allow rule is denied by \ac{selinux}. We will make up explicit \emph{deny} as required therefore an agreement is defined to be a combination of allow and deny rules. Allow and deny rules are mappings defined in listing ~\ref{lst:allowmappingastselinux}.

\lstset{language=AST}
\begin{lstlisting}[frame=single, caption={allow/deny rule as a mapping},label={lst:allowmappingastselinux}]
allow/deny rule : $T \times (T \times C) \rightarrow 2^{P}$
\end{lstlisting}

\lstset{language=AST}
\begin{lstlisting}[frame=single, caption={\ac{selinux} agreement},label={lst:agreementastselinux}]
<agreement> ::= <avRule> 
\end{lstlisting}


\lstset{mathescape, language=AST, escapechar=\&}  
\begin{lstlisting}[frame=single, caption={AV Rule},label={lst:avruleastselinux}]
<avRule> ::=  
	'allow' $(T1, T2, C, P)$	&\Comment{; allow rule }&
	'deny' $(T1, T2, C, P)$	    &\Comment{; deny rule }&
	'and'[ <avRule$_{1}$>, ..., <avRule$_{m}$> ]	&\Comment{; conjunction }&
\end{lstlisting}


\section{Environments}

Environments are collections of \emph{role-type} and \emph{user-role} relations. A role-type relation $role(R, T)$ simply associates a role with a type. A user-role relation $user(U, R)$ associates a user with a role. An environment is consistent with respect to a security context $<T, R, U>$, if and only if $role(R, T)$ and $user(U, R)$ relations hold in the environment. 

\section{Queries in \ac{selinux}}

The decision problem in \ac{selinux} access control is whether an entity with security context $<T1, R1, U1>$ may perform action $P1$ to entity with object class $C1$ with security context $<T2, R2, U2>$.

To answer such queries we use the authorization relation \emph{auth(C, P, T1, R1, U1, T2, R2, U2)} which is equivalent to the $Permitted$ predicate we saw earlier for \ac{odrl}.

\lstset{mathescape, language=AST} 
\begin{lstlisting}[frame=single, caption={$f^{+}_q$ for \ac{selinux}},label={lst:fqplussel}]

$allow (T1, T2, C, P)$ $\land$ $(E$ $consistent$ $wrt$ $<T1, R1, U1>$ and $<T2, R2, U2>)$ $\implies auth(C, P, T1, R1, U1, T2, R2, U2)$ 

\end{lstlisting}

\lstset{mathescape, language=AST} 
\begin{lstlisting}[frame=single, caption={$f^{-}_q$ for \ac{selinux}},label={lst:fqminussel}]

$deny (T1, T2, C, P)$ $\lor$ $\lnot (E$ $consistent$ $wrt$ $<T1, R1, U1>$ and $<T2, R2, U2>)$ $\implies \lnot auth(C, P, T1, R1, U1, T2, R2, U2)$ 

\end{lstlisting}

\section{Decidability of Queries in \ac{selinux}}

In this thesis we will be investigating the question of decidability for answering queries in \ac{selinux} policies based on the same four outcomes we encountered earlier in ~\ref{sec:answerqueriesodrl} namely error, permitted, denied and ``not applicable''. We will first state a decidability theorem similar to theorem ~\ref{lst:decidabilityodrl0coq} (minor adjustments may be needed to allow for differences with \ac{selinux} policy language) and present a proof for it in Coq. The literature in the \ac{selinux} implies only two outcomes are possible: permitted or denied. We will next attempt to prove this conjecture in Coq. Finally we will add constrain relations (see ~\ref{lst:fqplusconstsel} and ~\ref{lst:fqminusconstsel}) to \ac{selinux} policies (which we have not included so far) and prove the same decidability results again for the augmented policy.

\lstset{mathescape, language=AST} 
\begin{lstlisting}[frame=single, caption={$f^{+}_q$ for \ac{selinux}},label={lst:fqplusconstsel}]

$allow (T1, T2, C, P)$ $\land$ $constrain (C, P, T1, R1, U1, T2, R2, U2)$ $\land$
$(E$ $consistent$ $wrt$ $<T1, R1, U1>$ and $<T2, R2, U2>)$ $\implies auth(C, P, T1, R1, U1, T2, R2, U2)$ 

\end{lstlisting}

\lstset{mathescape, language=AST} 
\begin{lstlisting}[frame=single, caption={$f^{-}_q$ for \ac{selinux}},label={lst:fqminusconstsel}]

$deny (T1, T2, C, P)$ $\lor$ $\lnot$ $constrain (C, P, T1, R1, U1, T2, R2, U2)$ $\lor$
$\lnot (E$ $consistent$ $wrt$ $<T1, R1, U1>$ and $<T2, R2, U2>)$ $\implies \lnot auth(C, P, T1, R1, U1, T2, R2, U2)$ 

\end{lstlisting}























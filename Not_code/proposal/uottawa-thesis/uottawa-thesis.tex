% uOttawa (unofficial) Thesis Template for LaTeX 
% Edited by Wail Gueaieb based on Stephen Carr's uWaterloo Template

% DON'T USE THIS TEMPLATE IF YOU DON'T KNOW WHAT YOU'RE DOING!
% Remember, it comes WITH NO WARRANTY!

% Please read the "00readme.txt" file first.
% Here is how to use this template:
%
% DON'T FORGET TO ADD YOUR OWN NAME AND TITLE in the "hyperref" package
% configuration in the "thesis-preample.tex" file. THIS INFORMATION GETS 
% EMBEDDED IN THE PDF FINAL PDF DOCUMENT.
% You can view the information if you view Properties of the PDF document.

% The template is based on the standard "book" document class which provides 
% all necessary sectioning structures and allows multi-part theses.

% DISCLAIMER
% To the best of our knowledge, this template satisfies the current 
% uOttawa thesis requirements.
% However, it is your responsibility to assure that you have met all 
% requirements of the university and your particular department.
% Many thanks to the feedback from many graduates that assisted the 
% development of this template.

% -----------------------------------------------------------------------

% When using pdflatex, by default the output is geared toward generating a PDF 
% version optimized for viewing on an electronic display, including 
% hyperlinks within the PDF.
 
% E.g. to process a thesis based on this template, run:

% (pdf)latex uottawa-thesis	-- first pass of the (pdf)latex processor
% bibtex uottawa-thesis 	-- generates bibliography from .bib data file(s) 
% (pdf)latex uottawa-thesis	-- fixes cross-references, bibliographic references, etc
% (pdf)latex uottawa-thesis	-- fixes cross-references, bibliographic references, etc
% makeindex -s nomentbl.ist -o uottawa-thesis.nls uottawa-thesis.nlo
% (pdf)latex uottawa-thesis	-- fixes cross-references, bibliographic references, etc
% (pdf)latex uottawa-thesis	-- fixes cross-references, bibliographic references, etc



% N.B. The "pdftex" program allows graphics in the following formats to be
% included with the "\includegraphics" command: PNG, PDF, JPEG, TIFF
% Tip 1: Generate your figures and photos in the size you want them to appear
% in your thesis, rather than scaling them with \includegraphics options.
% Tip 2: Any drawings you do should be in scalable vector graphic formats:
% SVG, PNG, WMF, EPS and then converted to PNG or PDF, so they are scalable in
% the final PDF as well.
% Tip 3: Photographs should be cropped and compressed so as not to be too large.

% To create a PDF output that is optimized for double-sided printing: 
%
% 1) comment-out the \documentclass statement in the preamble below, and
% un-comment the second \documentclass line.
%
% 2) change the value assigned below to the boolean variable
% "PrintVersion" from "false" to "true".

% --------------------- Start of Document Preamble -----------------------

% Specify the document class, default style attributes, and page dimensions
% For hyperlinked PDF, suitable for viewing on a computer, use this:
\documentclass[letterpaper,12pt,titlepage,oneside,final]{book}
 
% For PDF, suitable for double-sided printing, change the PrintVersion variable below
% to "true" and use this \documentclass line instead of the one above:
% \documentclass[letterpaper,12pt,titlepage,openright,twoside,final]{book}


% This package allows if-then-else control structures.
\usepackage{ifthen}
\newboolean{PrintVersion}
%\setboolean{PrintVersion}{false} 
% \setboolean{PrintVersion}{true} 
% CHANGE THIS VALUE TO "true" as necessary, to improve printed results 
% for hard copies by overriding some options of the hyperref package.


% Load your needed packages and other commands of yours.

% Load your needed packages and other commands of yours here:
%\usepackage{} % ... note that old .sty files can be included here
\usepackage[]{inputenc}
\usepackage[T1]{fontenc}
\usepackage{fullpage}

\usepackage{coqdoc}
\usepackage{color}
\usepackage{listings}
\usepackage{etoolbox}
\usepackage{fixltx2e}
\usepackage{syntax}
\usepackage{lstcoq}
\usepackage{url}
\usepackage{acro}

\definecolor{maroon}{rgb}{0.5,0,0}
\definecolor{darkgreen}{rgb}{0,0.5,0}
\lstdefinelanguage{XML}
{
  basicstyle=\ttfamily,
  frame=single,
  breaklines=true,
  morestring=[s]{"}{"},
  morecomment=[s]{?}{?},
  morecomment=[s]{!--}{--},
  commentstyle=\color{darkgreen},
  moredelim=[s][\color{black}]{>}{<},
  moredelim=[s][\color{red}]{\ }{=},
  stringstyle=\color{blue},
  identifierstyle=\color{maroon}
}
\lstdefinelanguage{Pucella2006}
{
  morekeywords={
    agreement, prin, asset, with, prePay, and, display, print, count
  }
}
\lstdefinelanguage{selinux}
{
  morekeywords={
    type, types, attrib, role, allow, user, constrain, avkind, 
	sourcetype, targettype, object-class, perm, allow, auditallow, dontaudit, neverallow,
	constrain, classes, perms, sourcetype, sourcerole, sourceuser, targettype, targetrole, targetuser
  }
}

\lstdefinelanguage{AST}
{
  basicstyle=\ttfamily,
  breaklines=true,
  morekeywords={
    agreement, prin, asset, subject, policySet, policy, act, policyId, preRequisite 
  }
  tabsize=1,
}

\newtheorem{innercustomthm}{Theorem}
\newenvironment{customthm}[1]
  {\renewcommand\theinnercustomthm{#1}\innercustomthm}
  {\endinnercustomthm}












%--------------------------------------------------------------------------
% Do NOT edit the rest of the preample UNLESS YOU KNOW WHAT YOU'RE DOING!
%--------------------------------------------------------------------------

\ifthenelse{\boolean{PrintVersion}}{
\usepackage[top=1in,bottom=1in,left=0.75in,right=1.25in]{geometry}   % For twoside document
}{
\usepackage[top=1in,bottom=1in,left=0.75in,right=1.25in]{geometry}   % For oneside document
}

\usepackage{amsmath,amssymb,amstext} % Lots of math symbols and environments
\usepackage{graphicx} % For including graphics 

\usepackage{nomentbl} 
\makenomenclature 

\usepackage{ifpdf}

\setcounter{secnumdepth}{4}
\setcounter{tocdepth}{4}

%\newcommand{\href}[1]{#1} % does nothing, but defines the command so the
    % print-optimized version will ignore \href tags (redefined by hyperref pkg).
%\newcommand{\texorpdfstring}[2]{#1} % does nothing, but defines the command
% Anything defined here may be redefined by packages added below...


% Hyperlinks make it very easy to navigate an electronic document.
% In addition, this is where you should specify the thesis title
% and author as they appear in the properties of the PDF document.
% Use the "hyperref" package 
% N.B. HYPERREF MUST BE THE LAST PACKAGE LOADED; ADD ADDITIONAL PKGS ABOVE
%\usepackage[\ifpdf pdftex,\fi letterpaper=true,pagebackref=false]{hyperref} % with basic options
		% N.B. pagebackref=true provides links back from the References to the body text. This can cause trouble for printing.
%\hypersetup{
%    plainpages=false,       % needed if Roman numbers in frontpages
%    pdfpagelabels=true,     % adds page number as label in Acrobat's page count
%    bookmarks=true,         % show bookmarks bar?
%    unicode=false,          % non-Latin characters in Acrobat's bookmarks
%    pdftoolbar=true,        % show Acrobat's toolbar?
%    pdfmenubar=true,        % show Acrobat's menu?
%    pdffitwindow=false,     % window fit to page when opened
%    pdfstartview={FitH},    % fits the width of the page to the window
%%    pdftitle={uOttawa\ LaTeX\ Thesis\ Template},    % title: CHANGE THIS TEXT!
%%    pdfauthor={Author},    % author: CHANGE THIS TEXT! and uncomment this line
%%    pdfsubject={Subject},  % subject: CHANGE THIS TEXT! and uncomment this line
%%    pdfkeywords={keyword1} {key2} {key3}, % list of keywords, and uncomment this line if desired
%    pdfnewwindow=true,      % links in new window
%    colorlinks=true,        % false: boxed links; true: colored links
%    linkcolor=blue,         % color of internal links
%    citecolor=green,        % color of links to bibliography
%    filecolor=magenta,      % color of file links
%    urlcolor=cyan           % color of external links
%}

%\ifthenelse{\boolean{PrintVersion}}{   % for improved print quality, change some hyperref options
%\hypersetup{	% override some previously defined hyperref options
%%    colorlinks,%
%    citecolor=black,%
%    filecolor=black,%
%    linkcolor=black,%
%    urlcolor=black}
%}{} % end of ifthenelse (no else)



\DeclareAcronym{drm}
{
  short = DRM ,
  long  = Digital Rights Management ,
  class = abbrev
}

\DeclareAcronym{odrl}
{
  short = ODRL ,
  long  = Open Digital Rights Language ,
  class = abbrev
}

\DeclareAcronym{xrml}
{
  short = XrML ,
  long  = eXtensible rights Markup Language ,
  class = abbrev
}
\DeclareAcronym{rel}
{
  short = REL ,
  long  = Rights Expression Languages ,
  class = abbrev
}
\DeclareAcronym{drel}
{
  short = DREL ,
  long  = Digital Rights Expression Languages ,
  class = abbrev
}
\DeclareAcronym{cic}
{
  short = CIC ,
  long  = Calculus of Inductive Constructions ,
  class = abbrev
}
\DeclareAcronym{coc}
{
  short = CoC ,
  long  = Calculus of Constructions ,
  class = abbrev
}
\DeclareAcronym{xacml}
{
  short = XACML ,
  long  = eXtensible Access Control Markup Language ,
  class = abbrev
}
\DeclareAcronym{selinux}
{
  short = SELinux ,
  long  = Security Enhanced Linux ,
  class = abbrev
}
\DeclareAcronym{DAC}
{
  short = DAC ,
  long  = Discretionary access control ,
  class = abbrev
}
\DeclareAcronym{MAC}
{
  short = MAC ,
  long  = Mandatory access control ,
  class = abbrev
}
\DeclareAcronym{RBAC}
{
  short = RBAC ,
  long  = role-based access control ,
  class = abbrev
}
\DeclareAcronym{rhs}
{
  short = rhs ,
  long  = right hand side ,
  class = abbrev
}
\DeclareAcronym{lhs}
{
  short = lhs ,
  long  = left hand side ,
  class = abbrev
}




% This is where thesis margins and spaces are set.
% Setting up the page margins...
% A minimum of 1 inch (72pt) margin at the
% top, bottom, and outside page edges and a 1.125 in. (81pt) gutter
% margin (on binding side). While this is not an issue for electronic
% viewing, a PDF may be printed, and so we have the same page layout for
% both printed and electronic versions, we leave the gutter margin in.
% Set margins:
\setlength{\marginparwidth}{0pt} % width of margin notes
% N.B. If margin notes are used, you must adjust \textwidth, \marginparwidth
% and \marginparsep so that the space left between the margin notes and page
% edge is less than 15 mm (0.6 in.)
\setlength{\marginparsep}{0pt} % width of space between body text and margin notes
\setlength{\evensidemargin}{0.125in} % Adds 1/8 in. to binding side of all 
% even-numbered pages when the "twoside" printing option is selected
\setlength{\oddsidemargin}{0.125in} % Adds 1/8 in. to the left of all pages
% when "oneside" printing is selected, and to the left of all odd-numbered
% pages when "twoside" printing is selected
\setlength{\textwidth}{6.375in} % assuming US letter paper (8.5 in. x 11 in.) and 
% side margins as above
\raggedbottom

% The following statement specifies the amount of space between
% paragraphs. Other reasonable specifications are \bigskipamount and \smallskipamount.
\setlength{\parskip}{\medskipamount}

% The following statement controls the line spacing.  The default
% spacing corresponds to good typographic conventions and only slight
% changes (e.g., perhaps "1.2"), if any, should be made.
\renewcommand{\baselinestretch}{1} % this is the default line space setting

% By default, each chapter will start on a recto (right-hand side)
% page.  We also force each section of the front pages to start on 
% a recto page by inserting \cleardoublepage commands.
% In many cases, this will require that the verso page be
% blank and, while it should be counted, a page number should not be
% printed.  The following statements ensure a page number is not
% printed on an otherwise blank verso page.
\let\origdoublepage\cleardoublepage
\newcommand{\clearemptydoublepage}{%
  \clearpage{\pagestyle{empty}\origdoublepage}}
\let\cleardoublepage\clearemptydoublepage



%======================================================================
%   L O G I C A L    D O C U M E N T -- the content of your thesis
%======================================================================
\begin{document}

% For a large document, it is a good idea to divide your thesis
% into several files, each one containing one chapter.
% To illustrate this idea, the "front pages" (i.e., title page,
% declaration, borrowers' page, abstract, acknowledgements,
% dedication, table of contents, list of tables, list of figures,
% nomenclature).
%----------------------------------------------------------------------
% FRONT MATERIAL
%----------------------------------------------------------------------
%
% C O V E R  P A G E
% ------------------
\newcommand{\thesisauthor}{Bahman Sistany}
\newcommand{\thesisadvisor}{Amy Felty}
\newcommand{\thesistitlecoverpage}{%
  Certifying Digital Rights' Expression Languages
}
\newcommand{\degree}{Ph.D.} % possible values are:
                            
\newcommand{\nameofprogram}{Computer Science}
\newcommand{\academicunit}{School of Electrical Engineering and Computer Science}
\newcommand{\faculty}{Faculty of Engineering}
\newcommand{\graduationyear}{2014}
%

% T I T L E   P A G E
% -------------------
% Last updated May 24, 2011, by Stephen Carr, IST-Client Services
% The title page is counted as page `i' but we need to suppress the
% page number.  We also don't want any headers or footers.
\pagestyle{empty}
\pagenumbering{roman}

% The contents of the title page are specified in the "titlepage"
% environment.
\begin{titlepage}
        \begin{center}
        \vspace*{1.0cm}

        \Huge
        {\bf \thesistitlecoverpage }

        \vspace*{1.0cm}

        \normalsize
        by \\

        \vspace*{1.0cm}

        \Large
        \thesisauthor \\

        \vspace*{3.0cm}

        \normalsize
        \degree Thesis Proposal 
        For the \degree~degree in\\
        \nameofprogram\\

        \vspace*{2.0cm}

        \academicunit\\
        \faculty\\
        University of Ottawa\\

        \vspace*{1.0cm}
        %\copyright~\thesisauthor, Ottawa, Canada, \graduationyear\\
        \degree Thesis Advisor: ~\thesisadvisor
        \end{center}
\end{titlepage}

% The rest of the front pages should contain no headers and be numbered using Roman numerals starting with `ii'
\pagestyle{plain}
\setcounter{page}{2}

\cleardoublepage % Ends the current page and causes all figures and tables that have so far appeared in the input to be printed.
% In a two-sided printing style, it also makes the next page a right-hand (odd-numbered) page, producing a blank page if necessary.

%
%
% R E S T  O F  F R O N T  P A G E S
% ----------------------------------
% % D E C L A R A T I O N   P A G E
% -------------------------------
  % This page is not needed for a uOttawa thesis. Don't include it.
  % It is designed for an electronic thesis.
  \noindent
I hereby declare that I am the sole author of this thesis. This is a true copy of the thesis, including any required final revisions, as accepted by my examiners.

  \bigskip
  
  \noindent
I understand that my thesis may be made electronically available to the public.

\cleardoublepage
%\newpage
 %This is not needed in a uOttawa thesis.
%
% Edit the following 3 files with your abstract, acknowledgements, 
% and dedication.
% A B S T R A C T
% ---------------

\begin{center}\textbf{Abstract}\end{center}




\cleardoublepage
%\newpage

% A C K N O W L E D G E M E N T S
% -------------------------------

\begin{center}\textbf{Acknowledgements}\end{center}

I would like to thank all the little people who made this possible.


\cleardoublepage
%\newpage
% D E D I C A T I O N
% -------------------

\begin{center}\textbf{Dedication}\end{center}




\cleardoublepage
%\newpage

%
%
% No need to edit this file.
% T A B L E   O F   C O N T E N T S
% ---------------------------------
\renewcommand\contentsname{Table of Contents}
\tableofcontents
\cleardoublepage
\phantomsection
%\newpage

% L I S T   O F   T A B L E S
% ---------------------------
%\addcontentsline{toc}{chapter}{List of Tables}
%\listoftables
%\cleardoublepage
%\phantomsection		% allows hyperref to link to the correct page
%\newpage

% L I S T   O F   F I G U R E S
% -----------------------------
%\addcontentsline{toc}{chapter}{List of Figures}
%\listoffigures
%\cleardoublepage
%\phantomsection		% allows hyperref to link to the correct page
%\newpage

% L I S T   O F   LISTINGS
% -----------------------------
\addcontentsline{toc}{chapter}{List of Listings}
\lstlistoflistings
\cleardoublepage
\phantomsection		% allows hyperref to link to the correct page
\newpage

%
% No need to edit this file. But you may want to comment the whole line if you
% don't have or want a Nomenclature section.
%% L I S T   O F   S Y M B O L S
% -----------------------------
% To include a Nomenclature section
\addcontentsline{toc}{chapter}{\textbf{Nomenclature}}

\renewcommand{\nomname}{Nomenclature}
\renewcommand{\nomAname}{\textbf{\large Abbreviations}}
\renewcommand{\nomGname}{\textbf{\large Mathematical Symbols}}
\renewcommand{\nomXname}{\textbf{\large Superscripts}}
\renewcommand{\nomZname}{\textbf{\large Subscripts}}

\printnomenclature
\cleardoublepage
\phantomsection % allows hyperref to link to the correct page
% \newpage






%%% Local Variables: 
%%% mode: latex
%%% TeX-master: "../uottawa-thesis"
%%% End:   


% Change page numbering back to Arabic numerals
\pagenumbering{arabic}

%----------------------------------------------------------------------
% MAIN BODY
%---------------------------------------------------------------------- 
% Chapters 
% Include your "sub" source files here (must have extension .tex)
%======================================================================
\chapter{Introduction}
%======================================================================
Digital rights management, \emph{DRM}, refers to the digital management of rights associated with the access or usage of digital assets. There are various aspects of rights management however. According to the authors of the whitepaper ``A digital rights management ecosystem model for the education community,'' digital rights management systems cover the following four areas: 1) defining rights 2) distributing/acquiring rights 3) enforcing rights and 4) tracking usage \cite{collier}. 

Rights Expression Languages, \emph{RELs}, or more precisely when dealing with digital assets, Digital Rights Expression Languages \emph{DRELs} deal with the ``rights definition'' aspect of the DRM ecosystem. A DREL, allows the expression and definition of digital asset usage rights such that other areas of the DRM ecosystem namely the enforcement mechanism and the usage tracking components can function correctly.

Currently the most popular RELs are the eXtensible rights Markup Language, \emph{XrML} [bib], and the Open Digital Rights Language, \emph{ODRL} [bib]. Both of these languages are XML based and are considered declarative languages. XrML has been selected to be the REL for \emph{MPEG-21} which is an ISO standard for multimedia applications. ODRL is also a standards based REL which has been accepted as part of the W3C community with the mandate of standardizing how rights and policies, related to the usage of digital content on the Open Web Platform, \emph{OWP}, are expressed [wikipedia]. ODRL 2.0 supports expression of rights and also privacy rules for social media while ODRL 1.0 was only dealing with the mobile ecosystem -- ODRL 1.0 was adopted by the Open Mobile Alliance, \emph{OMA} in 2000.

As popular as both XrML and ODRL are, their adoption and usage is still somewhat limited in practice. Both Apple and Microsoft for example have defined their own lightweight RELs [problem with RELs paper] in \emph{Fair Play} (Apple) and in \emph{PlayReady} (Microsoft). The authors of [the problem with RELs] argue that both these RELs and other ones are simply too complex to be used effectively since they try to cover much of the DRM ecosystem. 

Another issue with the current batch of RELs are due to their semantics being expressed in a natural language (e.g. English). By necessity natural languages are ambiguous and open to interpretation. 

To formalize the semantics of RELS several approaches have been attempted by various authors. The main categories are logic based, operational semantics based interpreters and finally web ontology based (from the Knowledge Representation Field). In this thesis we will focus on the logic based approach to formalizing semantics and will study a specific logic based language that is a translation from a subset of ODRL.









%----------------------------------------------------------------------
\section{Logic Based Semantics for ODRL}
%----------------------------------------------------------------------

%See equation \ref{eqn_pi} on page \pageref{eqn_pi}.\footnote{A famous %equation.}

Formal logic can represent the statements and facts we express in a natural language like English. Propositional logic is expressive enough to express simple facts as propositions and allows uses connectives to allow for the negation, conjunction and disjunction of the facts. However propositional logic is not expressive enough to express policies of the kind used in languages like ODRL and XrML. For example, a simple policy expressed in English like ``All who pay 5 dollars can watch the movie Toy Story'' cannot be expressed in propositional logic because the concept of  variables doesn't exist. 

The higher order logic called ``Predicate Logic'' or ``First Order Logic'' \emph{FOL} is more suitable and has the expressive power to represent policies written in English. Moreover, FOL can be used to capture the meaning of policies in an unambiguous way.

Halpern and Weissman [Using First Order Logic to Reason about Policies] propose a fragment of FOL to represent and reason about policies. The fragment of FOL they arrive at is called \emph{Lithium} which is decidable and allows for efficiently answering interesting queries. Lithium restricts policies to be written based on the concept of ``bipolarity'' which disallows by construction policies that both permit and deny an action on an object.

\section{Pucella 2006}
Pucella and Weissman \cite{pucella2006} specify a predicate logic based based language that represents a subset of ODRL.

\section{what will I do?}


\subsection{Coq}


	• Program correctness
	• Formal verification of software
	• Certified programs
	• Proof assistant
	• Interactive and mechanized theorem proving
	• Examples of machine assisted proofs: CompCert, four-color theorem proof
	• Coq is based on a higher-order functional programming language
	• Dependent Types
		○ Subset types
		○ Easier than writing explicit proofs
	• Write formal specification and proofs that programs comply to their specification (a-short-intro-to-coq)
	• Automatically extract code from specifications as Ocaml or Haskell (a-short-intro)
	• Properties, programs and proofs are all formalized in the same language called CIC (Calculus of inductive Constructions). (a-short-intro)
	• Coq uses a sort called Prop for propositions
	• Coq art:
	• Well-formed propositions are assertions  one can express about values such as mathematical objects or even programs e.g. 3 < 8
		○ Note that assertions may be true, false or simply conjectures
		○ An assertion is only true in general if a proof is provided
		○ However hand written proofs are difficult to verify
		○ Coq provides an environment for developing proofs including a formal language to express proofs in, the language itself being built using proof theory making it possible to step by step verification of the proofs
		○ Mechanized proof verification requires a "proof" that the verification algorithm is correct itself in applying all the formal rules correctly







\section{Abstract Syntax}

\cite{pucella2006} uses abstract syntax instead of XML to express statements in the ODRL language. The abstract syntax used is a more compact representation than XML based language ODRL policies are written in and furthermore it simplifies specifying the semantics as we shall see. As an example here is an agreement written in ODRL and the comparable agreement expressed in the abstract syntax \cite{pucella2006}.

\lstset{language=XML}
\begin{lstlisting}[caption={agreement for Mary Smith in XML},label={lst:agreementxml}]
<agreement>
 <asset> <context> <uid> Treasure Island </uid> </context> </asset>
 <permission>
   <display>
    <constraint>
     <cpu> <context> <uid> Mary's computer </uid> </context> </cpu>
    </constraint>
   </display>
   <print>
    <constraint> <count> 2 </count> </constraint>
   </print>
  <requirement>
   <prepay>
    <payment> <amount currency="AUD"> 5.00</amount> </payment>
   </prepay>
  </requirement>
 </permission>
 <party> <context> <name> Mary Smith </name> </context> </party>
</agreement>
\end{lstlisting}

The agreement ~\ref{lst:agreementxml} is shown below using the syntax from \cite{pucella2006}.

\lstset{language=Pucella2006}
\begin{minipage}[c]{0.95\textwidth}
\begin{lstlisting}[frame=single, caption={agreement for Mary Smith as BNF (as used in ~\cite{pucella2006})},label={lst:agreementpucella2006}]
agreement
 for Mary Smith 
 about Treasure Island 
 with prePay[5.00] -> and[cpu[Mary's Computer] => display,
                                      count[2] => print].
\end{lstlisting}
\end{minipage} 

% \emph{prin\textsubscript{u}}
In the following we will cover the \emph{abstract syntax} of a subset of ODRL expressed as Coq's constructs such as \emph{Inductive Types} and Definitions. We will call this subset \emph{ODRL0} both because it is a variation of Pucella's ODRL language and also because it is missing some ODRL constructs such as \emph{Requirements} and \emph{Conditions} - we will add the missing pieces making up what we will call \emph{ODRL1} and perhaps \emph{ODRL2} (the latter only if needed). We will also describe ODRL0 in a \emph{BNF} grammar that looks more like Pucella's ODRL grammar. BNF style grammars are less formal as they give some suggestions about the surface syntax of expressions [Pierce1] without getting into lexical analysis and parsing related aspects such as precedence order of operators. The Coq version in contrast is more formal and could be directly used for building compilers and interpreters. We will present both the BNF version and the Coq version for each construct of ODRL0 [Pierce1]. To get started let's see what the listing ~\ref{lst:agreementpucella2006} would look like in ODRL0's Coq version.

\lstset{language=Coq}
\begin{lstlisting}[frame=single, caption={Coq version of agreement for Mary Smith},label={lst:marysmithagreementcoq}]

Agreement (Single MarySmith) Treasure Island 
 (PrimitivePolicySet (Constraint (PrePay 5.00))
  (AndPolicy 
   (NewList (PrimitivePolicy (Constraint 
                              (Principal 
                               (Single MarysComputer))) id1 Display)
   (Single (PrimitivePolicy (Constraint (Count 2)) id2 Print))))).
\end{lstlisting}


%\coqdoceol
%\coqdocvar{Agreement} (\coqdocvar{Single} \coqdocvar{Mary Smith}) \coqdocvar{Treasure Island} (\coqdocvar{PrimitivePolicySet} (\coqdocvar{Constraint} (\coqdocvar{PrePay} 5.00))\coqdoceol
%\coqdocindent{0.50em}
%(\coqdocvar{AndPolicy} (\coqdocvar{NewList} (\coqdocvar{PrimitivePolicy} (\coqdocvar{Constraint} (\coqdocvar{Principal} (\coqdocvar{Single} \coqdocvar{Mary's Computer}))) \coqdocvar{id1} \coqdocvar{Display})\coqdoceol
%\coqdocindent{6.00em}
%(\coqdocvar{Single} (\coqdocvar{PrimitivePolicy} (\coqdocvar{Constraint} (\coqdocvar{Count} 2)) \coqdocvar{id2} \coqdocvar{Print}))))).\coqdoceol
%\coqdocemptyline
%\coqdocemptyline

The top level ODRL0 production is the \emph{agreement}. An agreement expresses what actions a set of subjects may perform on an object and under what conditions. Syntactically an agreement is composed of a set of subjects/users called a \emph{principal} (\emph{prin}), an \emph{asset} and a \emph{Policy Set} (\emph{PolicySet}).

% agreement
\lstset{language=AST}
\begin{lstlisting}[frame=single, caption={agreement},label={lst:agreementast}]
<agreement> ::= 'agreement' 'for' <prin> 'about' <asset> 'with' <policySet> 
\end{lstlisting}

Principals or prins are composed of \emph{subjects} which are specified based on the application e.g. Alice, Bob, etc for the DRM application we will be using throughout.

% prin
\lstset{mathescape, language=AST}  
\begin{lstlisting}[frame=single, caption={prin},label={lst:prinast}]
<prin> ::=  { <subject$_{1}$>, ..., <subject$_{m}$> }
\end{lstlisting}

% subject
\lstset{mathescape, language=AST}  
\begin{lstlisting}[frame=single, caption={subject},label={lst:subjectast}]
<subject> ::= N
\end{lstlisting}

Assets are also application specific but similar to subjects we will use specific ones for the DRM application (taken from \cite{pucella2006}). \emph{ebook}, \emph{The Report} and \emph{latestJingle} are examples of specific subjects we will be using throughout. Syntactically an asset is just a positive number (\emph{N}).

% asset
\lstset{mathescape, language=AST}  
\begin{lstlisting}[frame=single, caption={asset},label={lst:assetast}]
<asset> ::= N
\end{lstlisting}

Agreements include policy sets. Each policy set specifies a \emph{prerequisite} and a \emph{policy}. In general if the prerequisite holds the policy is taken into consideration. Otherwise the policy will not be looked at. Some policy sets are specified as \emph{exclusive}. The \emph{Primitive Exclusive Policy Sets} are exclusive to agreement's users in that only those users may perform the actions specified in the policy set. The implication is that all other users who are not specified in the agreement's principal (prin) are forbidden from performing the specified actions. Finally policy sets could be grouped together in a \emph{conjunction} allowing a single agreement to be associated with many policy sets. 


% policySet

\lstset{mathescape, language=AST}  
\begin{lstlisting}[frame=single, caption={policySet},label={lst:policySetast}]
<policySet> ::=  
	$\vert$ <PrimitivePolicySet> : <preRequisite> $\rightarrow$ <policy> 
	$\vert$ <PrimitiveExclusivePolicySet> : <preRequisite> $\mapsto$ <policy>	 
	$\vert$ <AndPolicySet> : 'and'[ <policySet$_{1}$>, ..., <policySet$_{m}$> ]
\end{lstlisting}

A policy specifies an action to be performed on an asset, depending of whether the policy's prerequisite holds or not. If the prerequisite holds the agreement's user is permitted to perform the action on the agreement's asset; otherwise permission is denied. Similar to policy sets, policies could also be grouped together in a conjunction. The policy also includes a unique identifier. The policy identifier is added to help the translation (from agreements to formulas) but is optional in ODRL proper.

% policy

\lstset{mathescape, language=AST}  
\begin{lstlisting}[frame=single, caption={policy},label={lst:policyast}]
<policy> ::=  
	$\vert$ <PrimitivePolicy> : <preRequisite> $\Rightarrow_{<policyId>}$ <act>
	$\vert$ <AndPolicy> : 'and'[ <policy$_{1}$>, ..., <policy$_{m}$> ]
\end{lstlisting}

An \emph{Action} (\emph{act}) is simply a positive number. Similar to assets and subjects, actions are application specific. Some example actions taken from \cite{pucella2006} are \emph{Display} and \emph{Print}.

% act
\lstset{mathescape, language=AST}  
\begin{lstlisting}[frame=single, caption={act},label={lst:actast}]
<act> ::= N
\end{lstlisting}

A \emph{Policy Id} (\emph{policyId}) is a unique identifier specified as (increasing) positive integers. 

% id
\lstset{mathescape, language=AST}  
\begin{lstlisting}[frame=single, caption={policyId},label={lst:policyIdast}]
<policyId> ::= N
\end{lstlisting}

In ODRL0 a \emph{prerequisite} is either true or it is a \emph{constraint}. The \emph{true} prerequisite always holds. A constraint is an intrinsic part of a policy and cannot be influenced by agreement's user. Minimum height requirements for popular attractions and rides are examples of we would consider a constraint. The constraint \emph{ForEachMember} is interesting in its expressive power but has complicated semantics as we shall see in the ~\ref{sec:Semantics} section. Roughly speaking, ForEachMember takes a prin (a list of subjects) and a list L of constraints. The ForEachConstraint holds if each subject in prin satisfies each constraint in L.\emph{NotCons} is a negation of a constraint. The set of prerequisites are closed under conjunction (\emph{AndPrqs}), disjunction (\emph{OrPrqs}) and exclusive disjunction (\emph{XorPrqs}).

% prq

\lstset{mathescape, language=AST}  
\begin{lstlisting}[frame=single, caption={preRequisite},label={lst:preRequisiteast}]
<preRequisite> ::=  
	$\vert$ <TruePrq> : 'True'
	$\vert$ <Constraint> : <constraint>	 
	$\vert$ <ForEachMember> : 'ForEachMember' [<prin> ; <constraint$_{1}$>, ..., <constraint$_{m}$> ]	
	$\vert$ <NotCons> : 'not' [ <constraint> ]
	$\vert$ <AndPrqs> : 'and'[ <preRequisite$_{1}$>, ..., <preRequisite$_{m}$> ]
	$\vert$ <OrPrqs> : 'or'[ <preRequisite$_{1}$>, ..., <preRequisite$_{m}$> ]
	$\vert$ <XorPrqs> : 'xor'[ <preRequisite$_{1}$>, ..., <preRequisite$_{m}$> ]		
\end{lstlisting}

Constraints are either \emph{Principal}, \emph{Count} or \emph{CountByPrin}. Principal constraints basically require matching to specified prins. For example, the user being Alice is a Principal constraint. A count constraint refers to a set of policies \emph{P} and specifies the number of times the user of an agreement has invoked the policies in P to justify her actions. If the count constraint is part of a policy then the set P is composed of the single policy. In the case that the count constraint is part of a policy set, the set P is the set of policies specified in the policy set.

% constraint
\lstset{mathescape, language=AST}  
\begin{lstlisting}[frame=single, caption={constraint},label={lst:constraintast}]
<constraint> ::=  
	$\vert$ <Principal> : <prin>
	$\vert$ <Count> : 'Count' [N]
	$\vert$ <CountByPrin> : <prin> ('Count' [N])
\end{lstlisting}

% ---------------------------------- COQ -----------------------
\section{Coq Version}

\lstset{language=Coq}
\begin{lstlisting}[frame=single, caption={Coq version of agreement},label={lst:agreementcoq}]
Inductive agreement : Set :=
  | Agreement : prin -> asset -> policySet -> agreement.
\end{lstlisting}

% prin 
\lstset{language=Coq}
\begin{lstlisting}[frame=single, caption={prin},label={lst:princoq}]
Definition prin := nonemptylist subject.
\end{lstlisting}

% asset
\lstset{language=Coq}
\begin{lstlisting}[frame=single, caption={asset},label={lst:assetcoq}]
Definition asset := nat.
\end{lstlisting}



% subject
\lstset{language=Coq}
\begin{lstlisting}[frame=single, caption={subject},label={lst:subjectcoq}]
Definition subject := nat.
\end{lstlisting}


% policySet
\lstset{language=Coq}
\begin{lstlisting}[frame=single, caption={policySet},label={lst:policySetcoq}]
Inductive policySet : Set :=
  | PrimitivePolicySet : preRequisite -> policy -> policySet 
  | PrimitiveExclusivePolicySet : preRequisite -> policy  -> policySet 
  | AndPolicySet : nonemptylist policySet -> policySet.
\end{lstlisting}

% policy
\lstset{language=Coq}
\begin{lstlisting}[frame=single, caption={policy},label={lst:policycoq}]
Inductive policy : Set :=
  | PrimitivePolicy : preRequisite -> policyId -> act -> policy 
  | AndPolicy : nonemptylist policy -> policy.
\end{lstlisting}

% act
\lstset{language=Coq}
\begin{lstlisting}[frame=single, caption={act},label={lst:actcoq}]
Definition act := nat.
\end{lstlisting}

% id
\lstset{language=Coq}
\begin{lstlisting}[frame=single, caption={policyId},label={lst:policyIdcoq}]
Definition policyId := nat.
\end{lstlisting}

% prq
\lstset{language=Coq}
\begin{lstlisting}[frame=single, caption={preRequisite},label={lst:preRequisitecoq}]
Inductive preRequisite : Set :=
  | TruePrq : preRequisite
  | Constraint : constraint -> preRequisite 
  | ForEachMember : prin  -> nonemptylist constraint -> preRequisite
  | NotCons : constraint -> preRequisite 
  | AndPrqs : nonemptylist preRequisite -> preRequisite
  | OrPrqs : nonemptylist preRequisite -> preRequisite
  | XorPrqs : nonemptylist preRequisite -> preRequisite.
\end{lstlisting}

% constraint
\lstset{language=Coq}
\begin{lstlisting}[frame=single, caption={constraint},label={lst:constraintcoq}]
Inductive constraint : Set :=
  | Principal : prin  -> constraint 
  | Count : nat -> constraint 
  | CountByPrin : prin -> nat -> constraint.
\end{lstlisting}


%----------------------------------------------------------------------
\section{Semantics}
\label{sec:Semantics}

In this section, we describe the semantics of ODRL0 language by a translation from each language object (e.g. agreement) to a proposition in \emph{Coq}. The semantics will help answer queries of the form ``may subject \emph{s} perform action \emph{act} to asset \emph{a}?''. If the answer is yes, we say permission is granted. Otherwise permission is denied. 

Whether a permission is granted or denied depends on the agreements in question but also on the facts recorded in the environment. For ODRL0 those facts revolve around the number of times a policy has been used to justify an action. We encode this information in an \emph{environment} which is a conjunction of equalities of the form \emph{count(s, policyId) = n}. 

The Coq version of the count equality is a new inductive type called \emph{count_equality}. An environment is defined to be a non-empty list of count_equality objects.

\lstset{language=Coq, frame=single, caption={Environments and Counts},label={lst:environmentcoq}}
\begin{lstlisting}
Inductive count_equality : Set := 
   | CountEquality : subject -> policyId -> nat -> count_equality.

Inductive environment : Set := 
  | SingleEnv : count_equality -> environment
  | ConsEnv :  count_equality ->  environment -> environment.

\end{lstlisting}


The translation starts with the top level agreement element and proceeds by case analysis on the structure of the agreement. Note that each translation function takes an environment parameter.


\lstset{language=Coq, frame=single, caption={Translation of agreement},label={lst:transagreement}}
\begin{lstlisting}

Definition trans_agreement (e:environment)(ag:agreement) : Prop :=
  match ag with 
    | Agreement prin_u a ps => trans_ps e ps prin_u a
  end.

\end{lstlisting}

Translation of a policy set proceeds with case analysis of different Policy Set constructors. We then recurse into translation functions for the composing elements. The specific Coq propositions for each constructor is taken from the formula translation for each constructor defined in \cite{pucella2006}.

\lstset{mathescape, language=AST}  
\begin{lstlisting}[frame=single, caption={Policy Translation As Formulas},label={lst:transpolicyformula}]
$[\![ prq \rightarrow p]\!]^{prin_{u}, a}$ $\triangleq$ $\forall$ $(\!( [\![prin_{u}]\!]_{x} \land [\![prq]\!]_{x})\!)$
\end{lstlisting}





\lstset{language=Coq, frame=single, caption={Translation of Policy Set},label={lst:transps}}
\begin{lstlisting}

Fixpoint trans_ps
  (e:environment)(ps:policySet)(prin_u:prin)(a:asset){struct ps} : Prop :=

let trans_ps_list := (fix trans_ps_list (ps_list:nonemptylist policySet)(prin_u:prin)(a:asset){struct ps_list}:=
  match ps_list with
    | Single ps1 => trans_ps e ps1 prin_u a
    | NewList ps ps_list' => ((trans_ps e ps prin_u a) /\ (trans_ps_list ps_list' prin_u a))
  end) in
    match ps with
    | PrimitivePolicySet prq p => forall x, (((trans_prin x prin_u) /\ 
                                   (trans_preRequisite e x prq (getId p) prin_u)) -> 
                                   (trans_policy_positive e x p prin_u a))  

    | PrimitiveExclusivePolicySet prq p => forall x, ((((trans_prin x prin_u) /\ 
                                              (trans_preRequisite e x prq (getId p) prin_u)) -> 
                                             (trans_policy_positive e x p prin_u a)) /\
                                            ((not (trans_prin x prin_u)) -> (trans_policy_negative e x p a)))
                   
    | AndPolicySet ps_list => trans_ps_list ps_list prin_u a
    end.
\end{lstlisting}




 
%
%======================================================================
\chapter{Motivation}
%======================================================================

%----------------------------------------------------------------------
\section{Logic Based Semantics for ODRL}
%----------------------------------------------------------------------

%See equation \ref{eqn_pi} on page \pageref{eqn_pi}.\footnote{A famous %equation.}

Formal logic can represent the statements and facts we express in a natural language like English. Propositional logic is expressive enough to express simple facts as propositions and uses connectives to allow for the negation, conjunction and disjunction of the facts. However propositional logic is not expressive enough to express policies of the kind used in languages like ODRL and XrML. For example, a simple policy expressed in English like ``All who pay 5 dollars can watch the movie Toy Story'' cannot be expressed in propositional logic because the concept of  variables doesn't exist in propositional logic. 

A richer logic such as ``Predicate Logic'' or ``First Order Logic'' (\emph{FOL}) is more suitable and has the expressive power to represent policies written in English. Moreover, FOL can be used to capture the meaning of policies in an unambiguous way.

Halpern and Weissman [REF][Using First Order Logic to Reason about Policies] propose a fragment of FOL to represent and reason about policies. The fragment of FOL they arrive at is called \emph{Lithium} which is decidable and allows for efficiently answering interesting queries. Lithium restricts policies to be written based on the concept of ``bipolarity'' which disallows by construction policies that both permit and deny an action on an object. Pucella and Weissman ~\cite{pucella2006} specify a predicate logic based language that represents a subset of ODRL.


\section{Specific Problem}

Policy languages and the agreements written in those languages are meant to implement specific goals such as limiting access to specific assets. The tension in designing a policy language is usually between how to make the language expressive enough, such that the design goals for the policy language may be expressed, and how to make the policies verifiable with respect to the stated goals.

As stated earlier, an important part of fulfilling the verifiability goal is to have formal semantics defined for policy languages. For \ac{odrl}, authors of ~\cite{pucella2006} have defined a formal semantics based on which they declare and prove a number of important theorems (their main focus is on stating and proving algorithm complexity results). However as with many paper-proofs, the language used to do the proofs while mathematical in nature, uses a lot of intuitive justifications to show the proofs. As such these proofs are difficult to verify or more importantly to ``derive''. Furthermore the proofs can not be used directly to render a decision on a sample policy (e.g. whether to allow or deny access to an asset). Of course one may (carefully) construct a program based on these proofs for practical purposes but we will have no way of certifying those programs correct, even assuming the original proofs were in fact correct.

While there are paper-proofs for \ac{odrl}, as far as we know, similar paper-proofs do not exist for an important (mandatory) access-control policy system, namely \ac{selinux}[REF]. In particular no formal proofs (paper based or otherwise) of decidablity of \ac{selinux} policies exist in the literature. 


\section{Contributions}

In this thesis we will build a language representation framework based on \ac{odrl} and definitions in ~\cite{pucella2006}. The framework will in \emph{Coq} which is both a programming language and a proof-assistant. We will declare and prove decidablity results of subsets of \ac{odrl} all the way up to the complete \ac{odrl} fragement defined in ~\cite{pucella2006}. We will extract programs from the proofs and demonstrate how they can be used on specific policies to render a specific decision such as ``a conflict has been detected''. 

Beside ``certified decidablity results'' for \ac{odrl}, we will investigate decidablity for \ac{selinux} policies, proving decidablity or show why a proof is not possible (if that is the case) and provide proposals to make the policy language decidable.

By using the Coq framework originally built for \ac{odrl} to encode and verify agreements written in a second policy language (albeit a different class of policy language: \ac{rel} vs access-control) we will demonstrate the suitability of this Coq based framework for other policy languages such as \ac{xacml}[REF].

\section{What specific work has been accomplished until this point in time? what results were obtained so far?}

The encodings for a subset of \ac{odrl} called ODRL0 (see ~\ref{sec:odrl0}) plus some important functions implementing some of the algorithms in ~\cite{pucella2006} have been implemented in Coq. Some of the intermediate theorems have been also been defined and proved.

\section{What remains to be done to complete the thesis research?}
The main decidablity result and its proof for ODRL0 will be completed first. We will add the remaining \ac{odrl} constructs incrementally while maintaining decidablity for the main decision algorithm. The remaining constructs include a trouble-some construct (~\cite{pucella2006}), namely $not[policySet]$. We will show this construct does not change the decidablity result already established. 

ODRL0 is enough to be used as a basis for \ac{selinux} policies without \emph{constrain}s. \ac{selinux} constrains are extra  conditions that need to be satisfied (in addition to policies) in order for a permission to be granted. We will investigate decidablity for this subset first. We will then add constrains to the ODRL0 subset (as pre-requisites) and investigate decidablity.



\section{What is the timetable to complete the work?}

The study plan calls for August of 2015 for everything to be completed.
 
%======================================================================
\chapter{ODRL0 Syntax}
\label{chap:odrl0syntax}
%======================================================================

\section{Introduction}

Authors of~\cite{pucella2006} use abstract syntax instead of XML (ODRL 2.0 can also be encoded in \emph{JSON} and \emph{RDF/OWL Ontology}) to express statements in the \ac{odrl} language. Abstract syntax is a more compact representation than XML in which \ac{odrl} policies may be written in and furthermore abstract syntax simplifies specifying the semantics as we shall see later. As an example the agreement ``If Mary Smith pays five dollars, then she is allowed to print the eBook 'Treasure Island' twice and she is allowed to display it on her computer as many times as she likes'' written in \ac{odrl}'s XML encoding is illustrated in Listing~\ref{lst:agreementxml}~\cite{pucella2006}. 


\lstset{language=XML}
%\begin{minipage}[c]{0.95\textwidth}
\begin{lstlisting}[caption={Agreement for Mary Smith in XML},label={lst:agreementxml}]
<agreement>
 <asset> <context> <uid> Treasure Island </uid> </context> </asset>
 <permission>
   <display>
    <constraint>
     <cpu> <context> <uid> Mary's computer </uid> </context> </cpu>
    </constraint>
   </display>
   <print>
    <constraint> <count> 2 </count> </constraint>
   </print>
  <requirement>
   <prepay>
    <payment> <amount currency="AUD"> 5.00</amount> </payment>
   </prepay>
  </requirement>
 </permission>
 <party> <context> <name> Mary Smith </name> </context> </party>
</agreement>
\end{lstlisting}
%\end{minipage} 

The agreement in Listing~\ref{lst:agreementxml} is shown in~\ref{lst:agreementpucella2006} using the syntax from~\cite{pucella2006}.

\lstset{language=Pucella2006}
\begin{lstlisting}[frame=single, caption={Agreement for Mary Smith as BNF (as used in~\cite{pucella2006})},label={lst:agreementpucella2006}]
agreement
 for Mary Smith 
 about Treasure Island 
 with prePay[5.00] -> and[cpu[Mary's Computer] => display,
                                      count[2] => print].
\end{lstlisting}


% \emph{prin\textsubscript{u}}
In the following we will cover the \emph{abstract syntax} of a subset of ODRL that we later express using Coq's constructs such as \emph{Inductive Types} and Definitions. We will call this subset \emph{ODRL0} both because it is a variation of Pucella's ODRL language and also because it is missing some constructs from Pucella's ODRL (and hence from ODRL proper). 

\section{ODRL0}\label{sec:odrl0}
In ODRL0, agreements and facts (i.e. environments) will only contain the number of times each policy has been used to justify an action. In ODRL0 agreements and facts will not contain: \begin{enumerate}
  \item Which payments have been made
  \item Which acknowledgments have been made
\end{enumerate} 

This means \emph{Paid} and \emph{Attributed} predicates are not used in ODRL0. Also removed are related constructs \emph{prepay} and \emph{attribution}. We also had to remove two other constructs based on $prepay$ and $attribution$ out of ODRL0 in \emph{inSeq} and \emph{anySeq}. $prepay$, $attribution$, $inSeq$ and $anySeq$ make up what is called \emph{requirements} in ODRL.

In ODRL a \emph{prerequisite} is either $true$, a $constraint$, a $requirement$ or a $condition$. $true$ is the prerequisite that always holds. Constraints are facts that are outside of control of users. For example, there is nothing $Alice$ can do to satisfy the constraint ``user must be Bob''. Requirements are facts that are in users' control~\cite{pucella2006}. For example, $Alice$ may satisfy the requirement ``The user must pay 5 dollars''. Finally conditions are constraints that must not hold. The listing~\ref{lst:agreementpucella2006}'s Coq version is illustrated in listing~\ref{lst:marysmithagreementcoq}.


\lstset{language=Coq}
\begin{minipage}[c]{0.95\textwidth}
\begin{lstlisting}[frame=single, caption={Coq Version of Agreement for Mary Smith},label={lst:marysmithagreementcoq}]

Agreement (Single MarySmith) Treasure Island 
 (PrimitivePolicySet (Constraint (PrePay 5.00))
  (AndPolicy 
   (NewList (PrimitivePolicy (Constraint 
                              (Principal 
                               (Single MarysComputer))) id1 Display)
   (Single (PrimitivePolicy (Constraint (Count 2)) id2 Print))))).
\end{lstlisting}
\end{minipage}



In ODRL0, a prerequisite is either $true$, a $constraint$, or the negation of a $constraint$. So we have removed requirements from the picture and don't have explicit conditions. Conditions are replaced by a category called $NotCons$ directly in the production for prerequisites (see listing~\ref{lst:preRequisiteast}). Note that we have also removed the condition $not[policySet]$ from ODRL since the authors in~\cite{pucella2006} have shown the semantics of this component are not well-defined and including it leads to intractability results.

We will add the missing pieces as described above making up what we will call \emph{ODRL1} and perhaps \emph{ODRL2} (the latter only if needed). We will also describe ODRL0 in a \emph{BNF} grammar that looks more like Pucella's ODRL grammar~\cite{pucella2006}. BNF style grammars are less formal as they give some suggestions about the surface syntax of expressions without getting into lexical analysis and parsing related aspects such as precedence order of operators~\cite{piercesf2011}. The Coq version in contrast is more formal and could be directly used for building compilers and interpreters. We will present both the BNF version and the Coq version for each construct of ODRL0. 

%\coqdoceol
%\coqdocvar{Agreement} (\coqdocvar{Single} \coqdocvar{Mary Smith}) \coqdocvar{Treasure Island} (\coqdocvar{PrimitivePolicySet} (\coqdocvar{Constraint} (\coqdocvar{PrePay} 5.00))\coqdoceol
%\coqdocindent{0.50em}
%(\coqdocvar{AndPolicy} (\coqdocvar{NewList} (\coqdocvar{PrimitivePolicy} (\coqdocvar{Constraint} (\coqdocvar{Principal} (\coqdocvar{Single} \coqdocvar{Mary's Computer}))) \coqdocvar{id1} \coqdocvar{Display})\coqdoceol
%\coqdocindent{6.00em}
%(\coqdocvar{Single} (\coqdocvar{PrimitivePolicy} (\coqdocvar{Constraint} (\coqdocvar{Count} 2)) \coqdocvar{id2} \coqdocvar{Print}))))).\coqdoceol
%\coqdocemptyline
%\coqdocemptyline

\section{Productions} \label{sec:productionast}

The top level ODRL0 production is the \emph{agreement}. An agreement expresses what actions a set of subjects may perform on an object and under what conditions. Syntactically an agreement is composed of a set of subjects/users called a \emph{principal} or \emph{prin}, an \emph{asset} and a \emph{policySet}.

% agreement
\lstset{language=AST}
\begin{lstlisting}[frame=single, caption={agreement},label={lst:agreementast}]
<agreement> ::= 'agreement' 'for' <prin> 'about' <asset> 'with' <policySet> 
\end{lstlisting}

Principals or prins are composed of \emph{subjects} which are specified based on the application e.g. Alice, Bob, etc.

% prin
\lstset{mathescape, language=AST}  
\begin{lstlisting}[frame=single, caption={prin},label={lst:prinast}]
<prin> ::=  { <subject$_{1}$>, ..., <subject$_{m}$> }
\end{lstlisting}

% subject
\lstset{mathescape, language=AST}  
\begin{lstlisting}[frame=single, caption={subject},label={lst:subjectast}]
<subject> ::= N
\end{lstlisting}

Assets are also application specific but similar to subjects we will use specific ones for the DRM application (taken from \cite{pucella2006}). \emph{ebook}, \emph{The Report} and \emph{latestJingle} are examples of specific subjects we will be using throughout. Syntactically an asset is represented as a natural number (\emph{N}). Similarly for subjects.

% asset
\lstset{mathescape, language=AST}  
\begin{lstlisting}[frame=single, caption={asset},label={lst:assetast}]
<asset> ::= N
\end{lstlisting}

Agreements include policy sets. Each policy set specifies a \emph{prerequisite} and a \emph{policy}. In general if the prerequisite holds the policy is taken into consideration. Otherwise the policy will not be looked at. Some policy sets are specified as \emph{exclusive}. The \emph{Primitive Exclusive Policy Sets} are exclusive to agreement's users in that only those users may perform the actions specified in the policy set. The implication is that all other users who are not specified in the agreement's principal (prin) are forbidden from performing the specified actions. Finally policy sets could be grouped together in a \emph{conjunction} allowing a single agreement to be associated with many policy sets. 


% policySet
\newcommand*{\Comment}[1]{\hfill\makebox[7.0cm][l]{#1}}%
\lstset{mathescape, language=AST, escapechar=\&}  
\begin{lstlisting}[frame=single, caption={policySet},label={lst:policySetast}]
<policySet> ::=  
	<preRequisite> $\rightarrow$ <policy>	&\Comment{; primitive policy set }&
	<preRequisite> $\mapsto$ <policy>		&\Comment{; primitive exclusive policy set }&
	'and'[ <policySet$_{1}$>, ..., <policySet$_{m}$> ]	&\Comment{; conjunction }&
\end{lstlisting}

A policy specifies an action to be performed on an asset, depending of whether the policy's prerequisite holds or not. If the prerequisite holds the agreement's user is permitted to perform the action on the agreement's asset; otherwise permission is denied. Similar to policy sets, policies could also be grouped together in a conjunction. The policy also includes a unique identifier. The policy identifier is added to help the translation (from agreements to formulas) but is optional in ODRL proper.

% policy

\lstset{mathescape, language=AST, escapechar=\&}  
\begin{lstlisting}[frame=single, caption={policy},label={lst:policyast}]
<policy> ::=  
	<preRequisite> $\Rightarrow_{<policyId>}$ <act> &\Comment{; primitive policy}&
	'and'[ <policy$_{1}$>, ..., <policy$_{m}$> ] &\Comment{; conjunction }&
\end{lstlisting}

An \emph{Action} (\emph{act}) is represented as a natural number. Similar to assets and subjects, actions are application specific. Some example actions taken from \cite{pucella2006} are \emph{Display} and \emph{Print}.

% act
\lstset{mathescape, language=AST}  
\begin{lstlisting}[frame=single, caption={act},label={lst:actast}]
<act> ::= N
\end{lstlisting}

A \emph{Policy Id} (\emph{policyId}) is a unique identifier specified as (increasing) positive integers. 

% id
\lstset{mathescape, language=AST}  
\begin{lstlisting}[frame=single, caption={policyId},label={lst:policyIdast}]
<policyId> ::= N
\end{lstlisting}

In ODRL0 a \emph{prerequisite} is either true or it is a \emph{constraint}. The \emph{true} prerequisite always holds. A constraint is an intrinsic part of a policy and cannot be influenced by agreement's user. Minimum height requirements for popular attractions and rides are examples of we would consider a constraint. The constraint \emph{ForEachMember} is interesting in its expressive power but has complicated semantics as we shall see in the~\ref{chap:semantics} chapter. Roughly speaking, ForEachMember takes a prin (a list of subjects) and a list $L$ of constraints. The ForEachConstraint holds if each subject in prin satisfies each constraint in $L$.\emph{NotCons} is a negation of a constraint. The set of prerequisites are closed under conjunction (\emph{AndPrqs}), disjunction (\emph{OrPrqs}) and exclusive disjunction (\emph{XorPrqs}).

% prq

\lstset{mathescape, language=AST, escapechar=\&}  
\begin{lstlisting}[frame=single, caption={preRequisite},label={lst:preRequisiteast}]
<preRequisite> ::=  
	'True' &\Comment{; always true}&
	<constraint>	 &\Comment{; constraint}&
	'ForEachMember' [<prin> ; <constraint$_{1}$>, ..., <constraint$_{m}$> ]	&\Comment{; constraint distribution}&
	'not' [ <constraint> ] &\Comment{; suspending constraint}&
	'and'[ <preRequisite$_{1}$>, ..., <preRequisite$_{m}$> ] &\Comment{; conjunction }&
	'or'[ <preRequisite$_{1}$>, ..., <preRequisite$_{m}$> ] &\Comment{; disjunction}&
	'xor'[ <preRequisite$_{1}$>, ..., <preRequisite$_{m}$> ] &\Comment{; exclusive disjunction}&
\end{lstlisting}

Constraints are either \emph{Principal}, \emph{Count} or \emph{CountByPrin}. Principal constraints basically require matching to specified prins. For example, the user being Alice is a Principal constraint. A count constraint refers to a set of policies \emph{P} and specifies the number of times the user of an agreement has invoked the policies in P to justify her actions. If the count constraint is part of a policy then the set P is composed of the single policy. In the case that the count constraint is part of a policy set, the set P is the set of policies specified in the policy set.

% constraint
\lstset{mathescape, language=AST, escapechar=\&}  
\begin{lstlisting}[frame=single, caption={constraint},label={lst:constraintast}]
<constraint> ::=  
	<prin> &\Comment{; principal}&
	'Count' [N] &\Comment{; number of executions}&
	<prin> ('Count' [N]) &\Comment{; number of executions by prin}&
\end{lstlisting}
 
%======================================================================
\chapter{ODRL0 Syntax In Coq}
%======================================================================

% ---------------------------------- COQ -----------------------

ODRL0 productions were presented as high level abstract syntax in ~\ref{sec:productionast}. Below we present the corresponding encodings in Coq. 

An agreement is a new inductive type in Coq by the same name. The constructor $Agreement$ takes a $prin$, an $asset$ and a $policySet$. $prin$ is defined to be a non empty list of $subject$s. 

Types $asset$, $subject$, $act$ and $policyId$ are simply defined as $nat$ which is the datatype of natural numbers defined in coq's library module $Coq.Init.Datatypes$ ($nat$ is itself an inductive datatype). We use Coq constants to refer to specific objects of each type. For example, the subject 'Alice' is defined as $Definition Alice:subject := 101.$ and the act 'Play' as $Definition Play : act := 301.$. For each ``nat'' type in ODRL0 we have also used constants that play the role of ``Null'' objects (see ``Null Object Pattern'' ~\cite{martin1998pattern}), for example $NullSubject$. This is needed partly because of the way ODRL0 language elements are defined which corresponds to the need to use $nonemptylist$ exclusively even though at intermediate stages during the various algorithms Coq's $list$ is a better fit because it allows empty lists.

Next we define the $policySet$ datatype. Note the close/one-to-one mapping to its counterpart in ~\ref{lst:policySetast}. There are three ways a $policySet$ can be constructed (see ~\ref{lst:policySetast}) corresponding to three constructors: $PrimitivePolicySet$, $PrimitiveExclusivePolicySet$ and $AndPolicySet$. Both $PrimitivePolicySet$ and $PrimitiveExclusivePolicySet$ take a $preRequisite$ and a $policy$ as parameters. Finally $AndPolicySet$ takes a non empty list of $policySet$s.

A $policy$ is defined as a datatype with constructors $PrimitivePolicy$ and $AndPolicy$ (see ~\ref{lst:policyast}). $PrimitivePolicy$ takes a $preRequisite$, a $policyId$ and a action $act$. Ignoring the $policyId$ for a moment (it is only added to help the translation otherwise $policyId$s don't exist in ODRL proper), a primitive policy consists of a prerequisite and an action. If the prerequisite holds the action is allowed to be performed on the asset. The $AndPolicy$ constructor is simply a non empty list of $policy$s.


\lstset{language=Coq}
\begin{minipage}[c]{0.95\textwidth}
\begin{lstlisting}[frame=single, caption={Coq version of agreement},label={lst:agreementcoq}]
Inductive agreement : Set :=
  | Agreement : prin -> asset -> policySet -> agreement.

Definition prin := nonemptylist subject.

Definition asset := nat.

Definition subject := nat.

Definition act := nat.

Definition policyId := nat.

Inductive policySet : Set :=
  | PrimitivePolicySet : preRequisite -> policy -> policySet 
  | PrimitiveExclusivePolicySet : preRequisite -> policy  -> policySet 
  | AndPolicySet : nonemptylist policySet -> policySet.

Inductive policy : Set :=
  | PrimitivePolicy : preRequisite -> policyId -> act -> policy 
  | AndPolicy : nonemptylist policy -> policy.

\end{lstlisting}
\end{minipage}

A $preRequisite$ (see ~\ref{lst:preRequisiteast} for the abstract syntax equivalent) is defined as a new datatype with constructors $TruePrq$, $Constraint$, $ForEachMember$, $NotCons$, $AndPrqs$, $OrPrqs$ and $XorPrqs$. 

$TruePrq$ represents the always true prerequisite. The $Constraint$ prerequisite is defined as the type $constraint$ so its description is deferred here. Intuitively a constraint is a prerequisite to be satisfied that is outside the control of the user(s). For example, the constraint of being 'Alice' if you are 'Bob' (or 'Alice' for that matter). The constructor $ForEachMember$ is defined to be a $prin$ and a non empty list of $constraint$s. Intuitively a $ForEachMember$ prerequisite holds if each subject in $prin$ satisfies each constraint in the list of $constraint$s. The constructor $NotCons$ is defined the same way the $Constraint" constructor$ is. This constructor is defined as the type $constraint$ and it is meant to represent the negation of a $constraint$ as we shall see in the translation (see ~\ref{lst:transnotConsCoq}). The remaining constructors $AndPrqs$, $OrPrqs$ and $XorPrqs$ take as parameters non empty lists of prerequisites. They represent conjunction, inclusive disjunction and exclusive disjunction of prerequisites respectively. 

Finally a $constraint$ (see  ~\ref{lst:constraintast} for the abstract syntax equivalent) is defined as a new datatype with constructors $Principal$, $Count$ and $CountByPrin$. 

Principal constraint takes a $prin$ to match. For example, the constraint of the user being Bob would be represented as ``Principal constraint''. The $Count$ constructor takes a $nat$ which represents the number of times the user of an agreement has invoked the corresponding policies to justify her actions. If the count constraint is part of a policy then the corresponding policies is basically the single policy, whereas in the case that the count constraint is part of a policy set, the corresponding policies would be the set of those policies specified in the policy set. The $CountByPrin$ is similar to $Count$ but it takes an additional $prin$ parameter. In this case the subjects specified in the $prin$ parameter override the agreements' user(s).   

% prq
\lstset{language=Coq}
\begin{minipage}[c]{0.95\textwidth}
\begin{lstlisting}[frame=single, caption={preRequisite},label={lst:preRequisitecoq}]
Inductive preRequisite : Set :=
  | TruePrq : preRequisite
  | Constraint : constraint -> preRequisite 
  | ForEachMember : prin  -> nonemptylist constraint -> preRequisite
  | NotCons : constraint -> preRequisite 
  | AndPrqs : nonemptylist preRequisite -> preRequisite
  | OrPrqs : nonemptylist preRequisite -> preRequisite
  | XorPrqs : nonemptylist preRequisite -> preRequisite.
  
  
Inductive constraint : Set :=
  | Principal : prin  -> constraint 
  | Count : nat -> constraint 
  | CountByPrin : prin -> nat -> constraint.

\end{lstlisting}
\end{minipage}

 
%======================================================================
\chapter{ODRL0 Semantics}\label{chap:semantics}
%======================================================================

                  
%----------------------------------------------------------------------
\section{Introduction}\label{sec:introsemantics}


In this section, we describe the semantics of ODRL0 language by a translation from agreements to a subset of many-sorted first-order logic formulas with equality. Note that in the listings in this chapter we use $[\![]\!]$ (double square brackets) notation as a mapping of ODRL0 syntactic elements to their translations as many-sorted first-order logic formulas. We also use $\triangleq$ between a translation and its corresponding formula to mean the translation is ``mapped to'' the formula.

The semantics will help answer queries of the form ``may subject \emph{s} perform action \emph{act} to asset \emph{a}?''. If the answer is yes, we say permission is granted. Otherwise permission is denied. 


At a high-level, an agreement is translated into a conjunction of formulas of the form $\forall x ( prerequisites(x) \rightarrow P(x))$ where $P(x)$ itself is a conjunction of formulas of the form $ prerequisites(x) \rightarrow (\lnot) Permitted (x, act, a)$, where $Permitted (x, act, a)$ means the subject $x$ is permitted to perform action $act$ on asset $a$.

\section{Agreement Translation}
The translation of an $agreement$ returns the translation for a $policySet$ with arguments $prin_{u}$, the agreement's user and $a$, the asset.


\lstset{mathescape, language=AST}  
\begin{lstlisting}[frame=single, caption={Agreement Translation},label={lst:transAgreementast}]
$[\![ agreement$ $for$ $prin_{u}$ $about$ $a$ $with$ $policySet]\!]$ $\triangleq$ $[\![policySet]\!]^{prin_{u}, a}$
\end{lstlisting}

\section{Policy Set Translation}
The translation for a $policySet$ ($[\![policySet]\!]^{prin_{u}, a}$) is defined by cases, one for each clause of the grammar in listing ~\ref{lst:policySetast}. Recall that a $policySet$ is either a $PrimitivePolicySet$, a $\linebreak PrimitiveExclusivePolicySet$ or a $AndPolicySet$. Each of these has its own translation function, which will be defined in the next 3 subsections.

\subsection{PrimitivePolicySet Translation}
Translation of a $PrimitivePolicySet$ ($preRequisite \rightarrow policy$) yields a formula that includes a test on whether the subject is in the set of agreements' users, the translation of the policy and the translation of the $prerequisite$. Basically if the subject in question is a user of the agreement and the policySet prerequisites hold, then the policy holds. Translation of the policy for a $PrimitivePolicySet$ is called a \emph{positive translation}. A positive translation is one where the actions described by the policies are permitted.   

\lstset{mathescape, language=AST}  
\begin{lstlisting}[frame=single, caption={Policy Set Translation {$\colon$} PrimitivePolicySet},label={lst:transpolicyformulaPrimitivePolicySet}]
$[\![ preRequisite \rightarrow policy]\!]^{e, prin_{u}, a}$ $\triangleq$ $\forall x$ $(\!( [\![prin_{u}]\!]_{x}$ $\land$ $[\![preRequisite]\!]^{e, getId (p), prin_{u}, a}_{x}) \rightarrow [\![policy]\!]^{positive, e, prin_{u}, a}_{x}\!)$
\end{lstlisting}



\lstset{mathescape, language=AST} 
\begin{lstlisting}[frame=single, caption={Positive Policy Translation {$\colon$} Single policy},label={lst:transpolicypositiveSingle}]

$[\![ preRequisite \Rightarrow_{policyId} act ]\!]^{positive, e, prin_{u}, a}_{x}$ $\triangleq$ $([\![ preRequisite ]\!]^{e, policyId, prin_{u}}_{x}) \Rightarrow Permitted(x, [\![act]\!], a)$

\end{lstlisting}


If the policy is a $AndPolicy$, the translation yields a conjunction of positive translations of each policy in turn.

\lstset{mathescape, language=AST}  
\begin{lstlisting}[frame=single, caption={Positive Policy Translation {$\colon$} List of policies},label={lst:transpolicypositiveListOfPolicies}]

$[\![ and [policy_{1}, ..., policy_{m}]]\!]^{positive, e, prin_{u}, a}$ $\triangleq$ $[\![policy_{1}]\!]^{positive, e, prin_{u}, a}$ $\land$ $...$ $\land$ $[\![policy_{m}]\!]^{positive, e, prin_{u}, a}$

\end{lstlisting}


\subsection{PrimitiveExclusivePolicySet Translation}
$PrimitiveExclusivePolicySet$ ($preRequisite \mapsto policy$) yields the conjunction of two implications. The first implication, is the same as one found in the translation of $PrimitivePolicySet$. The second implication however restricts access (to make the policy set exclusive) to only those subjects that are in the agreement's user. Translation of the policy in the second implication is called a \emph{negative translation}. A negative translation is one where the actions described by the policies are not permitted. 


\lstset{mathescape, language=AST}  
\begin{lstlisting}[frame=single, caption={Policy Set Translation {$\colon$} PrimitiveExclusivePolicySet},label={lst:transpolicyformulaPrimitiveExclusivePolicySet}]
$[\![ preRequisite \mapsto policy]\!]^{e, prin_{u}, a}$ $\triangleq$ $\forall x$ $(\!( [\![prin_{u}]\!]_{x}$ $\land$ $[\![preRequisite]\!]^{e, getId (p), prin_{u}, a}_{x}) \rightarrow [\![policy]\!]^{positive, e, prin_{u}, a}_{x}\!)$ $\land$ $\forall x$ $(\neg[\![prin_{u}]\!]_{x} \rightarrow [\![policy]\!]^{negative, e, a}_{x})$
\end{lstlisting}


\lstset{mathescape, language=AST}  
\begin{lstlisting}[frame=single, caption={Negative Policy Translation {$\colon$} Single policy},label={lst:transpolicynegativeSingle}]

$[\![ preRequisite \Rightarrow_{policyId} act ]\!]^{negative, e, prin_{u}, a}_{x}$ $\triangleq$ $([\![ preRequisite ]\!]^{e, policyId, prin_{u}}_{x}) \Rightarrow \lnot (Permitted(x, [\![act]\!], a))$
\end{lstlisting}

If the policy is a $AndPolicy$, the translation yields a conjunction of negative translations of each policy in turn.

\lstset{mathescape, language=AST}  
\begin{lstlisting}[frame=single, caption={Negative Policy Translation {$\colon$} List of policies},label={lst:transpolicynegativeListOfPolicies}]

$[\![ and [policy_{1}, ..., policy_{m}]]\!]^{negative, e, a}$ $\triangleq$ $[\![policy_{1}]\!]^{negative, e, a}$ $\land$ $...$ $\land$ $[\![policy_{m}]\!]^{negative, e, a}$

\end{lstlisting}

\subsection{AndPolicySet Translation}
$AndPolicySet$ translates to conjunctions of the corresponding policy set translations. 

\lstset{mathescape, language=AST}  
\begin{lstlisting}[frame=single, caption={Policy Set Translation {$\colon$} AndPolicySet},label={lst:transpolicyformulaAndPolicySet}]
$[\![ and [policySet_{1}, ..., policySet_{m}]]\!]^{e, prin_{u}, a}$ $\triangleq$ $[\![policySet_{1}]\!]^{e, prin_{u}, a}$ $\land$ $...$ $\land$ $[\![policySet_{m}]\!]^{e, prin_{u}, a}$

\end{lstlisting}

\section{Principal Translation}
Translation for a \emph{prin} ($[\![ prin ]\!]_{x}$) is a formula that is true if and only if the subject $x$ is in the prin set. A $prin$ is either a single subject or a list of subjects ($\{ subject_{1}, ..., subject_{m} \}$ so the translation covers both cases. Each of these has its own translation function, which will be defined in the following 2 subsections.


If the $prin$ is a single subject, the translation is a formula that is true if and only if the subject $x$ is the same as the single subject $subject$.

\subsection{Single Subject Translation}
\lstset{mathescape, language=AST}  
\begin{lstlisting}[frame=single, caption={Prin Translation {$\colon$} Single subject},label={lst:transprinSingle}]
$[\![ subject ]\!]_{x}$ $\triangleq$ $x=subject$
\end{lstlisting}

\subsection{List of Subjects Translation}
Translation of a list of subjects is the disjunction of the translations for each subject.

\lstset{mathescape, language=AST}  
\begin{lstlisting}[frame=single, caption={Prin Translation {$\colon$} List of subjects},label={lst:transprinListOfSubjects}]

$[\![ \{ subject_{1}, ..., subject_{m} \} ]\!]_{x}$ $\triangleq$ $[\![subject_{1}]\!]_{x}$ $\lor$ $...$ $\lor$ $[\![subject_{m}]\!]_{x}$

\end{lstlisting}




\section{Prerequisite Translation}

Translation for a prerequisite is a formula $[\![prerequisite]\!]^{[id_{1}, ..., id_{m}], prin, a}_{x}$, where the set of $id$s refer to identifiers for policies that are implied by the prerequisites, $prin$ is the agreement's user(s) (and to which the prerequisites apply), $a$ is the asset and $x$ is a variable of type $subject$. The translation for a $prerequisite$ is described by translation formulas for each type of $prerequisite$. A $prerequisite$ is either always $true$, a $Constraint$, a $ForEachMember$, a $NotCons$, a $AndPrqs$, a $OrPrqs$ or a $XorPrqs$. Each of these has its own translation function, which will be defined in the following subsections.

\subsection{True Prerequisite Translation}
The translation for a $TruePrq$ yields a formula that is always \emph{true}.

\lstset{mathescape, language=AST}  
\begin{lstlisting}[frame=single, caption={Prerequisite Translation {$\colon$} Always True Prerequisite},label={lst:transpreRequisiteTruePrq}]
	$[\![ prerequisite::true ]\!]$ $\triangleq$ True
\end{lstlisting}

\subsection{Constraint Prerequisite Translation}
The translation for a $Constraint$ is handled by a specialized constraint translation function (coverage of which starts at ~\ref{lst:transconstraintPrin}.

\lstset{mathescape, language=AST}  
\begin{lstlisting}[frame=single, caption={Prerequisite Translation {$\colon$} Constraint},label={lst:transpreRequisiteConstraint}]

$[\![ prerequisite::constraint ]\!]^{[id_{1}, ..., id_{m}], prin_{u}}_{x}$ $\triangleq$ $[\![ constraint ]\!]^{[id_{1}, ..., id_{m}], prin_{u}}_{x}$ 
\end{lstlisting}

\subsection{ForEachMember Prerequisite Translation}
The translation for a $ForEachMember$ is also is handled by a specialized translation function (covered at ~\ref{lst:transforEachMember}.

\lstset{mathescape, language=AST}  
\begin{lstlisting}[frame=single, caption={Prerequisite Translation {$\colon$} ForEachMember},label={lst:transpreRequisiteForEachMember}]

$[\![ prerequisite::forEachMember ]\!]^{[subject_{1}, ..., subject_{k}], [constraint_{1}, ..., constraint_{m}], [id_{1}, ..., id_{n}]}_{x}$ $\triangleq$ $[\![ forEachMember ]\!]^{[subject_{1}, ..., subject_{k}], [constraint_{1}, ..., constraint_{m}], [id_{1}, ..., id_{n}]}_{x}$ 	
\end{lstlisting}

\subsection{NotCons Prerequisite Translation}
The translation for a $NotCons$ yields a formula that is simply the negation of the translation for a constraint.

\lstset{mathescape, language=AST}  
\begin{lstlisting}[frame=single, caption={Prerequisite Translation {$\colon$} Not Constraint},label={lst:transpreRequisiteNotConstraint}]

$[\![ not$ $prerequisite::constraint ]\!]^{[id_{1}, ..., id_{m}], prin_{u}}_{x}$ $\triangleq$ $\lnot[\![ constraint ]\!]^{[id_{1}, ..., id_{m}], prin_{u}}_{x}$ 
\end{lstlisting}

\subsection{AndPrqs Prerequisite Translation}
The translation for a $AndPrqs$ yields a formula that is the conjunction of the translation for each $preRequisite$.

\lstset{mathescape, language=AST}  
\begin{lstlisting}[frame=single, caption={Prerequisite Translation {$\colon$} Conjunction },label={lst:transpreRequisiteAndPrqs}]

$[\![and$ $[preRequisite_{1}, ..., preRequisite_{k}]]\!]^{[id_{1}, ..., id_{m}], prin_{u}}$ $\triangleq$ $[\![preRequisite_{1}]\!]^{[id_{1}, ..., id_{m}], prin_{u}}$ $\land$ $...$ $\land$ $[\![preRequisite_{k}]\!]^{[id_{1}, ..., id_{m}], prin_{u}}$

\end{lstlisting}

\subsection{OrPrqs Prerequisite Translation}
The translation for a $OrPrqs$ yields a formula that is the inclusive disjunction of the translation for each $preRequisite$.

\lstset{mathescape, language=AST}  
\begin{lstlisting}[frame=single, caption={Prerequisite Translation {$\colon$} Inclusive Disjunction},label={lst:transpreRequisiteOrPrqs}]

$[\![or$ $[preRequisite_{1}, ..., preRequisite_{k}]]\!]^{[id_{1}, ..., id_{m}], prin_{u}}$ $\triangleq$ $[\![preRequisite_{1}]\!]^{[id_{1}, ..., id_{m}], prin_{u}}$ $\lor$ $...$ $\lor$ $[\![preRequisite_{k}]\!]^{[id_{1}, ..., id_{m}], prin_{u}}$

\end{lstlisting}

\subsection{XorPrqs Prerequisite Translation}
The translation for a $XorPrqs$ yields a formula that is the exclusive disjunction of the translation for each $preRequisite$.


\lstset{mathescape, language=AST}  
\begin{lstlisting}[frame=single, caption={Prerequisite Translation {$\colon$} Exclusive Disjunction},label={lst:transpreRequisiteXorPrqs}]

$[\![Xor$ $[preRequisite_{1}, ..., preRequisite_{k}]]\!]^{[id_{1}, ..., id_{m}], prin_{u}}$ $\triangleq$ $[\![preRequisite_{1}]\!]^{[id_{1}, ..., id_{m}], prin_{u}}$ $\oplus $ $...$ $\oplus$ $[\![preRequisite_{k}]\!]^{[id_{1}, ..., id_{m}], prin_{u}}$

\end{lstlisting}


\section{Constraint Translation}

Translation for a constraint is a formula $[\![constraint]\!]^{[id_{1}, ..., id_{m}], prin_{u}, a}_{x}$, where the set of $id$s refer to identifiers for policies that are implied by the constraint, $prin_{u}$ is the agreement's user(s) (and to which the constraint applies), $a$ is the asset and $x$ is a variable of type $subject$. The translation for a $constraint$ is described by translation formulas for each type of $constraint$. A $constraint$ is either a $Principal$, a $Count$, or a $CountByPrin$. Each of these has its own translation function, which will be defined in the following subsections.

\subsection{Principal Constraint Translation}
The translation for a $Principal$ is handled by a specialized translation function (covered at ~\ref{lst:transprin}.   

\lstset{mathescape, language=AST}  
\begin{lstlisting}[frame=single, caption={Constraint Translation {$\colon$} Principal},label={lst:transconstraintPrin}]

$[\![ constraint::prin ]\!]^{[subject_{1}, ..., subject_{m}]}_{x}$ $\triangleq$ $[\![ prin ]\!]^{[subject_{1}, ..., subject_{m}]}_{x}$ 
\end{lstlisting}

\subsection{Count Constraint Translation}
The translation for a $Count$ is handled by a specialized translation function (covered at ~\ref{lst:transcount}.

\lstset{mathescape, language=AST}  
\begin{lstlisting}[frame=single, caption={Constraint Translation {$\colon$} Count},label={lst:transconstraintCount}]

$[\![ constraint::count [N] ]\!]^{[id_{1}, ..., id_{m}], prin_{u}}_{x}$ $\triangleq$ $[\![ count [N] ]\!]^{[id_{1}, ..., id_{m}], prin_{u}}_{x}$ 
\end{lstlisting}

\subsection{CountByPrin Constraint Translation}
The translation for a $CountByPrin$ is handled by the same specialized translation function as that for $Count$. The difference is that $CountByPrin$ overrides the subjects in $prin_{u}$ by a different set of subjects (covered at ~\ref{lst:transcount}.

\lstset{mathescape, language=AST}  
\begin{lstlisting}[frame=single, caption={Constraint Translation {$\colon$} Count by Principal},label={lst:transconstraintCountbyPrin}]

$[\![ constraint::prin(count [N]) ]\!]^{[subject_{1}, ..., subject_{m}], [id_{1}, ..., id_{n}]}_{x}$ $\triangleq$ $[\![ prin(count [N]) ]\!]^{[subject_{1}, ..., subject_{m}], [id_{1}, ..., id_{n}]}_{x}$ 
\end{lstlisting}


\subsection{forEachMember Translation}

\lstset{mathescape, language=AST}  
\begin{lstlisting}[frame=single, caption={ForEachMember Translation {$\colon$} Count by Principal},label={lst:transforEachMember}]

$[\![ forEachMember ]\!]^{[subject_{1}, ..., subject_{k}], [constraint_{1}, ..., constraint_{m}], [id_{1}, ..., id_{n}]}_{x}$ $\triangleq$ $[\![constraint]\!]^{(subject_{1}, constraint_{1}), [id_{1}, ..., id_{n}]}_{x}$ $\land$ $...$ $\land$ $[\![constraint]\!]^{(subject_{1}, constraint_{m}), [id_{1}, ..., id_{n}]}_{x}$ $\land$ $...$ $\land$ $[\![constraint]\!]^{(subject_{2}, constraint_{1}), [id_{1}, ..., id_{n}]}_{x}$ $\land$ $...$ $\land$ $[\![constraint]\!]^{(subject_{2}, constraint_{m}), [id_{1}, ..., id_{n}]}_{x}$ $\land$ $...$ $\land$ $[\![constraint]\!]^{(subject_{k}, constraint_{1}), [id_{1}, ..., id_{n}]}_{x}$ $\land$ $...$ $\land$ $[\![constraint]\!]^{(subject_{k}, constraint_{m}), [id_{1}, ..., id_{n}]}_{x}$ 

\end{lstlisting}

\subsection{''Not Constraint" Translation}

The translation for ``Not Constraint'' was listed in listing ~\ref{lst:transpreRequisiteNotConstraint} earlier but we repeat it here to go along the Coq version. 

\lstset{mathescape, language=AST}  
\begin{lstlisting}[frame=single, caption={Not Constraint Translation},label={lst:transnotCons}]

$[\![ not$ $constraint ]\!]^{[id_{1}, ..., id_{m}], prin_{u}}_{x}$ $\triangleq$ $\lnot[\![ constraint ]\!]^{[id_{1}, ..., id_{m}], prin_{u}}_{x}$ 

\end{lstlisting}

\section{Count Translation}
Translation for $Count$ or $CountByPrin$ is based on whether the translation is on a single pair and multiple pairs of subject/policyIds. Each of these two cases will be described in the following subsections.

\subsection{Count Translation For Subject/ID Pair}
The translation for $Count$ or $CountByPrin$ for a pair of subject and policy identifier is a formula that is true if the number of times the $subject_{1}$ has invoked a policy with policy identifier $id_{1}$ is smaller than $N$.

\lstset{mathescape, language=AST}  
\begin{lstlisting}[frame=single, caption={Count Translation {$\colon$} subject and policyId pair},label={lst:transcountSinglePair}]

$[\![ count [N] ]\!]^{subject_{1}, id_{1}}_{x}$ $\triangleq$ $getCount(subject_{1}, id_{1}) < N$
\end{lstlisting}

\subsection{Count Translation For Subject/ID Pairs}
The translation for $Count$ or $CountByPrin$ for subject and policy identifier pairs is a formula that is true if the total number of times that a subject has invoked a policy with policy identifier $id_{i}$ is smaller than $N$.

\lstset{mathescape, language=AST}  
\begin{lstlisting}[frame=single, caption={Count Translation {$\colon$} subject and policyId pairs},label={lst:transcountPairs}]

$[\![ count [N] ]\!]^{[id_{1}, ..., id_{m}], prn}_{x}$ $\triangleq$ 

$(getCount(getSubject(prn)_{1}, id_{1})$ $+ ... +$ $getCount(getSubject(prn)_{1}, id_{m})$ $+ ... +$ $getCount(getSubject(prn)_{k}, id_{1})$ $+ ... +$ $getCount(getSubject(prn)_{k}, id_{m}))$ < $N$


\end{lstlisting}



 
%======================================================================
\chapter{ODRL0 Semantics In Coq}
%======================================================================

% ---------------------------------- COQ -----------------------

\section{Introduction}


% COQ
The translation functions plus the auxiliary types and infrastructure which implement the semantics have been encoded in Coq. Translation functions build Coq terms of type $Prop$. Well-formed propositions (or $Prop$s) are assertions one can express about values such as mathematical objects or even programs (e.g. 3 < 8) in Coq. 

Whether a permission is granted or denied depends on the agreements in question but also on the facts recorded in the environment. For ODRL0 those facts revolve around the number of times a policy has been used to justify an action (see section~\ref{sec:odrl0} for more details on odrl0). We encode this information in an \emph{environment} which is a conjunction of equalities of the form $count(s, policyId) = n$. 

% COQ
The Coq version of the count equality is a new inductive type called \emph{count\_equality}. An environment is defined to be a non-empty list of $count\_equality$ objects (see listing~\ref{lst:environmentcoq}). Function $make\_count\_equality$ in listing~\ref{lst:environmentcoq} is simply a convenience function that builds $count\_equality$s. For an example of how environments are created see listing~\ref{lst:environmentusagecoq}.

\lstset{language=Coq, frame=single, caption={Environments and Counts},label={lst:environmentcoq}}
%\begin{minipage}[c]{0.95\textwidth}
\begin{lstlisting}
Inductive count_equality : Set := 
   | CountEquality : subject -> policyId -> nat -> count_equality.

Definition make_count_equality
  (s:subject)(id:policyId)(n:nat): count_equality :=
  CountEquality s id n.
  
Inductive environment : Set := 
  | SingleEnv : count_equality -> environment
  | ConsEnv :  count_equality ->  environment -> environment.

\end{lstlisting}
%\end{minipage}

\lstset{language=Coq, frame=single, caption={Defining Environments},label={lst:environmentusagecoq}}
\begin{minipage}[c]{0.95\textwidth}
\begin{lstlisting}

Definition e1 : environment := 
  (SingleEnv (make_count_equality Alice id1 8)).

\end{lstlisting}
\end{minipage}
  
  
We also define a \emph{getCount} function (see listing~\ref{lst:getCountcoq}) that given a pair consisting of a subject and policy id, looks for a corresponding count in the environment.
$getCount$ assumes the given environment is consistent, so it returns the first matched $count$ it sees for a $(subject, id)$ pair. If a $count$ for a $(subject, id)$ pair is not found it returns 0.

\lstset{language=Coq, frame=single, caption={getCount Function},label={lst:getCountcoq}}
\begin{minipage}[c]{0.95\textwidth}
\begin{lstlisting}
Fixpoint getCount 
  (e:environment)(s:subject)(id: policyId): nat :=
  match e with
  | SingleEnv f  => 
      match f with 
	  | CountEquality s1 id1 n1 => 
          if (beq_nat s s1) 
          then if (beq_nat id id1) then n1 else 0 
          else 0  
      end			
  | ConsEnv f rest =>
      match f with 
	  | CountEquality s1 id1 n1 => 
          if (beq_nat s s1)
          then if (beq_nat id id1) then n1 else (getCount rest s id)  
          else (getCount rest s id)
      end
  end.
\end{lstlisting}
\end{minipage}

\section{Translations}

Translation of the top level $agreement$ element proceeds by case analysis on the structure of the agreement. However an agreement can only be built one way; by calling the constructor $Agreement$. The translation proceeds by calling the translation function for the corresponding $policySet$ namely the parameter to $Agreement$ called $ps$.


\lstset{language=Coq, frame=single, caption={Translation of Agreement},label={lst:transagreement}}
\begin{lstlisting}

Definition trans_agreement (e:environment)(ag:agreement) : Prop :=
  match ag with 
    | Agreement prin_u a ps => trans_ps e ps prin_u a
  end.

\end{lstlisting}

Translation of a $policySet$ (called $trans\_ps$ in listing~\ref{lst:transpsCoq}), takes as input $e$, the environment, $ps$, the policy set, $prin_{u}$, the agreement's user, and $a$, the asset, and proceeds by case analysis of different policySet constructors and recursing into translation functions for the composing elements. A policySet is either a $PrimitivePolicySet$, $PrimitiveExclusivePolicySet$ or a $AndPolicySet$. 

Note that to implement the translation for an $AndPolicySet$ a local function $trans_ps_list$ has been defined where for a single $policySet$, $trans_ps$ is called, and for a list of $policySet$s, the conjunction of $trans_ps$s are returned.

\lstset{language=Coq, frame=single, caption={Translation of Policy Set},label={lst:transpsCoq}}
\begin{minipage}{\linewidth}
\begin{lstlisting}

Fixpoint trans_ps
  (e:environment)(ps:policySet)(prin_u:prin)(a:asset){struct ps} : Prop :=

let trans_ps_list := (fix trans_ps_list (ps_list:nonemptylist policySet)(prin_u:prin)(a:asset){struct ps_list}:=
  match ps_list with
    | Single ps1 => trans_ps e ps1 prin_u a
    | NewList ps ps_list' => ((trans_ps e ps prin_u a) /\ (trans_ps_list ps_list' prin_u a))
  end) in
    match ps with
    | PrimitivePolicySet prq p => forall x, (((trans_prin x prin_u) /\ 
                                   (trans_preRequisite e x prq (getId p) prin_u)) -> 
                                   (trans_policy_positive e x p prin_u a))  

    | PrimitiveExclusivePolicySet prq p => forall x, ((((trans_prin x prin_u) /\ 
                                              (trans_preRequisite e x prq (getId p) prin_u)) -> 
                                             (trans_policy_positive e x p prin_u a)) /\
                                            ((not (trans_prin x prin_u)) -> (trans_policy_negative e x p a)))
                   
    | AndPolicySet ps_list => trans_ps_list ps_list prin_u a
    end.
\end{lstlisting}
\end{minipage}



% COQ
Translation of a \emph{prin} (called $trans\_prin$ in listing~\ref{lst:transprin}) takes as input $x$, the $subject$ in question, $p$, the principal or the $prin$,  and proceeds based on whether $p$ is a single subject or a list of subjects. If $p$ is a single subject, $s$, the $Prop$ $x=s$ is returned. Otherwise the disjunction of the translation of the first subject in $p$ ($s$) and the $rest$ of the subjects is returned.

\lstset{language=Coq, frame=single, caption={Translation of a Prin},label={lst:transprin}}
\begin{lstlisting}

Fixpoint trans_prin
  (x:subject)(p: prin): Prop :=

  match p with
    | Single s => (x=s)
    | NewList s rest => ((x=s) \/ trans_prin x rest)
  end.
\end{lstlisting}


A positive translation for a policy (called $trans\_policy\_positive$ in listing~\ref{lst:transpolicypositiveCoq}) takes as input $e$, the $environment$, $x$, the $subject$, $p$, the $policy$ to translate, $prin_{u}$, the agreement's user, and $a$, the asset and proceeds based on whether we have a $PrimitivePolicy$ or a $AndPolicy$. If the policy is a $PrimitivePolicy$ an implication is returned which indicates $x$ is \emph{permitted} to do $action$ to $a$, if the $preRequisite$ holds. 

$Permitted$ is a predicate specified as $Parameter Permitted : subject -> act -> asset -> Prop.$ So $Permitted$ predicate takes a $subject$, an $act$ (an action) and an $asset$ and builds a term of type $Prop$. 

Note that to implement the translation for an $AndPolicy$ a local function $trans\_p\_list$ has been defined where for a single $policy$, $trans\_policy\_positive$ is returned, and for a list of $policy$s, the conjunction of $trans\_policy\_positive$s are returned.

\lstset{language=Coq, frame=single, caption={Translation of a Positive Policy},label={lst:transpolicypositiveCoq}}
\begin{lstlisting}

Fixpoint trans_policy_positive
  (e:environment)(x:subject)(p:policy)(prin_u:prin)(a:asset){struct p} : Prop :=

let trans_p_list := (fix trans_p_list (p_list:nonemptylist policy)(prin_u:prin)(a:asset){struct p_list}:=
                  match p_list with
                    | Single p1 => trans_policy_positive e x p1 prin_u a
                    | NewList p p_list' => 
                        ((trans_policy_positive e x p prin_u a) /\ 
                         (trans_p_list p_list' prin_u a))
                  end) in


  match p with
    | PrimitivePolicy prq policyId action => ((trans_preRequisite e x prq (Single policyId) prin_u) ->
                                              (Permitted x action a))
    | AndPolicy p_list => trans_p_list p_list prin_u a
  end.
\end{lstlisting}

A negative translation for a policy (called $trans\_policy\_negative$ in listing~\ref{lst:transpolicynegativeCoq}) takes as input $e$, the $environment$, $x$, the $subject$, $p$, the $policy$ to translate, and $a$ the asset and proceeds based on whether we have a $PrimitivePolicy$ or a $AndPolicy$. If the policy is a $PrimitivePolicy$ an implication is returned which indicates $x$ is forbidden to do $action$ to $a$ regardless of whether $preRequisite$ holds. As is the case for the positive translation, to implement the translation for an $AndPolicy$ a local function $trans\_p\_list$ has been defined where for a single $policy$, $trans\_policy\_negative$ is returned, and for a list of $policy$s, the conjunction of $trans\_policy\_negative$s are returned.


\lstset{language=Coq, frame=single, caption={Translation of a Negative Policy},label={lst:transpolicynegativeCoq}}
\begin{minipage}[c]{0.95\textwidth}
\begin{lstlisting}

Fixpoint trans_policy_negative
  (e:environment)(x:subject)(p:policy)(a:asset){struct p} : Prop :=
let trans_p_list := (fix trans_p_list (p_list:nonemptylist policy)(a:asset){struct p_list}:=
                  match p_list with
                    | Single p1 => trans_policy_negative e x p1 a
                    | NewList p p_list' => ((trans_policy_negative e x p a) /\ 
                                            (trans_p_list p_list' a))
                  end) in


  match p with
    | PrimitivePolicy prq policyId action => not (Permitted x action a)
    | AndPolicy p_list => trans_p_list p_list a
  end.
\end{lstlisting}
\end{minipage}

The translation of a $prerequisite$ (called $trans\_preRequisite$ in listing~\ref{lst:transpreRequisiteCoq}) takes as input $e$, the $environment$, $x$, the $subject$, $prq$, the $preRequisite$ to translate, $IDs$, the set of identifiers (of policies implied by the $prq$), $prin_{u}$, the agreement's user, and proceeds by case analysis on the structure of the $prerequisite$. A $prerequisite$ is either a $TruePrq$, a $Constraint$, a $ForEachMember$, a $NotCons$, a $AndPrqs$, a $OrPrqs$ or a $XorPrqs$. 

In listing~\ref{lst:transpreRequisiteCoq} the translation for $TruePrq$ is the Prop $True$, the translations for $Constraint$, $ForEachMember$ and $NotCons$ simply call respective translation functions for corresponding types $constraint$ and $forEachMember$ (namely $trans\_constraint$, $trans\_forEachMember$ and $trans\_notCons$). Note that the translation for $AndPrqs$, $OrPrqs$ and $XorPrqs$ have not yet been implemented but based on the their many-sorted-logic formulas' specifications (listing~\ref{lst:preRequisiteast}) they will be conjunctions, disjunctions and exclusive disjunctions of translations for each $prerequisite$.

\lstset{language=Coq, frame=single, caption={Translation of a PreRequisite},label={lst:transpreRequisiteCoq}}
\begin{lstlisting}

Definition trans_preRequisite
  (e:environment)(x:subject)(prq:preRequisite)(IDs:nonemptylist policyId)(prin_u:prin) : Prop := 

  match prq with
    | TruePrq => True
    | Constraint const => trans_constraint e x const IDs prin_u 
    | ForEachMember prn const_list => trans_forEachMember e x prn const_list IDs 
    | NotCons const => trans_notCons e x const IDs prin_u 
    | AndPrqs prqs => True 
    | OrPrqs prqs => True 
    | XorPrqs prqs => True 
  end.
\end{lstlisting}

The translation of a $constraint$ (called $trans\_constraint$ in listing~\ref{lst:transconstraintCoq}) takes as input $e$ the $environment$, $x$ the $subject$, $const$, the $constraint$ to translate, $IDs$, the set of identifiers (of policies implied by the parent $preRequisite$) and $prin_{u}$, the agreement's user and proceeds by case analysis on the structure of the $constraint$. A $constraint$ is either a $Principal$, a $Count$ or a $CountByPrin$. The translation for $Principal$ returns the translation function (namely $trans\_prin$) for the $prn$ (the $prin$ that accompanies the $const$ constraint). The translation for $Count$ and $CountByPrin$ return the translation function $trans\_count$. For $Count$ the $prin$ used is the agreement's user, whereas the $prin$ used is the one passed to $CountByPrin$ namely $prn$.


\lstset{language=Coq, frame=single, caption={Translation of a Constraint},label={lst:transconstraintCoq}}
\begin{lstlisting}

Fixpoint trans_constraint 
  (e:environment)(x:subject)(const:constraint)(IDs:nonemptylist policyId)
  (prin_u:prin){struct const} : Prop := 
  match const with
    | Principal prn => trans_prin x prn
  
    | Count n => trans_count e n IDs prin_u 

    | CountByPrin prn n => trans_count e n IDs prn 

  end.
  
\end{lstlisting}

The translation of a $forEachMember$ (called $trans\_forEachMember$ in listing~\ref{lst:transforEachMemberCoq}) takes as input $e$ the $environment$, $x$ the $subject$, $principals$, the set of subjects that override the agreement's user(s), $const\_list$ the set of constraints and $IDs$, the set of identifiers (of policies implied by the parent $preRequisite$).

To implement the translation for a $forEachMember$ we start by calling an auxiliary function $process\_two\_lists$ that effectively returns a new list composed of pairs of members of the first list and the second list (the cross-product of the two input lists). In the case of a $forEachMember$ translation, the call is ``$process\_two\_lists$ $principals$ $const\_list$'' which returns a list of pairs of subject and constraint namely $prins\_and\_constraints$. $prins\_and\_constraints$ is then passed to a locally defined function \emph{ trans_forEachMember_Aux} where for a single pair of subject and constraint $trans\_constraint$ is called and for a list of pairs of subject and constraints, the conjunction of $trans\_constraint$s (for the first pair) and $trans\_forEachMember\_Aux$s (for the rest of the pairs) are returned.



\lstset{language=Coq, frame=single, caption={Translation of forEachMember},label={lst:transforEachMemberCoq}}
\begin{minipage}[c]{0.95\textwidth}
\begin{lstlisting}

Fixpoint trans_forEachMember
         (e:environment)(x:subject)(principals: nonemptylist subject)(const_list:nonemptylist constraint)
         (IDs:nonemptylist policyId){struct const_list} : Prop := 

let trans_forEachMember_Aux   
  := (fix trans_forEachMember_Aux
         (prins_and_constraints : nonemptylist (Twos subject constraint))
         (IDs:nonemptylist policyId){struct prins_and_constraints} : Prop :=

      match prins_and_constraints with
        | Single pair1 => trans_constraint e x (right pair1) IDs (Single (left pair1)) 
        | NewList pair1 rest_pairs =>
            (trans_constraint e x (right pair1) IDs (Single (left pair1))) /\
            (trans_forEachMember_Aux rest_pairs IDs)
      end) in

      let prins_and_constraints := process_two_lists principals const_list in
      trans_forEachMember_Aux prins_and_constraints IDs.

\end{lstlisting}
\end{minipage}

The translation of a $NotCons$ (called $trans\_notCons$ in listing~\ref{lst:transnotConsCoq}) takes as input $e$ the $environment$, $x$ the $subject$, $const$, the $constraint$ to translate, $IDs$, the set of identifiers (of policies implied by the parent $preRequisite$) and $prin_{u}$, the agreement's user and proceeds to return the negation of $trans\_constraint$ (see listing~\ref{lst:transconstraintCoq}).


\lstset{language=Coq, frame=single, caption={Translation of NotCons},label={lst:transnotConsCoq}}
\begin{lstlisting}

Definition trans_notCons
  (e:environment)(x:subject)(const:constraint)(IDs:nonemptylist policyId)(prin_u:prin) : Prop :=
  ~ (trans_constraint e x const IDs prin_u).
\end{lstlisting}




The translation of a $Count$ or a $CountByPrin$ (called $trans\_count$ in listing~\ref{lst:transcountCoq}) takes as input $e$ the $environment$, $n$ the total number of times the subjects mentioned in $prin_{u}$ (last parameter) may invoke the policies identified by $IDs$ (third parameter).

To implement the translation for a $Count$ or a $CountByPrin$ we start by calling an auxiliary function $process\_two\_lists$ that effectively returns a new list composed of pairs of members of the first list and the second list (the cross-product of the two input lists). In the case of $trans\_count$, the call is ``$process\_two\_lists$ $IDs$ $prin_u$'' which returns a list of pairs of $policyId$ and $subject$ namely $ids\_and\_subjects$. $ids\_and\_subjects$ is then passed to a locally defined function \emph{trans_count_aux}.

$trans\_count\_aux$ returns the current count for a single pair of $policyId$ and $subject$ (the call to $getCount$ which looks up the environment $e$ and returns the current count per each $subject$ and $policyId$) and for a list of pairs of $policyId$ and $subject$s, the addition of $get\_count$ (for the first pair) and $trans\_count\_aux$s (for the rest of the pairs) is returned. 

A local variable $running_total$ has the value returned by $trans\_count\_aux$. Finally the proposition $running_total < n$ is returned as the translation for a $Count$ or a $CountByPrin$.

Note that the only difference between translations for a $Count$ and a $CountByPrin$ is the additional $prn$ parameter for $CountByPrin$ which allows for getting counts for subjects not necessarily the same as $prin_{u}$, the agreement's user(s).

\lstset{language=Coq, frame=single, caption={Translation of count},label={lst:transcountCoq}}
\begin{minipage}[c]{0.95\textwidth}
\begin{lstlisting}
Fixpoint trans_count 
  (e:environment)(n:nat)(IDs:nonemptylist policyId)
  (prin_u:prin) : Prop := 

  let trans_count_aux 
    := (fix trans_count_aux
         (ids_and_subjects : nonemptylist (Twos policyId subject)) : nat :=
     match ids_and_subjects with
        | Single pair1 => getCount e (right pair1) (left pair1)
        | NewList pair1 rest_pairs =>
            (getCount e (right pair1)(left pair1)) +
            (trans_count_aux rest_pairs)
      end) in
  
  let ids_and_subjects := process_two_lists IDs prin_u in
  let running_total := trans_count_aux ids_and_subjects in
  running_total < n.
\end{lstlisting}
\end{minipage} 
%======================================================================
\chapter{Queries}\label{chap:queries}
%======================================================================

% ---------------------------------------------------------

\section{Introduction}


We first mentioned queries in chapter~\ref{chap:semantics} on page~\pageref{chap:semantics}. Ultimately policy statements describing an agreement will be used to enforce those agreements. To enforce policy agreements, access queries are asked from the policy engine and access is granted or denied based on the answer.

By defining formal semantics for \ac{odrl} the authors of ~\cite{pucella2006} were able to prove that answering a query on whether access should be granted or not, is decidable and NP-hard for the full \ac{odrl}. 

In this chapter we will review our encoding of queries in Coq and Coq representations of other definitions used in ~\cite{pucella2006} which we will use to prove decidability results of our own. 


\section{Queries}

Queries are tuples of the form $(A, s, action, a, e)$ in ~\cite{pucella2006}. The tuple corresponds to the question of determining whether a set $A$ of agreements imply that a subject $s$ may perform action $action$ on an asset $a$ given the environment $e$. The Coq representation is listed in listing~\ref{lst:querycoq}. We distinguish single agreement queries from multiple agreement queries by defining two separate types: $single\_query$ and $general\_query$.

\lstset{language=Coq, frame=single, caption={Queries},label={lst:querycoq}}
\begin{minipage}[c]{0.95\textwidth}
\begin{lstlisting}

Inductive single_query : Set := 
   | SingletonQuery : agreement -> subject -> act -> asset -> environment -> single_query.
   

Inductive general_query : Set := 
   | GeneralQuery : nonemptylist agreement -> subject -> act -> asset -> environment -> general_query.
\end{lstlisting}
\end{minipage}

\section{Answering Queries}

Answering a query as defined earlier can lead to one of four outcomes: error (~\ref{lst:errordecision}), permitted (~\ref{lst:permitdecision}), denied (~\ref{lst:denydecision}) and ``not applicable'' (~\ref{lst:notapplicabledecision}) defined in ~\cite{Tschantz}.

\lstset{mathescape, language=AST} 
\begin{lstlisting}[frame=single, caption={Answerable Queries: Error},label={lst:errordecision}]

$(\bigwedge [\![ agreement]\!]) \land E \implies Permitted(s, act, a)$ and $(\bigwedge [\![ agreement]\!]) \land E \implies \lnot Permitted(s, act, a)$

\end{lstlisting}

\lstset{mathescape, language=AST} 
\begin{lstlisting}[frame=single, caption={Answerable Queries: Permit},label={lst:permitdecision}]

$(\bigwedge [\![ agreement]\!]) \land E \implies Permitted(s, act, a)$ and $(\bigwedge [\![ agreement]\!]) \land E \notimplies \lnot Permitted(s, act, a)$

\end{lstlisting}

\lstset{mathescape, language=AST} 
\begin{lstlisting}[frame=single, caption={Answerable Queries: Deny},label={lst:denydecision}]

$(\bigwedge [\![ agreement]\!]) \land E \notimplies Permitted(s, act, a)$ and $(\bigwedge [\![ agreement]\!]) \land E \implies \lnot Permitted(s, act, a)$

\end{lstlisting}

\lstset{mathescape, language=AST} 
\begin{lstlisting}[frame=single, caption={Answerable Queries: Not Applicable},label={lst:notapplicabledecision}]

$(\bigwedge [\![ agreement]\!]) \land E \notimplies Permitted(s, act, a)$ and $(\bigwedge [\![ agreement]\!]) \land E \notimplies \lnot Permitted(s, act, a)$

\end{lstlisting}


In (~\cite{pucella2006}) a slightly different formulation is used to denote the same four decision types. ``Query Inconsistent'',``Permission Granted'', ``Permission Denied'' and ``Permission Unregulated''. 

\lstset{mathescape, language=AST} 
\begin{lstlisting}[frame=single, caption={$f^{+}_q$},label={lst:fplusformula}]

$f^{+}_q \triangleq (\bigwedge [\![ agreement]\!]) \implies Permitted(s, act, a)$ 

\end{lstlisting}

\lstset{mathescape, language=AST} 
\begin{lstlisting}[frame=single, caption={$f^{-}_q$},label={lst:fminusformula}]

$f^{-}_q \triangleq (\bigwedge [\![ agreement]\!]) \implies \lnot Permitted(s, act, a)$ 

\end{lstlisting}


\lstset{mathescape, language=AST} 
\begin{lstlisting}[frame=single, caption={Answerable Queries: Query Inconsistent},label={lst:queryinconsistentdecision}]

$f^{+}_q$ and $f^{-}_q$ both hold

\end{lstlisting}

\lstset{mathescape, language=AST} 
\begin{lstlisting}[frame=single, caption={Answerable Queries: Permission Granted},label={lst:permission granteddecision}]

$f^{+}_q$ holds and $f^{-}_q$ does not hold

\end{lstlisting}

\lstset{mathescape, language=AST} 
\begin{lstlisting}[frame=single, caption={Answerable Queries: Permission Denied},label={lst:permission denieddecision}]

$f^{+}_q$ does not hold and $f^{-}_q$ holds

\end{lstlisting}

\lstset{mathescape, language=AST} 
\begin{lstlisting}[frame=single, caption={Answerable Queries: Permission Unregulated},label={lst:permission unregulateddecision}]

$f^{+}_q$ does not hold and $f^{-}_q$ does not hold

\end{lstlisting}

























 
%======================================================================
\chapter{Examples}
%======================================================================

%----------------------------------------------------------------------
\section{Introduction}
%----------------------------------------------------------------------

%%%%%%% Examples 

In this chapter we will take a tour of the syntax and semantics we have so far developed by examining some example agreements. In the following we will start by reviewing some of the examples used in ~\cite{pucella2006}.

Ultimately the goal of specifying all the syntax and semantics is to declare some interesting theorems about policy expressions and proving them in Coq however we will start with some specific propositions/theorems about these examples to get a feel for how proofs are done in Coq.

\section{Agreement 2.1}

Consider example 2.1 (from ~\cite{pucella2006}) where the $policySet$ is a $AndPolicySet$ with $p1$ and $p2$ as the individual $policySet$s. Let $p1$ be defined as $Count[5]$ $\rightarrow$ $print$ and $p2$ as $and[Alice, Count[5]]$ $\rightarrow$ $print$. 

The agreement is that the asset \emph{The Report} may be printed a total of five times by either \emph{Alice} or \emph{Bob}, and twice more by Alice. So if Alice and Bob have used policy $p1$ to justify their printing of the asset $m_{p1}$ and $n_{p1}$ times, respectively, then either may do so again if $m_{p1} + n_{p1} < 5$. If they have used $p2$ to justify their printing of the asset $m_{p2}$ and $n_{p2}$ times, respectively, then only Alice may do so again if $m_{p2} + n_{p2} < 2$. Note that since Bob doesn't meet the prerequisite of being Alice, $n_{p2}$ is effectively $0$, so we have $m_{p2} < 2$ as the condition for Alice being able to print again (Alice does meet the prerequisite of being Alice).

\lstset{language=Pucella2006}
\begin{lstlisting}[frame=single, caption={Agreement 2.1 (as used in ~\cite{pucella2006})},label={lst:example21pucella2006}]
agreement
 for {Alice, Bob} 
 about TheReport 
 with and [p1, p2].
\end{lstlisting}

The Coq version of the agreement 2.1 (listing ~\ref{lst:example21pucella2006}) and its sub-parts is listed below. It is best to start with the agreement itself called $A2.1$ in the listing and compare to the agreement 2.1 listed in ~\ref{lst:example21pucella2006}.

\lstset{language=Coq}
\begin{minipage}[c]{0.95\textwidth}
\begin{lstlisting}[frame=single, caption={Agreement 2.1 in Coq},label={lst:example21}]


Definition p1A1:policySet :=
  PrimitivePolicySet
    TruePrq
    (PrimitivePolicy (Constraint (Count  5)) id1 Print).

Definition p2A1prq1:preRequisite := (Constraint (Principal (Single Alice))).
Definition p2A1prq2:preRequisite := (Constraint (Count 2)).

Definition p2A1:policySet :=
  PrimitivePolicySet
    TruePrq
    (PrimitivePolicy (AndPrqs (NewList p2A1prq1 (Single p2A1prq2))) id2 Print).

Definition A2.1 := Agreement (NewList Alice (Single Bob)) TheReport
                  (AndPolicySet (NewList p1A1 (Single p2A1))).

\end{lstlisting}                  
\end{minipage}

\section{Agreement 2.5}

Consider example 2.5 (from ~\cite{pucella2006}) where the $policySet$ is a $PrimitivePolicySet$ with a $Count$ constraint as prerequisite and a $AndPolicy$ as the policy. The $AndPolicy$ is the conjunction of two $PrimitivePolicy$s. Both policies have prerequisites of type $ForEachMember$ with actions $display$ and $print$ respectively. The $prin$ component for both $ForEachMember$s is ${Alice, Bob}$, whereas the constraint for the first $ForEachMember$ is $Count[5]$ and for the second is $Count[2]$.

\lstset{language=Pucella2006}
\begin{lstlisting}[frame=single, caption={Agreement 2.5 (as used in ~\cite{pucella2006})},label={lst:example25pucella2006}]
agreement
 for {Alice, Bob} 
 about ebook 
 with Count [10] $\rightarrow$ and [forEachMember[{Alice, Bob}; Count[5]] $\Rightarrow_{id1}$ display,
                                 forEachMember[{Alice, Bob}; Count[1]] $\Rightarrow_{id2}$ print].
\end{lstlisting}

The Coq version of the agreement 2.5 (listing ~\ref{lst:example25pucella2006}) and its sub-parts is listed below. It is best to start with the agreement itself called $A2.5$ in the listing and compare to the agreement 2.5 listed in ~\ref{lst:example25pucella2006}.

The agreement is that the asset \emph{ebook} may be displayed up to five times by Alice and Bob each, and printed once by each. However the total number of actions (either $display$ or $print$) justified by the two policies by either Alice and Bob is at most 10.


\lstset{language=Coq}
\begin{lstlisting}[frame=single, caption={Example 2.5},label={lst:example25}]

Definition tenCount:preRequisite := (Constraint (Count 10)).
Definition fiveCount:constraint := (Count 5).
Definition oneCount:constraint := (Count 1).

Definition prins2_5 := (NewList Alice (Single Bob)).
Definition forEach_display:preRequisite := ForEachMember prins2_5 (Single fiveCount).
Definition forEach_print:preRequisite := ForEachMember prins2_5 (Single oneCount).

Definition primPolicy1:policy := PrimitivePolicy forEach_display id1 Display.
Definition primPolicy2:policy := PrimitivePolicy forEach_print id2 Print.

Definition policySet2_5:policySet :=
  PrimitivePolicySet tenCount (AndPolicy (NewList primPolicy1 (Single primPolicy2))).
                     
Definition A2.5 := Agreement prins2_5 ebook policySet2_5.

\end{lstlisting}


\lstset{language=Coq}
\begin{lstlisting}[frame=single, caption={Example 2.6},label={lst:example26}]
Definition prins2_6 := prins2_5.

Definition aliceCount10:preRequisite := Constraint (CountByPrin (Single Alice) 10).
Definition primPolicy2_6:policy := PrimitivePolicy aliceCount10 id3 Play.
Definition policySet2_6_modified:= PrimitiveExclusivePolicySet TruePrq primPolicy2_6.
\end{lstlisting}


\lstset{language=Coq}
\begin{lstlisting}[frame=single, caption={Example 2.4},label={lst:example24}]

\end{lstlisting}


\lstset{language=Coq}
\begin{lstlisting}[frame=single, caption={Example 2.5},label={lst:example25}]

\end{lstlisting}


\lstset{language=Coq}
\begin{lstlisting}[frame=single, caption={Example 2.6},label={lst:example26}]

\end{lstlisting}


          
%======================================================================
\chapter{Some Simple Theorems}
%======================================================================

%----------------------------------------------------------------------
\section{Introduction}
%----------------------------------------------------------------------

In this chapter we will declare and prove some very simple theorems about the examples from chapter~\ref{chap:examples}. This simple introduction is only meant to give us a feel for how theorems are stated in Coq and how proofs are constructed using Coq \emph{tactics}. 

As mentioned earlier, propositions are types in Coq whose type is the sort $Prop$. Any term $t$ whose type is a proposition is a proof term or, for short, a proof. A \emph{Hypothesis} is a local declaration $h : P$ where $h$ is an identifier and $P$ is a proposition. An \emph{Axiom} is similar to a hypothesis except it is declared at the global scope and so it is always available. 

A \emph{Theorem} or \emph{Lemma} is stated by giving an identifier whose type is a proposition (~\cite{BC04}).  The proposition is the statement of the theorem or lemma.  It must be followed by a proof. Keywords ``Hypothesis'', ``Axiom'' and ``Theorem'' or ``Lemma'' are used in each case respectively. 

To build a proof in Coq the user states the proposition to prove; this is called a goal to be proved or discharge, along with some hypothesis that makes up the local context. The user then uses commands called tactics to manipulate the local context and to decompose the goal into simpler goals. The goal simplification into sub-goals will continue until all the sub-goals are solved.

In listing~\ref{lst:proofexample} we have declared a theorem called $example1$ and the corresponding proposition $forall x:nat, x < x + 1.$
 
Note that the notation $P : T$ is also used to declare program $P$ has type $T$. This duality of notation is due to Curry-Howard isomorphism which relates the two worlds of type theory and structural logic together~\cite{BC04}. Once the Theorem has been declared Coq displays the proposition to be proved under a horizontal line written --------, and displays the context of local facts and hypothesis, if any, above the horizontal line. At this point one can enter proof mode by using \emph{Proof.} upon which Coq is ready to accept tactics. Entering tactics that can break the stated goal (under the horizontal line) into one or more sub-goals is how one progresses until no goals left at which point Coq responds with ``No more subgoals''~\cite{CoqHurry}.

\lstset{language=Coq}
\begin{lstlisting}[frame=single, caption={Proof Example},label={lst:proofexample}]
Theorem example1: forall x:nat, x < x + 1.
\end{lstlisting}

In the following listings, for the sake of completeness of the presentation, we will include the Coq commands that complete the proof of the respective theorem. We will not however explain the individual commands further here. Showing the commands in the listings, is meant as an indication of the size of the proof in terms of lines of Coq script.


\section{Theorem One}

In listing~\ref{lst:theoremone} we define a $policySet$ with a $constraint$ such that if $Alice$ has used the policy with $id1$ to justify her printing $a_{1}$ times, she may do so again if $a_{1} < 5$. The agreement $AgreeCan$ simply links the asset $TheReport$ with the subject $Alice$ and the $policySet$ previously defined. 

We capture the fact that $Alice$ has used the policy with $id1$ to justify her printing $2$ times in an environment called $eA1$. Recall that environments are defined to be non-empty lists of $count_equality$ objects (see listing~\ref{lst:environmentcoq}). 

We also declare a hypothesis $H$ with the proposition that results from the translation of the agreement (see definition of $trans_agreement$ in listing~\ref{lst:transagreement}) and the environment. The proposition can be shown in Coq after some clean-up (e.g. replaced 101 by Alice) and using the form $Eval$ $compute$ : 

\lstset{language=Coq}
\begin{lstlisting}[frame=single, caption={Hypothesis for Theorem One},label={lst:theoremonehypo}]
forall x : subject, x = Alice /\ True -> 2 < 5 -> Permitted x Print TheReport.
\end{lstlisting}

The theorem $One$ that we are going to prove is trivial but nonetheless in English it states that Alice is Permitted to Print TheReport. The proof comes after the command 'Proof.' and ends with 'Qed'. 

\lstset{language=Coq}
\begin{lstlisting}[frame=single, caption={Theorem One},label={lst:theoremone}]

Definition psA1:policySet :=
  PrimitivePolicySet
    TruePrq
    (PrimitivePolicy (Constraint (Count  5)) id1 Print).

Definition AgreeCan := Agreement (Single Alice) TheReport psA1.

Definition eA1 : environment := 
  (SingleEnv (make_count_equality Alice id1 2)).

Hypothesis H: trans_agreement eA1 AgreeCan.

Theorem One: Permitted Alice Print TheReport.

Proof. simpl in H. apply H. split. reflexivity. auto.  omega. Qed.


\end{lstlisting}

\section{Theorem Two}

In listing~\ref{lst:theoremtwo} we define an exclusive policy set $policySet$ containing a policy $pol$ that allows printing. The agreement $AgreeA5$ includes the exclusive policy set to express that Bob may print $LoveAndPeace$. However any subject that is not the agreement's user (e.g. Bob) is forbidden from printing $LoveAndPeace$. 

Notice that due to the fact that environments are defined as non-empty lists, we have added a Null count to it (see $eA5$). We continue to capture the relevant facts from the environment and the agreement through defining a hypothesis (e.g. $H$). The hypothesis is shown in listing~\ref{lst:theoremtwo}. 

\lstset{language=Coq}
\begin{lstlisting}[frame=single, caption={Hypothesis for Theorem Two}, label={lst:theoremtwohypo}]
forall x : subject, 
        (x = Bob /\ True -> True -> Permitted x Print LoveAndPeace) /\
       ((x = Bob -> False) -> Permitted x Print LoveAndPeace -> False). 
\end{lstlisting}

Theorem $T1\_A5$ states the exclusivity of the policy set, namely that any subject that is not Bob is not permitted to print the asset LoveAndPeace. Theorem $T2\_A5$ uses $T1\_A5$ to prove Alice is not permitted to print the asset.

\lstset{language=Coq}
\begin{lstlisting}[frame=single, caption={Theorem Two},label={lst:theoremtwo}]

Definition prin_bob := (Single Bob).
Definition pol:policy := PrimitivePolicy TruePrq id3 Print.
Definition pol_set:policySet := PrimitiveExclusivePolicySet TruePrq pol.
Definition AgreeA5 := Agreement prin_bob LoveAndPeace pol_set.
Definition eA5 : environment := (SingleEnv (make_count_equality NullSubject NullId 0)).


Hypothesis H: trans_agreement eA5 AgreeA5.


Theorem T1_A5: forall x, x<>Bob -> ~Permitted x Print LoveAndPeace.
Proof. simpl in H. apply H. Qed.

Theorem T2_A5: ~Permitted Alice Print LoveAndPeace.
Proof. simpl in H. apply T1_A5. apply not_eq_S. omega. Qed.


End A5.


\end{lstlisting}


\section{Theorem Three}

In listing~\ref{lst:environmentcoq} we defined environments as non-empty lists of $count_equality$ objects which are in turn defined as counts per each subject, policy-id pair. These count formulas represent how many times each policy has been used to justify an action by a subject wrt a policy (specified by the policy id) and semantically it makes sense that they are unique in time. When and if two $count_equality$ objects with the same subject and policy id refer to different counts, we say we have \emph{inconsistent} count formulas. The listing~\ref{lst:inconsistentcounts} defines the binary predicate $inconsistent$.

\lstset{language=Coq}
\begin{lstlisting}[frame=single, caption={Inconsistent Count Formulas}, label={lst:inconsistentcounts}]

Definition inconsistent (f1 f2 : count_equality) : Prop :=
   match f1 with (CountEquality s1 id1 n1) =>
     match f2 with (CountEquality s2 id2 n2) =>       
       s1 = s2 -> id1 = id2 -> n1 <> n2
     end 
   end.

\end{lstlisting}


Next we would like to expand the notion of inconsistency to more than two count formulas. We first define a predicate over a count formula and an environment as in listing~\ref{lst:inconsistentcountandenv}. If the environment is a singleton then we just compare the two count formulas for inconsistency, else we build the disjunction of the inconsistency between the count formula on one hand and the head of the environment and the rest of the environment, respectively. 

\lstset{language=Coq}
\begin{lstlisting}[frame=single, caption={Inconsistent Count Formula And Environment}, label={lst:inconsistentcountandenv}]
Fixpoint formula_inconsistent_with_env (f: count_equality)
                          (e : environment) : Prop :=
  match e with
    | SingleEnv g =>  inconsistent f g
    | ConsEnv g rest => (inconsistent f g) \/ (formula_inconsistent_with_env f rest)
  end.
\end{lstlisting}

Finally we define a new inductive data type that represents $consistent$ environments (see listing~\ref{lst:inconsistentenv}). An environment is consistent if it is a singleton count formula, if it consists of only two consistent count formulas and finally if the environment consists of a consistent environment and the consistent  composition of a count formula and the consistent environment (see constructor \emph{consis_more}.

\lstset{language=Coq}
\begin{lstlisting}[frame=single, caption={Inconsistent Environment}, label={lst:inconsistentenv}]

Inductive env_consistent : environment -> Prop :=
| consis_1 : forall f, env_consistent (SingleEnv f)
| consis_2 : forall f g, ~(inconsistent f g) -> env_consistent (ConsEnv f (SingleEnv g))
| consis_more : forall f e, 
   env_consistent e -> ~(formula_inconsistent_with_env f e) -> env_consistent (ConsEnv f e).

\end{lstlisting}


We will now pose several small theorems about consistency of count formulas and environments and provide proofs for them (see listing~\ref{lst:inconsistenttheorems}).

\lstset{language=Coq}
\begin{lstlisting}[frame=single, caption={Inconsistent Environment}, label={lst:inconsistenttheorems}]

Theorem f1_and_f2_are_inconsistent: inconsistent f1 f2.
Proof. 
unfold inconsistent. simpl. omega. Qed.


Theorem f1_and___env_of_f2_inconsistent: formula_inconsistent_with_env f1 (SingleEnv f2).
Proof. 
unfold formula_inconsistent_with_env. apply f1_and_f2_are_inconsistent. Qed.


Theorem two_inconsistent_formulas_imply_env_inconsistent: 
  forall f g, inconsistent f g -> ~env_consistent (ConsEnv f (SingleEnv g)).
Proof. 
intros. unfold not. intros H'. 
inversion H'. intuition. intuition. Qed.


Theorem e2_is_inconsistent: ~env_consistent e2.
Proof.
apply two_inconsistent_formulas_imply_env_inconsistent. 
apply f1_and_f2_are_inconsistent. Qed.


Theorem env_consistent_implies_two_consistent_formulas: 
  forall (f g: count_equality), 
    env_consistent (ConsEnv f (SingleEnv g))-> ~inconsistent f g.
Proof. 
intros. inversion H. exact H1. intuition. Qed.


Theorem two_consistent_formulas_imply_env_consistent: 
  forall (f g: count_equality), 
    ~inconsistent f g -> env_consistent (ConsEnv f (SingleEnv g)).
Proof. 
intros. apply consis_2. exact H. Qed.

Theorem env_inconsistent_implies_two_inconsistent_formulas: 
  forall (f g: count_equality), 
    ~env_consistent (ConsEnv f (SingleEnv g))-> inconsistent f g.
Proof.
induction f.
induction g.
unfold inconsistent.
intros.
subst.
generalize (dec_eq_nat n n0).
intro h; elim h.
intro; subst.
elim H.
apply consis_2.
unfold inconsistent.
intro.
assert (s0=s0); auto.
assert (p0=p0); auto.
specialize H0 with (1:=H1) (2:=H2).
elim H0; auto.
auto.
Qed.


Theorem same_subjects_policyids_different_counts_means_inconsistent : forall (s1 s2: subject),
                forall (id1 id2: policyId),
                forall (n1 n2: nat),

  (s1 = s2 /\ id1 = id2 /\ n1 <> n2) -> 
  inconsistent (CountEquality s1 id1 n1) (CountEquality s2 id2 n2).

Proof. 
intros. unfold inconsistent. intros. intuition. Qed.



\end{lstlisting}
















          
%======================================================================
\chapter{Proposed Future Work}
%======================================================================

%----------------------------------------------------------------------
\section{Introduction}
%----------------------------------------------------------------------

We first mentioned queries in chapter~\ref{chap:semantics} on page~\pageref{chap:semantics}. Ultimately policy statements describing an agreement will be used to enforce those agreements. To enforce policy agreements, access queries are asked from the policy engine and access is granted or denied based on the answer.

By defining formal semantics for \ac{odrl} the authors of ~\cite{pucella2006} were able to prove that answering a query on whether access should be granted or not, is decidable and NP-hard for the full \ac{odrl}. 

In this chapter we will review our encoding of queries in Coq and Coq representations of other definitions used in ~\cite{pucella2006} which we will use to prove decidability results of our own. 


\section{Queries}

Queries are tuples of the form $(A, s, action, a, e)$ in ~\cite{pucella2006}. The tuple corresponds to the question of determining whether a set $A$ of agreements imply that a subject $s$ may perform action $action$ on an asset $a$ given the environment $e$. The Coq representation is listed in listing~\ref{lst:querycoq}. We distinguish single agreement queries from multiple agreement queries by defining two separate types: $single\_query$ and $general\_query$.

\lstset{language=Coq, frame=single, caption={Queries},label={lst:querycoq}}
\begin{minipage}[c]{0.95\textwidth}
\begin{lstlisting}

Inductive single_query : Set := 
   | SingletonQuery : agreement -> subject -> act -> asset -> environment -> single_query.
   

Inductive general_query : Set := 
   | GeneralQuery : nonemptylist agreement -> subject -> act -> asset -> environment -> general_query.
\end{lstlisting}
\end{minipage}

\section{Answering Queries}\label{sec:answerqueriesodrl}

Answering a query as defined earlier can lead to one of four outcomes: error(listing ~\ref{listing lst:errordecision}), permitted(listing~\ref{lst:permitdecision}), denied(listing~\ref{lst:denydecision}) and ``not applicable''(listing~\ref{lst:notapplicabledecision}) defined in ~\cite{Tschantz}.

\lstset{mathescape, language=AST} 
\begin{lstlisting}[frame=single, caption={Answerable Queries: Error},label={lst:errordecision}]

$(\bigwedge [\![ agreement]\!]) \land E \implies Permitted(s, act, a)$ and $(\bigwedge [\![ agreement]\!]) \land E \implies \lnot Permitted(s, act, a)$

\end{lstlisting}

\lstset{mathescape, language=AST} 
\begin{lstlisting}[frame=single, caption={Answerable Queries: Permit},label={lst:permitdecision}]

$(\bigwedge [\![ agreement]\!]) \land E \implies Permitted(s, act, a)$ and $(\bigwedge [\![ agreement]\!]) \land E \notimplies \lnot Permitted(s, act, a)$

\end{lstlisting}

\lstset{mathescape, language=AST} 
\begin{lstlisting}[frame=single, caption={Answerable Queries: Deny},label={lst:denydecision}]

$(\bigwedge [\![ agreement]\!]) \land E \notimplies Permitted(s, act, a)$ and $(\bigwedge [\![ agreement]\!]) \land E \implies \lnot Permitted(s, act, a)$

\end{lstlisting}

\lstset{mathescape, language=AST} 
\begin{lstlisting}[frame=single, caption={Answerable Queries: Not Applicable},label={lst:notapplicabledecision}]

$(\bigwedge [\![ agreement]\!]) \land E \notimplies Permitted(s, act, a)$ and $(\bigwedge [\![ agreement]\!]) \land E \notimplies \lnot Permitted(s, act, a)$

\end{lstlisting}


In ~\cite{pucella2006} a slightly different formulation is used to denote the same four decision types. ``Query Inconsistent'',``Permission Granted'', ``Permission Denied'' and ``Permission Unregulated''. The main difference is the fact that environments are explicitly taken into account in ~\cite{Tschantz} but are implicit in ~\cite{pucella2006} where the algorithm of ``holds'' takes into account the environment (notion of \emph{E-validity}).

\lstset{mathescape, language=AST} 
\begin{lstlisting}[frame=single, caption={$f^{+}_q$},label={lst:fplusformula}]

$f^{+}_q \triangleq (\bigwedge [\![ agreement]\!]) \implies Permitted(s, act, a)$ 

\end{lstlisting}

\lstset{mathescape, language=AST} 
\begin{lstlisting}[frame=single, caption={$f^{-}_q$},label={lst:fminusformula}]

$f^{-}_q \triangleq (\bigwedge [\![ agreement]\!]) \implies \lnot Permitted(s, act, a)$ 

\end{lstlisting}


\lstset{mathescape, language=AST} 
\begin{lstlisting}[frame=single, caption={Answerable Queries: Query Inconsistent},label={lst:queryinconsistentdecision}]

$f^{+}_q$ and $f^{-}_q$ both hold

\end{lstlisting}

\lstset{mathescape, language=AST} 
\begin{lstlisting}[frame=single, caption={Answerable Queries: Permission Granted},label={lst:permission granteddecision}]

$f^{+}_q$ holds and $f^{-}_q$ does not hold

\end{lstlisting}

\lstset{mathescape, language=AST} 
\begin{lstlisting}[frame=single, caption={Answerable Queries: Permission Denied},label={lst:permission denieddecision}]

$f^{+}_q$ does not hold and $f^{-}_q$ holds

\end{lstlisting}

\lstset{mathescape, language=AST} 
\begin{lstlisting}[frame=single, caption={Answerable Queries: Permission Unregulated},label={lst:permission unregulateddecision}]

$f^{+}_q$ does not hold and $f^{-}_q$ does not hold

\end{lstlisting}

\section{Machine-Checked Proof of Decidability of Queries}

By defining formal semantics for \ac{odrl} authors of ~\cite{pucella2006} were able to show some important results. First result is that answering the question of whether a set of ODRL statements imply a permission, denial or other possibilities is decidable and also that its complexity is NP-hard (see Theorem ~\ref{customthm41} from ~\cite{pucella2006} re-printed here below).

\begin{customthm}{4.1}\label{customthm41}
The problem of deciding, for a query $q = (A, s, act, a, E)$, whether $f^{+}_q$ is E-valid is decidable but NP-hard. Similarly for $f^{-}_q$.
\end{customthm}

The authors of ~\cite{pucella2006} then prove that by removing the construct $not[policySet]$ from \ac{odrl}'s syntax answering the same query remains decidable and efficient (polynomial time complexity). 

We will prove equivalent results as above starting with the decidability result of answering a query in ODRL0 (which does not include $not[policySet]$). The theorem in listing~\ref{lst:decidabilityodrl0coq} states that for all environments, all single agreements, all subjects, all actions and all assets, either permission is granted, permission is denied, permission is unregulated or query is inconsistent. 

\lstset{language=Coq, frame=single, caption={Environments and Counts},label={lst:decidabilityodrl0coq}}
\begin{minipage}[c]{0.95\textwidth}
\begin{lstlisting}
Theorem queriesAreDecidable: forall (e:environment), 
                forall (agr: agreement),
                forall (s:subject),
                forall (action:act),
                forall (a:asset),

(permissionGranted e [agr] s action a) \/
(permissionDenied e [agr] s action a)  \/
(queryInconsistent e [agr] s action a) \/
(permissionUnregulated e [agr] s action a).

\end{lstlisting}
\end{minipage}

We will then augment ODRL0 with the constructs we omitted from the full ODRL (resulting in what we have earlier called ODRL1 or ODRL2) including the troublesome construct $not[policySet]$ and attempt to prove that the decidability results remain intact. There is a chance that a proof is not possible due to particulars of the Coq encoding we have used, in in which case, we will adjust our encoding.

\section{SELinux}
 
We started out by looking at \ac{drel}s and specifically \ac{odrl} where rights expressions are used to arbitrate access to assets under conditions. Recall that the main construct in \ac{odrl} is the agreement which specifies users, asset(s) and policies (as part of policy sets) whereby controls on users' access to the assets are described. This is reminiscent of how access control conditions are expressed in access control policy languages such as \ac{xacml} and \ac{selinux}.

While \ac{xacml} is a high-level and platform independent access control system \ac{selinux} is platform dependent (e.g. Linux based) and low-level. \ac{selinux} enhances the \ac{DAC} that most unix based systems employ by \ac{MAC} where designed access control policies are applied throughout the system possibly overriding whatever \ac{DAC} is in place by the system users. 

\ac{selinux} uses Linux's extended file attributes to attach a \emph{security context} to passive entities (e.g. files, directories, sockets) and also to each active entity typically a Linux user space process. Security context is a data structure that is composed of a user, a role and a domain (or type). While users can map directly to ordinary user names they can also be defined separately. Roles are meant to group users and add flexibility when assigning permissions and are the basis for \ac{RBAC}. Finally domains or types are the basis for defining common access control requirements for both passive and active entities. 


The enforcement of \ac{selinux} policies are performed by the \emph{security server}. Whenever a security operation is requested from user space by a system call, the security server is invoked to arbitrate the operation and either allow the operation or to deny it. Each operation is identified by two pieces of information: an object class (e.g. file) and a permission (e.g. read, write). When an operation is requested to be performed on an object, the class and the permissions associated with the object along with security contexts of the source (typically the source entity is a process) and the object are passed to the security server. The security server consults the loaded policy (loaded at 
boot time) and allows or denies the access request [sarna-starosta]

\section{SELinux Policy Language}

The \ac{selinux} policy has four different kinds of statements: declarations, rules, constraints and assertions [archer]. Assertions are compile time checks that the \emph{checkpolicy} tool performs at compile time. The other three kinds of statements however are evaluated at run-time. 

Declaration statements declare a user, a role and a type. 

\lstset{language=selinux}
\begin{lstlisting}[frame=single, caption={Declarations},label={lst:declsselinux}]

user u types Ru;

role r types Tr;

type t, attrib_{1}, ..., attrib_{n};

\end{lstlisting}

Rule statements define access vector rules and type transition rules. Access vector (AV) rules (see listing~\ref{lst:avruleselinux}) specify which operations are allowed and whether to audit (log). Any operation not covered by AV rules are denied and all denied operations are logged. The semantics of the AV rule with avkind \emph{allow} is: processes with type \emph{sourcetype} are allowed to perform operations in \emph{perm} on objects with class \emph{obj-class} and type \emph{targettype}. \emph{auditallow} means to allow and audit, \emph{dontaudit} means to never audit and finally \emph{neverallow} provides a mechanism to override allow rules. When a process changes security context, the role may change, assuming a role transition rule exists relating the old and the new roles (listing~\ref{lst:roletransitionselinux}).

\lstset{language=selinux}
\begin{lstlisting}[frame=single, caption={AV rule},label={lst:avruleselinux}]

avkind sourcetype targettype:object-class perm

avkind=allow, auditallow, dontaudit, neverallow
\end{lstlisting}

Constraints are additional conditions that must hold for an attempted operation to be allowed. Constrains relate all of their arguments (the security contexts) to the server (see listing~\ref{lst:constrainselinux}. Whenever a permission is requested on an obj-class, the security server checks that the two security contexts are related by a constrain statement.

\lstset{language=selinux}
\begin{lstlisting}[frame=single, caption={Constrain rule},label={lst:constrainselinux}]

constrain classes, perms, sourcetype, sourcerole, sourceuser, targettype, targetrole, targetuser
\end{lstlisting}


\section{Agreements in \ac{selinux}}

As with \ac{odrl} we will start by limiting the policy language to only allow AV rules. As mentioned earlier an operation not covered by a allow rule is denied by \ac{selinux}. We will make up explicit \emph{deny} as required therefore an agreement is defined to be a combination of allow and deny rules. Allow and deny rules are mappings defined in listing~\ref{lst:allowmappingastselinux}.

\lstset{language=AST}
\begin{lstlisting}[frame=single, caption={allow/deny rule as a mapping},label={lst:allowmappingastselinux}]
allow/deny rule : $T \times (T \times C) \rightarrow 2^{P}$
\end{lstlisting}

\lstset{language=AST}
\begin{lstlisting}[frame=single, caption={\ac{selinux} agreement},label={lst:agreementastselinux}]
<agreement> ::= <avRule> 
\end{lstlisting}


\lstset{mathescape, language=AST, escapechar=\&}  
\begin{lstlisting}[frame=single, caption={AV Rule},label={lst:avruleastselinux}]
<avRule> ::=  
	'allow' $(T1, T2, C, P)$	&\Comment{; allow rule }&
	'deny' $(T1, T2, C, P)$	    &\Comment{; deny rule }&
	'and'[ <avRule$_{1}$>, ..., <avRule$_{m}$> ]	&\Comment{; conjunction }&
\end{lstlisting}


\section{Environments}

Environments are collections of \emph{role-type} and \emph{user-role} relations. A role-type relation $role(R, T)$ simply associates a role with a type. A user-role relation $user(U, R)$ associates a user with a role. An environment is consistent with respect to a security context $<T, R, U>$, if and only if $role(R, T)$ and $user(U, R)$ relations hold in the environment. 

\section{Queries in \ac{selinux}}

The decision problem in \ac{selinux} access control is whether an entity with security context $<T1, R1, U1>$ may perform action $P1$ to entity with object class $C1$ with security context $<T2, R2, U2>$.

To answer such queries we use the authorization relation \emph{auth(C, P, T1, R1, U1, T2, R2, U2)} which is equivalent to the $Permitted$ predicate we saw earlier for \ac{odrl}.

\lstset{mathescape, language=AST} 
\begin{lstlisting}[frame=single, caption={$f^{+}_q$ for \ac{selinux}},label={lst:fqplussel}]

$allow (T1, T2, C, P)$ $\land$ $(E$ $consistent$ $wrt$ $<T1, R1, U1>$ and $<T2, R2, U2>)$ $\implies auth(C, P, T1, R1, U1, T2, R2, U2)$ 

\end{lstlisting}

\lstset{mathescape, language=AST} 
\begin{lstlisting}[frame=single, caption={$f^{-}_q$ for \ac{selinux}},label={lst:fqminussel}]

$deny (T1, T2, C, P)$ $\lor$ $\lnot (E$ $consistent$ $wrt$ $<T1, R1, U1>$ and $<T2, R2, U2>)$ $\implies \lnot auth(C, P, T1, R1, U1, T2, R2, U2)$ 

\end{lstlisting}

\section{Decidability of Queries in \ac{selinux}}

In this thesis we will be investigating the question of decidability for answering queries in \ac{selinux} policies based on the same four outcomes we encountered earlier in ~\ref{sec:answerqueriesodrl} namely error, permitted, denied and ``not applicable''. We will first state a decidability theorem similar to theorem in listing~\ref{lst:decidabilityodrl0coq} (minor adjustments may be needed to allow for differences with \ac{selinux} policy language) and present a proof for it in Coq. The literature in the \ac{selinux} implies only two outcomes are possible: permitted or denied. We will next attempt to prove this conjecture in Coq. Finally we will add constrain relations (see listing~\ref{lst:fqplusconstsel} and listing~\ref{lst:fqminusconstsel}) to \ac{selinux} policies (which we have not included so far) and prove the same decidability results again for the augmented policy.

\lstset{mathescape, language=AST} 
\begin{lstlisting}[frame=single, caption={$f^{+}_q$ for \ac{selinux}},label={lst:fqplusconstsel}]

$allow (T1, T2, C, P)$ $\land$ $constrain (C, P, T1, R1, U1, T2, R2, U2)$ $\land$
$(E$ $consistent$ $wrt$ $<T1, R1, U1>$ and $<T2, R2, U2>)$ $\implies auth(C, P, T1, R1, U1, T2, R2, U2)$ 

\end{lstlisting}

\lstset{mathescape, language=AST} 
\begin{lstlisting}[frame=single, caption={$f^{-}_q$ for \ac{selinux}},label={lst:fqminusconstsel}]

$deny (T1, T2, C, P)$ $\lor$ $\lnot$ $constrain (C, P, T1, R1, U1, T2, R2, U2)$ $\lor$
$\lnot (E$ $consistent$ $wrt$ $<T1, R1, U1>$ and $<T2, R2, U2>)$ $\implies \lnot auth(C, P, T1, R1, U1, T2, R2, U2)$ 

\end{lstlisting}






















          
%% Some LaTeX commands I define for my own nomenclature.
% If you have to, it's better to change nomenclature once here than in a 
% million places throughout your thesis!
\newcommand{\package}[1]{\textbf{#1}} % package names in bold text
\newcommand{\cmmd}[1]{\textbackslash\texttt{#1}} % command name in tt font 


%======================================================================
\chapter{Observations}
%======================================================================

This would be a good place for some figures and tables.

Some notes on figures and photographs\ldots

\begin{itemize}
\item A well-prepared PDF should be 
  \begin{enumerate}
    \item Of reasonable size, {\it i.e.} photos cropped and compressed.
    \item Scalable, to allow enlargment of text and drawings. 
  \end{enumerate} 
\item Photos must be bit maps, and so are not scaleable by definition. TIFF and
BMP are uncompressed formats, while JPEG is compressed. Most photos can be
compressed without losing their illustrative value.
\item Drawings that you make should be scalable vector graphics, \emph{not} 
bit maps. Some scalable vector file formats are: EPS, SVG, PNG, WMF. These can
all be converted into PNG or PDF, that pdflatex recognizes. Your drawing 
package probably can export to one of these formats directly. Otherwise, a 
common procedure is to print-to-file through a Postscript printer driver to 
create a PS file, then convert that to EPS (encapsulated PS, which has a 
bounding box to describe its exact size rather than a whole page). 
Programs such as GSView (a Ghostscript GUI) can create both EPS and PDF from PS files.
Appendix~\ref{ch:Appendix-Matlab} shows how to generate properly sized Matlab plots and save them as PDF.
\item It's important to crop your photos and draw your figures to the size that
you want to appear in your thesis. Scaling photos with the 
includegraphics command will cause loss of resolution. And scaling down 
drawings may cause any text annotations to become too small.
\end{itemize}
 
For more information on \LaTeX\, see the uWaterloo Skills for the Academic Workplace 
course notes at \href{http://saw.uwaterloo.ca/latex}{saw.uwaterloo.ca/latex}. 
\footnote{
Note that while it is possible to include hyperlinks to external documents,
it is not wise to do so, since anything you can't control may change over time. 
It \emph{would} be appropriate and necessary to provide external links to 
additional resources for a multimedia ``enhanced'' thesis. 
But also note that if the \package{hyperref} package is not included, 
as for the print-optimized option in this thesis template, any \cmmd{href} 
commands in your logical document are no longer defined.
A work-around employed by this thesis template is to define a dummy \cmmd{href} 
command (which does nothing) in the preamble of the document, 
before the \package{hyperref} package is included. 
The dummy definition is then redifined by the
\package{hyperref} package when it is included.
}

%The classic book by Leslie Lamport~\cite{lamport.book}, author of \LaTeX , is worth a look too, and the many available add-on packages are described by 
%Goossens \textit{et~al.}~\cite{goossens.book}. Some on-line documentation is linked
%to from \href{http://saw.uwaterloo.ca/latex}{saw.uwaterloo.ca/latex}. 



Here is an example of how to include figures in \LaTeX. 
Figure~\ref{fig.beam} shows a cantilever beam of circular cross-section
subjected to a point load and a uniformly distributed load, both of which are uncertain. Note that it is better not to include the extension of the figure's source file.

\begin{figure}[!htbp]
 \begin{center}
  \includegraphics[clip=true]{figures/beam}
 \end{center}
\caption{Cantilever Beam}
\label{fig.beam}
\end{figure}



%----------------------------------------------------------------------
\section{Adding Nomenclature}
%----------------------------------------------------------------------

The following example is part of the ``nomentbl'' package. Refer to the package's documentation for more details.

\bigskip

\noindent
Let's start with equations to show how to use greek and mathematical symbols within Nomenclature.

Here is an equation
%
\begin{equation}\label{eq:heatflux}
  \dot{Q} = k \cdot A \cdot \Delta T
\end{equation}%
%
%% Greek and math symbols
\nomenclature[gQ]{$\dot{Q}$}{heat flux}{W}{}%
\nomenclature[gk]{$k$}{overall heat transfer coefficient}{$\frac{\mathrm{W}}{\mathrm{m}^2\mathrm{K}}$}{see eq.~(\ref{eq:ohtc})}%
\nomenclature[gA]{$A$}{area}{m$^2$}{$L^2$}%
\nomenclature[gL]{$L$}{length}{m}{SI base quantity}%
\nomenclature[gT]{$T$}{temperature}{K}{SI base quantity}%
\nomenclature[gT]{$\Delta T$}{temperature difference}{K}{SI base quantity}%

Here is another one
%
\begin{equation}\label{eq:ohtc}
  \frac{1}{k} = \left[\frac{1}{\alpha _{\mathrm{i}}\,r_{\mathrm{i}}} +
    \sum^n_{j=1}\frac{1}{\lambda _j}\,
    \ln \frac{r_{\mathrm{a},j}}{r_{\mathrm{i},j}} +
    \frac{1}{\alpha _{\mathrm{a}}\,
      r_{\mathrm{a}}}\right] \cdot r_{\mathrm{reference}}
\end{equation}%
%
%% Greek and math symbols
\nomenclature[ga]{$\alpha$}{convection heat transfer coefficient}{$\frac{\mathrm{W}}{\mathrm{m}^2\mathrm{K}}$}{}%
\nomenclature[gl]{$\lambda$}{thermal conductivity}{$\frac{\mathrm{W}}{\mathrm{m K}}$}{}%
%
%% Subscripts
\nomenclature[za]{a}{out}{}{}%
\nomenclature[zi]{i}{in}{}{}%
\nomenclature[zj]{$j$}{running parameter}{}{}% 
\nomenclature[zn]{$n$}{number of walls}{}{}%


\bigskip

\noindent
The following example is to show how to use abbreviations within the Nomenclature.\\
EECS is a school at the UO.
%% Abbreviations
\nomenclature[a]{EECS}{Electrical Engineering and Computer Science}{}{}%
\nomenclature[a]{UO}{University of Ottawa}{}{}%



\bigskip

\noindent
Don't forget to run:
\begin{verbatim}
makeindex -s nomentbl.ist -o uottawa-thesis.nls uottawa-thesis.nlo
\end{verbatim}



%%% Local Variables: 
%%% mode: latex
%%% TeX-master: "../uottawa-thesis"
%%% End: 



%----------------------------------------------------------------------
% APPENDICES
%---------------------------------------------------------------------- 
%\appendix
% Designate with \appendix declaration which just changes numbering style 
% from here on
% Add a title page before the appendices and a line in the Table of Contents
%\chapter*{APPENDICES}
\addcontentsline{toc}{chapter}{APPENDICES} 
%
%% An appendix
%======================================================================
\chapter{Sources of Information and Help}
\label{ch:Appendix-Sources-of-Info}
%======================================================================
The best source of information about \LaTeX\ is the two books mentioned in this course \cite{lamport.book,goossens.book}.
Another excellent resource is the UseNet newsgroup \verb=comp.text.tex=.
A frequently-asked-questions (FAQ) list is also maintained by this news group.
You might also search the World Wide Web for ``LaTeX'' for other sources of help.
 %"Sources of Information and Help"
%% An appendix
%======================================================================
\chapter[PDF Plots From Matlab]{Matlab Code for Making a PDF Plot}
\label{ch:Appendix-Matlab} 
%======================================================================
\section{Using the GUI}
Properties of Matab plots can be adjusted from the plot window via a graphical interface. Under the Desktop menu in the Figure window, select the Property Editor. You may also want to check the Plot Browser and Figure Palette for more tools. To adjust properties of the axes, look under the Edit menu and select Axes Properties.

To set the figure size and to save as PDF or other file formats, click the Export Setup button in the figure Property Editor.

\section{From the Command Line} 
All figure properties can also be manipulated from the command line. Here's an example: 
\begin{verbatim}
x=[0:0.1:pi];
hold on % Plot multiple traces on one figure
plot(x,sin(x))
plot(x,cos(x),'--r')
plot(x,tan(x),'.-g')
title('Some Trig Functions Over 0 to \pi') % Note LaTeX markup!
legend('{\it sin}(x)','{\it cos}(x)','{\it tan}(x)')
hold off
set(gca,'Ylim',[-3 3]) % Adjust Y limits of "current axes"
set(gcf,'Units','inches') % Set figure size units of "current figure"
set(gcf,'Position',[0,0,6,4]) % Set figure width (6 in.) and height (4 in.)
cd n:\thesis\plots % Select where to save
print -dpdf plot.pdf % Save as PDF
\end{verbatim}  %"Matlab Code for Making a PDF Plot"

%----------------------------------------------------------------------
% END MATERIAL
%----------------------------------------------------------------------

% B I B L I O G R A P H Y
% -----------------------
%
% The following statement selects the style to use for references.  It controls the sort order of the entries in the bibliography and also the formatting for the in-text labels.
\bibliographystyle{plain}
% This specifies the location of the file containing the bibliographic information.  
% It assumes you're using BibTeX (if not, why not?).
\ifthenelse{\boolean{PrintVersion}}{
\cleardoublepage % This is needed if the book class is used, to place the anchor in the correct page,
                 % because the bibliography will start on its own page.
}{
\clearpage       % Use \clearpage instead if the document class uses the "oneside" argument
}
\phantomsection  % With hyperref package, enables hyperlinking from the table of contents to bibliography             
% The following statement causes the title "References" to be used for the bibliography section:
\renewcommand*{\bibname}{References}

% Add the References to the Table of Contents
\addcontentsline{toc}{chapter}{\textbf{References}}

\bibliography{bibliography/keylatex}
% Tip 5: You can create multiple .bib files to organize your references. 
% Just list them all in the \bibliogaphy command, separated by commas (no spaces).


%----------------------------------------------------------------------
\end{document}
%======================================================================



%%% Local Variables: 
%%% mode: latex
%%% TeX-master: t
%%% End: 

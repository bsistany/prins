% uOttawa (unofficial) Thesis Template for LaTeX 
% Edited by Wail Gueaieb based on Stephen Carr's uWaterloo Template

% DON'T USE THIS TEMPLATE IF YOU DON'T KNOW WHAT YOU'RE DOING!
% Remember, it comes WITH NO WARRANTY!

% Please read the "00readme.txt" file first.
% Here is how to use this template:
%
% DON'T FORGET TO ADD YOUR OWN NAME AND TITLE in the "hyperref" package
% configuration in the "thesis-preample.tex" file. THIS INFORMATION GETS 
% EMBEDDED IN THE PDF FINAL PDF DOCUMENT.
% You can view the information if you view Properties of the PDF document.

% The template is based on the standard "book" document class which provides 
% all necessary sectioning structures and allows multi-part theses.

% DISCLAIMER
% To the best of our knowledge, this template satisfies the current 
% uOttawa thesis requirements.
% However, it is your responsibility to assure that you have met all 
% requirements of the university and your particular department.
% Many thanks to the feedback from many graduates that assisted the 
% development of this template.

% -----------------------------------------------------------------------

% When using pdflatex, by default the output is geared toward generating a PDF 
% version optimized for viewing on an electronic display, including 
% hyperlinks within the PDF.
 
% E.g. to process a thesis based on this template, run:

% (pdf)latex uottawa-thesis	-- first pass of the (pdf)latex processor
% bibtex uottawa-thesis 	-- generates bibliography from .bib data file(s) 
% (pdf)latex uottawa-thesis	-- fixes cross-references, bibliographic references, etc
% (pdf)latex uottawa-thesis	-- fixes cross-references, bibliographic references, etc
% makeindex -s nomentbl.ist -o uottawa-thesis.nls uottawa-thesis.nlo
% (pdf)latex uottawa-thesis	-- fixes cross-references, bibliographic references, etc
% (pdf)latex uottawa-thesis	-- fixes cross-references, bibliographic references, etc



% N.B. The "pdftex" program allows graphics in the following formats to be
% included with the "\includegraphics" command: PNG, PDF, JPEG, TIFF
% Tip 1: Generate your figures and photos in the size you want them to appear
% in your thesis, rather than scaling them with \includegraphics options.
% Tip 2: Any drawings you do should be in scalable vector graphic formats:
% SVG, PNG, WMF, EPS and then converted to PNG or PDF, so they are scalable in
% the final PDF as well.
% Tip 3: Photographs should be cropped and compressed so as not to be too large.

% To create a PDF output that is optimized for double-sided printing: 
%
% 1) comment-out the \documentclass statement in the preamble below, and
% un-comment the second \documentclass line.
%
% 2) change the value assigned below to the boolean variable
% "PrintVersion" from "false" to "true".

% --------------------- Start of Document Preamble -----------------------

% Specify the document class, default style attributes, and page dimensions
% For hyperlinked PDF, suitable for viewing on a computer, use this:
\documentclass[letterpaper,12pt,titlepage,oneside,final]{book}
 
% For PDF, suitable for double-sided printing, change the PrintVersion variable below
% to "true" and use this \documentclass line instead of the one above:
% \documentclass[letterpaper,12pt,titlepage,openright,twoside,final]{book}


% This package allows if-then-else control structures.
\usepackage{ifthen}
\newboolean{PrintVersion}
%\setboolean{PrintVersion}{false} 
% \setboolean{PrintVersion}{true} 
% CHANGE THIS VALUE TO "true" as necessary, to improve printed results 
% for hard copies by overriding some options of the hyperref package.


% Load your needed packages and other commands of yours.
% Load your needed packages and other commands of yours here:
%\usepackage{} % ... note that old .sty files can be included here
\usepackage[]{inputenc}
\usepackage[T1]{fontenc}
\usepackage{fullpage}

\usepackage{coqdoc}
\usepackage{color}
\usepackage{listings}
\usepackage{etoolbox}
\usepackage{fixltx2e}
\usepackage{syntax}
\usepackage{lstcoq}
\usepackage{url}
\usepackage{acro}

\definecolor{maroon}{rgb}{0.5,0,0}
\definecolor{darkgreen}{rgb}{0,0.5,0}
\lstdefinelanguage{XML}
{
  basicstyle=\ttfamily,
  frame=single,
  breaklines=true,
  morestring=[s]{"}{"},
  morecomment=[s]{?}{?},
  morecomment=[s]{!--}{--},
  commentstyle=\color{darkgreen},
  moredelim=[s][\color{black}]{>}{<},
  moredelim=[s][\color{red}]{\ }{=},
  stringstyle=\color{blue},
  identifierstyle=\color{maroon}
}
\lstdefinelanguage{Pucella2006}
{
  morekeywords={
    agreement, prin, asset, with, prePay, and, display, print, count
  }
}
\lstdefinelanguage{selinux}
{
  morekeywords={
    type, types, attrib, role, allow, user, constrain, avkind, 
	sourcetype, targettype, object-class, perm, allow, auditallow, dontaudit, neverallow,
	constrain, classes, perms, sourcetype, sourcerole, sourceuser, targettype, targetrole, targetuser
  }
}

\lstdefinelanguage{AST}
{
  basicstyle=\ttfamily,
  breaklines=true,
  morekeywords={
    agreement, prin, asset, subject, policySet, policy, act, policyId, preRequisite 
  }
  tabsize=1,
}

\newtheorem{innercustomthm}{Theorem}
\newenvironment{customthm}[1]
  {\renewcommand\theinnercustomthm{#1}\innercustomthm}
  {\endinnercustomthm}












%--------------------------------------------------------------------------
% Do NOT edit the rest of the preample UNLESS YOU KNOW WHAT YOU'RE DOING!
%--------------------------------------------------------------------------

\ifthenelse{\boolean{PrintVersion}}{
\usepackage[top=1in,bottom=1in,left=0.75in,right=1.25in]{geometry}   % For twoside document
}{
\usepackage[top=1in,bottom=1in,left=0.75in,right=1.25in]{geometry}   % For oneside document
}

\usepackage{amsmath,amssymb,amstext} % Lots of math symbols and environments
\usepackage{graphicx} % For including graphics 

\usepackage{nomentbl} 
\makenomenclature 

\usepackage{ifpdf}

\setcounter{secnumdepth}{4}
\setcounter{tocdepth}{4}

%\newcommand{\href}[1]{#1} % does nothing, but defines the command so the
    % print-optimized version will ignore \href tags (redefined by hyperref pkg).
%\newcommand{\texorpdfstring}[2]{#1} % does nothing, but defines the command
% Anything defined here may be redefined by packages added below...


% Hyperlinks make it very easy to navigate an electronic document.
% In addition, this is where you should specify the thesis title
% and author as they appear in the properties of the PDF document.
% Use the "hyperref" package 
% N.B. HYPERREF MUST BE THE LAST PACKAGE LOADED; ADD ADDITIONAL PKGS ABOVE
%\usepackage[\ifpdf pdftex,\fi letterpaper=true,pagebackref=false]{hyperref} % with basic options
		% N.B. pagebackref=true provides links back from the References to the body text. This can cause trouble for printing.
%\hypersetup{
%    plainpages=false,       % needed if Roman numbers in frontpages
%    pdfpagelabels=true,     % adds page number as label in Acrobat's page count
%    bookmarks=true,         % show bookmarks bar?
%    unicode=false,          % non-Latin characters in Acrobat's bookmarks
%    pdftoolbar=true,        % show Acrobat's toolbar?
%    pdfmenubar=true,        % show Acrobat's menu?
%    pdffitwindow=false,     % window fit to page when opened
%    pdfstartview={FitH},    % fits the width of the page to the window
%%    pdftitle={uOttawa\ LaTeX\ Thesis\ Template},    % title: CHANGE THIS TEXT!
%%    pdfauthor={Author},    % author: CHANGE THIS TEXT! and uncomment this line
%%    pdfsubject={Subject},  % subject: CHANGE THIS TEXT! and uncomment this line
%%    pdfkeywords={keyword1} {key2} {key3}, % list of keywords, and uncomment this line if desired
%    pdfnewwindow=true,      % links in new window
%    colorlinks=true,        % false: boxed links; true: colored links
%    linkcolor=blue,         % color of internal links
%    citecolor=green,        % color of links to bibliography
%    filecolor=magenta,      % color of file links
%    urlcolor=cyan           % color of external links
%}

%\ifthenelse{\boolean{PrintVersion}}{   % for improved print quality, change some hyperref options
%\hypersetup{	% override some previously defined hyperref options
%%    colorlinks,%
%    citecolor=black,%
%    filecolor=black,%
%    linkcolor=black,%
%    urlcolor=black}
%}{} % end of ifthenelse (no else)




% This is where thesis margins and spaces are set.
% Setting up the page margins...
% A minimum of 1 inch (72pt) margin at the
% top, bottom, and outside page edges and a 1.125 in. (81pt) gutter
% margin (on binding side). While this is not an issue for electronic
% viewing, a PDF may be printed, and so we have the same page layout for
% both printed and electronic versions, we leave the gutter margin in.
% Set margins:
\setlength{\marginparwidth}{0pt} % width of margin notes
% N.B. If margin notes are used, you must adjust \textwidth, \marginparwidth
% and \marginparsep so that the space left between the margin notes and page
% edge is less than 15 mm (0.6 in.)
\setlength{\marginparsep}{0pt} % width of space between body text and margin notes
\setlength{\evensidemargin}{0.125in} % Adds 1/8 in. to binding side of all 
% even-numbered pages when the "twoside" printing option is selected
\setlength{\oddsidemargin}{0.125in} % Adds 1/8 in. to the left of all pages
% when "oneside" printing is selected, and to the left of all odd-numbered
% pages when "twoside" printing is selected
\setlength{\textwidth}{6.375in} % assuming US letter paper (8.5 in. x 11 in.) and 
% side margins as above
\raggedbottom

% The following statement specifies the amount of space between
% paragraphs. Other reasonable specifications are \bigskipamount and \smallskipamount.
\setlength{\parskip}{\medskipamount}

% The following statement controls the line spacing.  The default
% spacing corresponds to good typographic conventions and only slight
% changes (e.g., perhaps "1.2"), if any, should be made.
\renewcommand{\baselinestretch}{1} % this is the default line space setting

% By default, each chapter will start on a recto (right-hand side)
% page.  We also force each section of the front pages to start on 
% a recto page by inserting \cleardoublepage commands.
% In many cases, this will require that the verso page be
% blank and, while it should be counted, a page number should not be
% printed.  The following statements ensure a page number is not
% printed on an otherwise blank verso page.
\let\origdoublepage\cleardoublepage
\newcommand{\clearemptydoublepage}{%
  \clearpage{\pagestyle{empty}\origdoublepage}}
\let\cleardoublepage\clearemptydoublepage



%======================================================================
%   L O G I C A L    D O C U M E N T -- the content of your thesis
%======================================================================
\begin{document}

% For a large document, it is a good idea to divide your thesis
% into several files, each one containing one chapter.
% To illustrate this idea, the "front pages" (i.e., title page,
% declaration, borrowers' page, abstract, acknowledgements,
% dedication, table of contents, list of tables, list of figures,
% nomenclature).
%----------------------------------------------------------------------
% FRONT MATERIAL
%----------------------------------------------------------------------
%
% C O V E R  P A G E
% ------------------
\newcommand{\thesisauthor}{Bahman Sistany}
\newcommand{\thesisadvisor}{Amy Felty}
\newcommand{\thesistitlecoverpage}{%
  Certifying Digital Rights' Expression Languages
}
\newcommand{\degree}{Ph.D.} % possible values are:
                            
\newcommand{\nameofprogram}{Electrical and Computer Engineering}
\newcommand{\academicunit}{School of Electrical Engineering and Computer Science}
\newcommand{\faculty}{Faculty of Engineering}
\newcommand{\graduationyear}{2014}
%
% T I T L E   P A G E
% -------------------
% Last updated May 24, 2011, by Stephen Carr, IST-Client Services
% The title page is counted as page `i' but we need to suppress the
% page number.  We also don't want any headers or footers.
\pagestyle{empty}
\pagenumbering{roman}

% The contents of the title page are specified in the "titlepage"
% environment.
\begin{titlepage}
        \begin{center}
        \vspace*{1.0cm}

        \Huge
        {\bf \thesistitlecoverpage }

        \vspace*{1.0cm}

        \normalsize
        by \\

        \vspace*{1.0cm}

        \Large
        \thesisauthor \\

        \vspace*{3.0cm}

        \normalsize
        \degree Thesis Proposal 
        For the \degree~degree in\\
        \nameofprogram\\

        \vspace*{2.0cm}

        \academicunit\\
        \faculty\\
        University of Ottawa\\

        \vspace*{1.0cm}
        %\copyright~\thesisauthor, Ottawa, Canada, \graduationyear\\
        \degree Thesis Advisor: ~\thesisadvisor
        \end{center}
\end{titlepage}

% The rest of the front pages should contain no headers and be numbered using Roman numerals starting with `ii'
\pagestyle{plain}
\setcounter{page}{2}

\cleardoublepage % Ends the current page and causes all figures and tables that have so far appeared in the input to be printed.
% In a two-sided printing style, it also makes the next page a right-hand (odd-numbered) page, producing a blank page if necessary.

%
%
% R E S T  O F  F R O N T  P A G E S
% ----------------------------------
% % D E C L A R A T I O N   P A G E
% -------------------------------
  % This page is not needed for a uOttawa thesis. Don't include it.
  % It is designed for an electronic thesis.
  \noindent
I hereby declare that I am the sole author of this thesis. This is a true copy of the thesis, including any required final revisions, as accepted by my examiners.

  \bigskip
  
  \noindent
I understand that my thesis may be made electronically available to the public.

\cleardoublepage
%\newpage
 %This is not needed in a uOttawa thesis.
%
% Edit the following 3 files with your abstract, acknowledgements, 
% and dedication.
% A B S T R A C T
% ---------------

\begin{center}\textbf{Abstract}\end{center}




\cleardoublepage
%\newpage

% A C K N O W L E D G E M E N T S
% -------------------------------

\begin{center}\textbf{Acknowledgements}\end{center}

I would like to thank all the little people who made this possible.


\cleardoublepage
%\newpage
% D E D I C A T I O N
% -------------------

\begin{center}\textbf{Dedication}\end{center}




\cleardoublepage
%\newpage

%
%
% No need to edit this file.
% T A B L E   O F   C O N T E N T S
% ---------------------------------
\renewcommand\contentsname{Table of Contents}
\tableofcontents
\cleardoublepage
\phantomsection
%\newpage

% L I S T   O F   T A B L E S
% ---------------------------
%\addcontentsline{toc}{chapter}{List of Tables}
%\listoftables
%\cleardoublepage
%\phantomsection		% allows hyperref to link to the correct page
%\newpage

% L I S T   O F   F I G U R E S
% -----------------------------
%\addcontentsline{toc}{chapter}{List of Figures}
%\listoffigures
%\cleardoublepage
%\phantomsection		% allows hyperref to link to the correct page
%\newpage

% L I S T   O F   LISTINGS
% -----------------------------
\addcontentsline{toc}{chapter}{List of Listings}
\lstlistoflistings
\cleardoublepage
\phantomsection		% allows hyperref to link to the correct page
\newpage

%
% No need to edit this file. But you may want to comment the whole line if you
% don't have or want a Nomenclature section.
%% L I S T   O F   S Y M B O L S
% -----------------------------
% To include a Nomenclature section
\addcontentsline{toc}{chapter}{\textbf{Nomenclature}}

\renewcommand{\nomname}{Nomenclature}
\renewcommand{\nomAname}{\textbf{\large Abbreviations}}
\renewcommand{\nomGname}{\textbf{\large Mathematical Symbols}}
\renewcommand{\nomXname}{\textbf{\large Superscripts}}
\renewcommand{\nomZname}{\textbf{\large Subscripts}}

\printnomenclature
\cleardoublepage
\phantomsection % allows hyperref to link to the correct page
% \newpage






%%% Local Variables: 
%%% mode: latex
%%% TeX-master: "../uottawa-thesis"
%%% End:   


% Change page numbering back to Arabic numerals
\pagenumbering{arabic}

%----------------------------------------------------------------------
% MAIN BODY
%---------------------------------------------------------------------- 
% Chapters 
% Include your "sub" source files here (must have extension .tex)
%======================================================================
\chapter{Introduction}
%======================================================================
Digital rights management, \emph{DRM}, refers to the digital management of rights associated with the access or usage of digital assets. There are various aspects of rights management however. According to the authors of the whitepaper ``A digital rights management ecosystem model for the education community,'' digital rights management systems cover the following four areas: 1) defining rights 2) distributing/acquiring rights 3) enforcing rights and 4) tracking usage \cite{collier}. 

Rights Expression Languages, \emph{RELs}, or more precisely when dealing with digital assets, Digital Rights Expression Languages \emph{DRELs} deal with the ``rights definition'' aspect of the DRM ecosystem. A DREL, allows the expression and definition of digital asset usage rights such that other areas of the DRM ecosystem namely the enforcement mechanism and the usage tracking components can function correctly.

Currently the most popular RELs are the eXtensible rights Markup Language, \emph{XrML} [bib], and the Open Digital Rights Language, \emph{ODRL} [bib]. Both of these languages are XML based and are considered declarative languages. XrML has been selected to be the REL for \emph{MPEG-21} which is an ISO standard for multimedia applications. ODRL is also a standards based REL which has been accepted as part of the W3C community with the mandate of standardizing how rights and policies, related to the usage of digital content on the Open Web Platform, \emph{OWP}, are expressed [wikipedia]. ODRL 2.0 supports expression of rights and also privacy rules for social media while ODRL 1.0 was only dealing with the mobile ecosystem -- ODRL 1.0 was adopted by the Open Mobile Alliance, \emph{OMA} in 2000.

As popular as both XrML and ODRL are, their adoption and usage is still somewhat limited in practice. Both Apple and Microsoft for example have defined their own lightweight RELs [problem with RELs paper] in \emph{Fair Play} (Apple) and in \emph{PlayReady} (Microsoft). The authors of [the problem with RELs] argue that both these RELs and other ones are simply too complex to be used effectively since they try to cover much of the DRM ecosystem. 

Another issue with the current batch of RELs are due to their semantics being expressed in a natural language (e.g. English). By necessity natural languages are ambiguous and open to interpretation. 

To formalize the semantics of RELS several approaches have been attempted by various authors. The main categories are logic based, operational semantics based interpreters and finally web ontology based (from the Knowledge Representation Field). In this thesis we will focus on the logic based approach to formalizing semantics and will study a specific logic based language that is a translation from a subset of ODRL.









%----------------------------------------------------------------------
\section{Logic Based Semantics for ODRL}
%----------------------------------------------------------------------

%See equation \ref{eqn_pi} on page \pageref{eqn_pi}.\footnote{A famous %equation.}

Formal logic can represent the statements and facts we express in a natural language like English. Propositional logic is expressive enough to express simple facts as propositions and allows uses connectives to allow for the negation, conjunction and disjunction of the facts. However propositional logic is not expressive enough to express policies of the kind used in languages like ODRL and XrML. For example, a simple policy expressed in English like ``All who pay 5 dollars can watch the movie Toy Story'' cannot be expressed in propositional logic because the concept of  variables doesn't exist. 

The higher order logic called ``Predicate Logic'' or ``First Order Logic'' \emph{FOL} is more suitable and has the expressive power to represent policies written in English. Moreover, FOL can be used to capture the meaning of policies in an unambiguous way.

Halpern and Weissman [Using First Order Logic to Reason about Policies] propose a fragment of FOL to represent and reason about policies. The fragment of FOL they arrive at is called \emph{Lithium} which is decidable and allows for efficiently answering interesting queries. Lithium restricts policies to be written based on the concept of ``bipolarity'' which disallows by construction policies that both permit and deny an action on an object.

\section{Pucella 2006}
Pucella and Weissman \cite{pucella2006} specify a predicate logic based based language that represents a subset of ODRL.

\section{what will I do?}


\subsection{Coq}


	• Program correctness
	• Formal verification of software
	• Certified programs
	• Proof assistant
	• Interactive and mechanized theorem proving
	• Examples of machine assisted proofs: CompCert, four-color theorem proof
	• Coq is based on a higher-order functional programming language
	• Dependent Types
		○ Subset types
		○ Easier than writing explicit proofs
	• Write formal specification and proofs that programs comply to their specification (a-short-intro-to-coq)
	• Automatically extract code from specifications as Ocaml or Haskell (a-short-intro)
	• Properties, programs and proofs are all formalized in the same language called CIC (Calculus of inductive Constructions). (a-short-intro)
	• Coq uses a sort called Prop for propositions
	• Coq art:
	• Well-formed propositions are assertions  one can express about values such as mathematical objects or even programs e.g. 3 < 8
		○ Note that assertions may be true, false or simply conjectures
		○ An assertion is only true in general if a proof is provided
		○ However hand written proofs are difficult to verify
		○ Coq provides an environment for developing proofs including a formal language to express proofs in, the language itself being built using proof theory making it possible to step by step verification of the proofs
		○ Mechanized proof verification requires a "proof" that the verification algorithm is correct itself in applying all the formal rules correctly







\section{Abstract Syntax}

\cite{pucella2006} uses abstract syntax instead of XML to express statements in the ODRL language. The abstract syntax used is a more compact representation than XML based language ODRL policies are written in and furthermore it simplifies specifying the semantics as we shall see. As an example here is an agreement written in ODRL and the comparable agreement expressed in the abstract syntax \cite{pucella2006}.

\lstset{language=XML}
\begin{lstlisting}[caption={agreement for Mary Smith in XML},label={lst:agreementxml}]
<agreement>
 <asset> <context> <uid> Treasure Island </uid> </context> </asset>
 <permission>
   <display>
    <constraint>
     <cpu> <context> <uid> Mary's computer </uid> </context> </cpu>
    </constraint>
   </display>
   <print>
    <constraint> <count> 2 </count> </constraint>
   </print>
  <requirement>
   <prepay>
    <payment> <amount currency="AUD"> 5.00</amount> </payment>
   </prepay>
  </requirement>
 </permission>
 <party> <context> <name> Mary Smith </name> </context> </party>
</agreement>
\end{lstlisting}

The agreement ~\ref{lst:agreementxml} is shown below using the syntax from \cite{pucella2006}.

\lstset{language=Pucella2006}
\begin{minipage}[c]{0.95\textwidth}
\begin{lstlisting}[frame=single, caption={agreement for Mary Smith as BNF (as used in ~\cite{pucella2006})},label={lst:agreementpucella2006}]
agreement
 for Mary Smith 
 about Treasure Island 
 with prePay[5.00] -> and[cpu[Mary's Computer] => display,
                                      count[2] => print].
\end{lstlisting}
\end{minipage} 

% \emph{prin\textsubscript{u}}
In the following we will cover the \emph{abstract syntax} of a subset of ODRL expressed as Coq's constructs such as \emph{Inductive Types} and Definitions. We will call this subset \emph{ODRL0} both because it is a variation of Pucella's ODRL language and also because it is missing some ODRL constructs such as \emph{Requirements} and \emph{Conditions} - we will add the missing pieces making up what we will call \emph{ODRL1} and perhaps \emph{ODRL2} (the latter only if needed). We will also describe ODRL0 in a \emph{BNF} grammar that looks more like Pucella's ODRL grammar. BNF style grammars are less formal as they give some suggestions about the surface syntax of expressions [Pierce1] without getting into lexical analysis and parsing related aspects such as precedence order of operators. The Coq version in contrast is more formal and could be directly used for building compilers and interpreters. We will present both the BNF version and the Coq version for each construct of ODRL0 [Pierce1]. To get started let's see what the listing ~\ref{lst:agreementpucella2006} would look like in ODRL0's Coq version.

\lstset{language=Coq}
\begin{lstlisting}[frame=single, caption={Coq version of agreement for Mary Smith},label={lst:marysmithagreementcoq}]

Agreement (Single MarySmith) Treasure Island 
 (PrimitivePolicySet (Constraint (PrePay 5.00))
  (AndPolicy 
   (NewList (PrimitivePolicy (Constraint 
                              (Principal 
                               (Single MarysComputer))) id1 Display)
   (Single (PrimitivePolicy (Constraint (Count 2)) id2 Print))))).
\end{lstlisting}


%\coqdoceol
%\coqdocvar{Agreement} (\coqdocvar{Single} \coqdocvar{Mary Smith}) \coqdocvar{Treasure Island} (\coqdocvar{PrimitivePolicySet} (\coqdocvar{Constraint} (\coqdocvar{PrePay} 5.00))\coqdoceol
%\coqdocindent{0.50em}
%(\coqdocvar{AndPolicy} (\coqdocvar{NewList} (\coqdocvar{PrimitivePolicy} (\coqdocvar{Constraint} (\coqdocvar{Principal} (\coqdocvar{Single} \coqdocvar{Mary's Computer}))) \coqdocvar{id1} \coqdocvar{Display})\coqdoceol
%\coqdocindent{6.00em}
%(\coqdocvar{Single} (\coqdocvar{PrimitivePolicy} (\coqdocvar{Constraint} (\coqdocvar{Count} 2)) \coqdocvar{id2} \coqdocvar{Print}))))).\coqdoceol
%\coqdocemptyline
%\coqdocemptyline

The top level ODRL0 production is the \emph{agreement}. An agreement expresses what actions a set of subjects may perform on an object and under what conditions. Syntactically an agreement is composed of a set of subjects/users called a \emph{principal} (\emph{prin}), an \emph{asset} and a \emph{Policy Set} (\emph{PolicySet}).

% agreement
\lstset{language=AST}
\begin{lstlisting}[frame=single, caption={agreement},label={lst:agreementast}]
<agreement> ::= 'agreement' 'for' <prin> 'about' <asset> 'with' <policySet> 
\end{lstlisting}

Principals or prins are composed of \emph{subjects} which are specified based on the application e.g. Alice, Bob, etc for the DRM application we will be using throughout.

% prin
\lstset{mathescape, language=AST}  
\begin{lstlisting}[frame=single, caption={prin},label={lst:prinast}]
<prin> ::=  { <subject$_{1}$>, ..., <subject$_{m}$> }
\end{lstlisting}

% subject
\lstset{mathescape, language=AST}  
\begin{lstlisting}[frame=single, caption={subject},label={lst:subjectast}]
<subject> ::= N
\end{lstlisting}

Assets are also application specific but similar to subjects we will use specific ones for the DRM application (taken from \cite{pucella2006}). \emph{ebook}, \emph{The Report} and \emph{latestJingle} are examples of specific subjects we will be using throughout. Syntactically an asset is just a positive number (\emph{N}).

% asset
\lstset{mathescape, language=AST}  
\begin{lstlisting}[frame=single, caption={asset},label={lst:assetast}]
<asset> ::= N
\end{lstlisting}

Agreements include policy sets. Each policy set specifies a \emph{prerequisite} and a \emph{policy}. In general if the prerequisite holds the policy is taken into consideration. Otherwise the policy will not be looked at. Some policy sets are specified as \emph{exclusive}. The \emph{Primitive Exclusive Policy Sets} are exclusive to agreement's users in that only those users may perform the actions specified in the policy set. The implication is that all other users who are not specified in the agreement's principal (prin) are forbidden from performing the specified actions. Finally policy sets could be grouped together in a \emph{conjunction} allowing a single agreement to be associated with many policy sets. 


% policySet

\lstset{mathescape, language=AST}  
\begin{lstlisting}[frame=single, caption={policySet},label={lst:policySetast}]
<policySet> ::=  
	$\vert$ <PrimitivePolicySet> : <preRequisite> $\rightarrow$ <policy> 
	$\vert$ <PrimitiveExclusivePolicySet> : <preRequisite> $\mapsto$ <policy>	 
	$\vert$ <AndPolicySet> : 'and'[ <policySet$_{1}$>, ..., <policySet$_{m}$> ]
\end{lstlisting}

A policy specifies an action to be performed on an asset, depending of whether the policy's prerequisite holds or not. If the prerequisite holds the agreement's user is permitted to perform the action on the agreement's asset; otherwise permission is denied. Similar to policy sets, policies could also be grouped together in a conjunction. The policy also includes a unique identifier. The policy identifier is added to help the translation (from agreements to formulas) but is optional in ODRL proper.

% policy

\lstset{mathescape, language=AST}  
\begin{lstlisting}[frame=single, caption={policy},label={lst:policyast}]
<policy> ::=  
	$\vert$ <PrimitivePolicy> : <preRequisite> $\Rightarrow_{<policyId>}$ <act>
	$\vert$ <AndPolicy> : 'and'[ <policy$_{1}$>, ..., <policy$_{m}$> ]
\end{lstlisting}

An \emph{Action} (\emph{act}) is simply a positive number. Similar to assets and subjects, actions are application specific. Some example actions taken from \cite{pucella2006} are \emph{Display} and \emph{Print}.

% act
\lstset{mathescape, language=AST}  
\begin{lstlisting}[frame=single, caption={act},label={lst:actast}]
<act> ::= N
\end{lstlisting}

A \emph{Policy Id} (\emph{policyId}) is a unique identifier specified as (increasing) positive integers. 

% id
\lstset{mathescape, language=AST}  
\begin{lstlisting}[frame=single, caption={policyId},label={lst:policyIdast}]
<policyId> ::= N
\end{lstlisting}

In ODRL0 a \emph{prerequisite} is either true or it is a \emph{constraint}. The \emph{true} prerequisite always holds. A constraint is an intrinsic part of a policy and cannot be influenced by agreement's user. Minimum height requirements for popular attractions and rides are examples of we would consider a constraint. The constraint \emph{ForEachMember} is interesting in its expressive power but has complicated semantics as we shall see in the ~\ref{sec:Semantics} section. Roughly speaking, ForEachMember takes a prin (a list of subjects) and a list L of constraints. The ForEachConstraint holds if each subject in prin satisfies each constraint in L.\emph{NotCons} is a negation of a constraint. The set of prerequisites are closed under conjunction (\emph{AndPrqs}), disjunction (\emph{OrPrqs}) and exclusive disjunction (\emph{XorPrqs}).

% prq

\lstset{mathescape, language=AST}  
\begin{lstlisting}[frame=single, caption={preRequisite},label={lst:preRequisiteast}]
<preRequisite> ::=  
	$\vert$ <TruePrq> : 'True'
	$\vert$ <Constraint> : <constraint>	 
	$\vert$ <ForEachMember> : 'ForEachMember' [<prin> ; <constraint$_{1}$>, ..., <constraint$_{m}$> ]	
	$\vert$ <NotCons> : 'not' [ <constraint> ]
	$\vert$ <AndPrqs> : 'and'[ <preRequisite$_{1}$>, ..., <preRequisite$_{m}$> ]
	$\vert$ <OrPrqs> : 'or'[ <preRequisite$_{1}$>, ..., <preRequisite$_{m}$> ]
	$\vert$ <XorPrqs> : 'xor'[ <preRequisite$_{1}$>, ..., <preRequisite$_{m}$> ]		
\end{lstlisting}

Constraints are either \emph{Principal}, \emph{Count} or \emph{CountByPrin}. Principal constraints basically require matching to specified prins. For example, the user being Alice is a Principal constraint. A count constraint refers to a set of policies \emph{P} and specifies the number of times the user of an agreement has invoked the policies in P to justify her actions. If the count constraint is part of a policy then the set P is composed of the single policy. In the case that the count constraint is part of a policy set, the set P is the set of policies specified in the policy set.

% constraint
\lstset{mathescape, language=AST}  
\begin{lstlisting}[frame=single, caption={constraint},label={lst:constraintast}]
<constraint> ::=  
	$\vert$ <Principal> : <prin>
	$\vert$ <Count> : 'Count' [N]
	$\vert$ <CountByPrin> : <prin> ('Count' [N])
\end{lstlisting}

% ---------------------------------- COQ -----------------------
\section{Coq Version}

\lstset{language=Coq}
\begin{lstlisting}[frame=single, caption={Coq version of agreement},label={lst:agreementcoq}]
Inductive agreement : Set :=
  | Agreement : prin -> asset -> policySet -> agreement.
\end{lstlisting}

% prin 
\lstset{language=Coq}
\begin{lstlisting}[frame=single, caption={prin},label={lst:princoq}]
Definition prin := nonemptylist subject.
\end{lstlisting}

% asset
\lstset{language=Coq}
\begin{lstlisting}[frame=single, caption={asset},label={lst:assetcoq}]
Definition asset := nat.
\end{lstlisting}



% subject
\lstset{language=Coq}
\begin{lstlisting}[frame=single, caption={subject},label={lst:subjectcoq}]
Definition subject := nat.
\end{lstlisting}


% policySet
\lstset{language=Coq}
\begin{lstlisting}[frame=single, caption={policySet},label={lst:policySetcoq}]
Inductive policySet : Set :=
  | PrimitivePolicySet : preRequisite -> policy -> policySet 
  | PrimitiveExclusivePolicySet : preRequisite -> policy  -> policySet 
  | AndPolicySet : nonemptylist policySet -> policySet.
\end{lstlisting}

% policy
\lstset{language=Coq}
\begin{lstlisting}[frame=single, caption={policy},label={lst:policycoq}]
Inductive policy : Set :=
  | PrimitivePolicy : preRequisite -> policyId -> act -> policy 
  | AndPolicy : nonemptylist policy -> policy.
\end{lstlisting}

% act
\lstset{language=Coq}
\begin{lstlisting}[frame=single, caption={act},label={lst:actcoq}]
Definition act := nat.
\end{lstlisting}

% id
\lstset{language=Coq}
\begin{lstlisting}[frame=single, caption={policyId},label={lst:policyIdcoq}]
Definition policyId := nat.
\end{lstlisting}

% prq
\lstset{language=Coq}
\begin{lstlisting}[frame=single, caption={preRequisite},label={lst:preRequisitecoq}]
Inductive preRequisite : Set :=
  | TruePrq : preRequisite
  | Constraint : constraint -> preRequisite 
  | ForEachMember : prin  -> nonemptylist constraint -> preRequisite
  | NotCons : constraint -> preRequisite 
  | AndPrqs : nonemptylist preRequisite -> preRequisite
  | OrPrqs : nonemptylist preRequisite -> preRequisite
  | XorPrqs : nonemptylist preRequisite -> preRequisite.
\end{lstlisting}

% constraint
\lstset{language=Coq}
\begin{lstlisting}[frame=single, caption={constraint},label={lst:constraintcoq}]
Inductive constraint : Set :=
  | Principal : prin  -> constraint 
  | Count : nat -> constraint 
  | CountByPrin : prin -> nat -> constraint.
\end{lstlisting}


%----------------------------------------------------------------------
\section{Semantics}
\label{sec:Semantics}

In this section, we describe the semantics of ODRL0 language by a translation from each language object (e.g. agreement) to a proposition in \emph{Coq}. The semantics will help answer queries of the form ``may subject \emph{s} perform action \emph{act} to asset \emph{a}?''. If the answer is yes, we say permission is granted. Otherwise permission is denied. 

Whether a permission is granted or denied depends on the agreements in question but also on the facts recorded in the environment. For ODRL0 those facts revolve around the number of times a policy has been used to justify an action. We encode this information in an \emph{environment} which is a conjunction of equalities of the form \emph{count(s, policyId) = n}. 

The Coq version of the count equality is a new inductive type called \emph{count_equality}. An environment is defined to be a non-empty list of count_equality objects.

\lstset{language=Coq, frame=single, caption={Environments and Counts},label={lst:environmentcoq}}
\begin{lstlisting}
Inductive count_equality : Set := 
   | CountEquality : subject -> policyId -> nat -> count_equality.

Inductive environment : Set := 
  | SingleEnv : count_equality -> environment
  | ConsEnv :  count_equality ->  environment -> environment.

\end{lstlisting}


The translation starts with the top level agreement element and proceeds by case analysis on the structure of the agreement. Note that each translation function takes an environment parameter.


\lstset{language=Coq, frame=single, caption={Translation of agreement},label={lst:transagreement}}
\begin{lstlisting}

Definition trans_agreement (e:environment)(ag:agreement) : Prop :=
  match ag with 
    | Agreement prin_u a ps => trans_ps e ps prin_u a
  end.

\end{lstlisting}

Translation of a policy set proceeds with case analysis of different Policy Set constructors. We then recurse into translation functions for the composing elements. The specific Coq propositions for each constructor is taken from the formula translation for each constructor defined in \cite{pucella2006}.

\lstset{mathescape, language=AST}  
\begin{lstlisting}[frame=single, caption={Policy Translation As Formulas},label={lst:transpolicyformula}]
$[\![ prq \rightarrow p]\!]^{prin_{u}, a}$ $\triangleq$ $\forall$ $(\!( [\![prin_{u}]\!]_{x} \land [\![prq]\!]_{x})\!)$
\end{lstlisting}





\lstset{language=Coq, frame=single, caption={Translation of Policy Set},label={lst:transps}}
\begin{lstlisting}

Fixpoint trans_ps
  (e:environment)(ps:policySet)(prin_u:prin)(a:asset){struct ps} : Prop :=

let trans_ps_list := (fix trans_ps_list (ps_list:nonemptylist policySet)(prin_u:prin)(a:asset){struct ps_list}:=
  match ps_list with
    | Single ps1 => trans_ps e ps1 prin_u a
    | NewList ps ps_list' => ((trans_ps e ps prin_u a) /\ (trans_ps_list ps_list' prin_u a))
  end) in
    match ps with
    | PrimitivePolicySet prq p => forall x, (((trans_prin x prin_u) /\ 
                                   (trans_preRequisite e x prq (getId p) prin_u)) -> 
                                   (trans_policy_positive e x p prin_u a))  

    | PrimitiveExclusivePolicySet prq p => forall x, ((((trans_prin x prin_u) /\ 
                                              (trans_preRequisite e x prq (getId p) prin_u)) -> 
                                             (trans_policy_positive e x p prin_u a)) /\
                                            ((not (trans_prin x prin_u)) -> (trans_policy_negative e x p a)))
                   
    | AndPolicySet ps_list => trans_ps_list ps_list prin_u a
    end.
\end{lstlisting}




 
%
%======================================================================
\chapter{Motivation}
%======================================================================

%----------------------------------------------------------------------
\section{Logic Based Semantics for ODRL}
%----------------------------------------------------------------------

%See equation \ref{eqn_pi} on page \pageref{eqn_pi}.\footnote{A famous %equation.}

Formal logic can represent the statements and facts we express in a natural language like English. Propositional logic is expressive enough to express simple facts as propositions and uses connectives to allow for the negation, conjunction and disjunction of the facts. However propositional logic is not expressive enough to express policies of the kind used in languages like ODRL and XrML. For example, a simple policy expressed in English like ``All who pay 5 dollars can watch the movie Toy Story'' cannot be expressed in propositional logic because the concept of  variables doesn't exist in propositional logic. 

A richer logic such as ``Predicate Logic'' or ``First Order Logic'' (\emph{FOL}) is more suitable and has the expressive power to represent policies written in English. Moreover, FOL can be used to capture the meaning of policies in an unambiguous way.

Halpern and Weissman [REF][Using First Order Logic to Reason about Policies] propose a fragment of FOL to represent and reason about policies. The fragment of FOL they arrive at is called \emph{Lithium} which is decidable and allows for efficiently answering interesting queries. Lithium restricts policies to be written based on the concept of ``bipolarity'' which disallows by construction policies that both permit and deny an action on an object. Pucella and Weissman ~\cite{pucella2006} specify a predicate logic based language that represents a subset of ODRL.


\section{Specific Problem}

Policy languages and the agreements written in those languages are meant to implement specific goals such as limiting access to specific assets. The tension in designing a policy language is usually between how to make the language expressive enough, such that the design goals for the policy language may be expressed, and how to make the policies verifiable with respect to the stated goals.

As stated earlier, an important part of fulfilling the verifiability goal is to have formal semantics defined for policy languages. For \ac{odrl}, authors of ~\cite{pucella2006} have defined a formal semantics based on which they declare and prove a number of important theorems (their main focus is on stating and proving algorithm complexity results). However as with many paper-proofs, the language used to do the proofs while mathematical in nature, uses a lot of intuitive justifications to show the proofs. As such these proofs are difficult to verify or more importantly to ``derive''. Furthermore the proofs can not be used directly to render a decision on a sample policy (e.g. whether to allow or deny access to an asset). Of course one may (carefully) construct a program based on these proofs for practical purposes but we will have no way of certifying those programs correct, even assuming the original proofs were in fact correct.

While there are paper-proofs for \ac{odrl}, as far as we know, similar paper-proofs do not exist for an important (mandatory) access-control policy system, namely \ac{selinux}[REF]. In particular no formal proofs (paper based or otherwise) of decidablity of \ac{selinux} policies exist in the literature. 


\section{Contributions}

In this thesis we will build a language representation framework based on \ac{odrl} and definitions in ~\cite{pucella2006}. The framework will in \emph{Coq} which is both a programming language and a proof-assistant. We will declare and prove decidablity results of subsets of \ac{odrl} all the way up to the complete \ac{odrl} fragement defined in ~\cite{pucella2006}. We will extract programs from the proofs and demonstrate how they can be used on specific policies to render a specific decision such as ``a conflict has been detected''. 

Beside ``certified decidablity results'' for \ac{odrl}, we will investigate decidablity for \ac{selinux} policies, proving decidablity or show why a proof is not possible (if that is the case) and provide proposals to make the policy language decidable.

By using the Coq framework originally built for \ac{odrl} to encode and verify agreements written in a second policy language (albeit a different class of policy language: \ac{rel} vs access-control) we will demonstrate the suitability of this Coq based framework for other policy languages such as \ac{xacml}[REF].

\section{What specific work has been accomplished until this point in time? what results were obtained so far?}

The encodings for a subset of \ac{odrl} called ODRL0 (see ~\ref{sec:odrl0}) plus some important functions implementing some of the algorithms in ~\cite{pucella2006} have been implemented in Coq. Some of the intermediate theorems have been also been defined and proved.

\section{What remains to be done to complete the thesis research?}
The main decidablity result and its proof for ODRL0 will be completed first. We will add the remaining \ac{odrl} constructs incrementally while maintaining decidablity for the main decision algorithm. The remaining constructs include a trouble-some construct (~\cite{pucella2006}), namely $not[policySet]$. We will show this construct does not change the decidablity result already established. 

ODRL0 is enough to be used as a basis for \ac{selinux} policies without \emph{constrain}s. \ac{selinux} constrains are extra  conditions that need to be satisfied (in addition to policies) in order for a permission to be granted. We will investigate decidablity for this subset first. We will then add constrains to the ODRL0 subset (as pre-requisites) and investigate decidablity.



\section{What is the timetable to complete the work?}

The study plan calls for August of 2015 for everything to be completed.

%% Some LaTeX commands I define for my own nomenclature.
% If you have to, it's better to change nomenclature once here than in a 
% million places throughout your thesis!
\newcommand{\package}[1]{\textbf{#1}} % package names in bold text
\newcommand{\cmmd}[1]{\textbackslash\texttt{#1}} % command name in tt font 


%======================================================================
\chapter{Observations}
%======================================================================

This would be a good place for some figures and tables.

Some notes on figures and photographs\ldots

\begin{itemize}
\item A well-prepared PDF should be 
  \begin{enumerate}
    \item Of reasonable size, {\it i.e.} photos cropped and compressed.
    \item Scalable, to allow enlargment of text and drawings. 
  \end{enumerate} 
\item Photos must be bit maps, and so are not scaleable by definition. TIFF and
BMP are uncompressed formats, while JPEG is compressed. Most photos can be
compressed without losing their illustrative value.
\item Drawings that you make should be scalable vector graphics, \emph{not} 
bit maps. Some scalable vector file formats are: EPS, SVG, PNG, WMF. These can
all be converted into PNG or PDF, that pdflatex recognizes. Your drawing 
package probably can export to one of these formats directly. Otherwise, a 
common procedure is to print-to-file through a Postscript printer driver to 
create a PS file, then convert that to EPS (encapsulated PS, which has a 
bounding box to describe its exact size rather than a whole page). 
Programs such as GSView (a Ghostscript GUI) can create both EPS and PDF from PS files.
Appendix~\ref{ch:Appendix-Matlab} shows how to generate properly sized Matlab plots and save them as PDF.
\item It's important to crop your photos and draw your figures to the size that
you want to appear in your thesis. Scaling photos with the 
includegraphics command will cause loss of resolution. And scaling down 
drawings may cause any text annotations to become too small.
\end{itemize}
 
For more information on \LaTeX\, see the uWaterloo Skills for the Academic Workplace 
course notes at \href{http://saw.uwaterloo.ca/latex}{saw.uwaterloo.ca/latex}. 
\footnote{
Note that while it is possible to include hyperlinks to external documents,
it is not wise to do so, since anything you can't control may change over time. 
It \emph{would} be appropriate and necessary to provide external links to 
additional resources for a multimedia ``enhanced'' thesis. 
But also note that if the \package{hyperref} package is not included, 
as for the print-optimized option in this thesis template, any \cmmd{href} 
commands in your logical document are no longer defined.
A work-around employed by this thesis template is to define a dummy \cmmd{href} 
command (which does nothing) in the preamble of the document, 
before the \package{hyperref} package is included. 
The dummy definition is then redifined by the
\package{hyperref} package when it is included.
}

%The classic book by Leslie Lamport~\cite{lamport.book}, author of \LaTeX , is worth a look too, and the many available add-on packages are described by 
%Goossens \textit{et~al.}~\cite{goossens.book}. Some on-line documentation is linked
%to from \href{http://saw.uwaterloo.ca/latex}{saw.uwaterloo.ca/latex}. 



Here is an example of how to include figures in \LaTeX. 
Figure~\ref{fig.beam} shows a cantilever beam of circular cross-section
subjected to a point load and a uniformly distributed load, both of which are uncertain. Note that it is better not to include the extension of the figure's source file.

\begin{figure}[!htbp]
 \begin{center}
  \includegraphics[clip=true]{figures/beam}
 \end{center}
\caption{Cantilever Beam}
\label{fig.beam}
\end{figure}



%----------------------------------------------------------------------
\section{Adding Nomenclature}
%----------------------------------------------------------------------

The following example is part of the ``nomentbl'' package. Refer to the package's documentation for more details.

\bigskip

\noindent
Let's start with equations to show how to use greek and mathematical symbols within Nomenclature.

Here is an equation
%
\begin{equation}\label{eq:heatflux}
  \dot{Q} = k \cdot A \cdot \Delta T
\end{equation}%
%
%% Greek and math symbols
\nomenclature[gQ]{$\dot{Q}$}{heat flux}{W}{}%
\nomenclature[gk]{$k$}{overall heat transfer coefficient}{$\frac{\mathrm{W}}{\mathrm{m}^2\mathrm{K}}$}{see eq.~(\ref{eq:ohtc})}%
\nomenclature[gA]{$A$}{area}{m$^2$}{$L^2$}%
\nomenclature[gL]{$L$}{length}{m}{SI base quantity}%
\nomenclature[gT]{$T$}{temperature}{K}{SI base quantity}%
\nomenclature[gT]{$\Delta T$}{temperature difference}{K}{SI base quantity}%

Here is another one
%
\begin{equation}\label{eq:ohtc}
  \frac{1}{k} = \left[\frac{1}{\alpha _{\mathrm{i}}\,r_{\mathrm{i}}} +
    \sum^n_{j=1}\frac{1}{\lambda _j}\,
    \ln \frac{r_{\mathrm{a},j}}{r_{\mathrm{i},j}} +
    \frac{1}{\alpha _{\mathrm{a}}\,
      r_{\mathrm{a}}}\right] \cdot r_{\mathrm{reference}}
\end{equation}%
%
%% Greek and math symbols
\nomenclature[ga]{$\alpha$}{convection heat transfer coefficient}{$\frac{\mathrm{W}}{\mathrm{m}^2\mathrm{K}}$}{}%
\nomenclature[gl]{$\lambda$}{thermal conductivity}{$\frac{\mathrm{W}}{\mathrm{m K}}$}{}%
%
%% Subscripts
\nomenclature[za]{a}{out}{}{}%
\nomenclature[zi]{i}{in}{}{}%
\nomenclature[zj]{$j$}{running parameter}{}{}% 
\nomenclature[zn]{$n$}{number of walls}{}{}%


\bigskip

\noindent
The following example is to show how to use abbreviations within the Nomenclature.\\
EECS is a school at the UO.
%% Abbreviations
\nomenclature[a]{EECS}{Electrical Engineering and Computer Science}{}{}%
\nomenclature[a]{UO}{University of Ottawa}{}{}%



\bigskip

\noindent
Don't forget to run:
\begin{verbatim}
makeindex -s nomentbl.ist -o uottawa-thesis.nls uottawa-thesis.nlo
\end{verbatim}



%%% Local Variables: 
%%% mode: latex
%%% TeX-master: "../uottawa-thesis"
%%% End: 



%----------------------------------------------------------------------
% APPENDICES
%---------------------------------------------------------------------- 
%\appendix
% Designate with \appendix declaration which just changes numbering style 
% from here on
% Add a title page before the appendices and a line in the Table of Contents
%\chapter*{APPENDICES}
\addcontentsline{toc}{chapter}{APPENDICES} 
%
%% An appendix
%======================================================================
\chapter{Sources of Information and Help}
\label{ch:Appendix-Sources-of-Info}
%======================================================================
The best source of information about \LaTeX\ is the two books mentioned in this course \cite{lamport.book,goossens.book}.
Another excellent resource is the UseNet newsgroup \verb=comp.text.tex=.
A frequently-asked-questions (FAQ) list is also maintained by this news group.
You might also search the World Wide Web for ``LaTeX'' for other sources of help.
 %"Sources of Information and Help"
%% An appendix
%======================================================================
\chapter[PDF Plots From Matlab]{Matlab Code for Making a PDF Plot}
\label{ch:Appendix-Matlab} 
%======================================================================
\section{Using the GUI}
Properties of Matab plots can be adjusted from the plot window via a graphical interface. Under the Desktop menu in the Figure window, select the Property Editor. You may also want to check the Plot Browser and Figure Palette for more tools. To adjust properties of the axes, look under the Edit menu and select Axes Properties.

To set the figure size and to save as PDF or other file formats, click the Export Setup button in the figure Property Editor.

\section{From the Command Line} 
All figure properties can also be manipulated from the command line. Here's an example: 
\begin{verbatim}
x=[0:0.1:pi];
hold on % Plot multiple traces on one figure
plot(x,sin(x))
plot(x,cos(x),'--r')
plot(x,tan(x),'.-g')
title('Some Trig Functions Over 0 to \pi') % Note LaTeX markup!
legend('{\it sin}(x)','{\it cos}(x)','{\it tan}(x)')
hold off
set(gca,'Ylim',[-3 3]) % Adjust Y limits of "current axes"
set(gcf,'Units','inches') % Set figure size units of "current figure"
set(gcf,'Position',[0,0,6,4]) % Set figure width (6 in.) and height (4 in.)
cd n:\thesis\plots % Select where to save
print -dpdf plot.pdf % Save as PDF
\end{verbatim}  %"Matlab Code for Making a PDF Plot"

%----------------------------------------------------------------------
% END MATERIAL
%----------------------------------------------------------------------

% B I B L I O G R A P H Y
% -----------------------
%
% The following statement selects the style to use for references.  It controls the sort order of the entries in the bibliography and also the formatting for the in-text labels.
\bibliographystyle{plain}
% This specifies the location of the file containing the bibliographic information.  
% It assumes you're using BibTeX (if not, why not?).
\ifthenelse{\boolean{PrintVersion}}{
\cleardoublepage % This is needed if the book class is used, to place the anchor in the correct page,
                 % because the bibliography will start on its own page.
}{
\clearpage       % Use \clearpage instead if the document class uses the "oneside" argument
}
\phantomsection  % With hyperref package, enables hyperlinking from the table of contents to bibliography             
% The following statement causes the title "References" to be used for the bibliography section:
\renewcommand*{\bibname}{References}

% Add the References to the Table of Contents
\addcontentsline{toc}{chapter}{\textbf{References}}

\bibliography{bibliography/keylatex}
% Tip 5: You can create multiple .bib files to organize your references. 
% Just list them all in the \bibliogaphy command, separated by commas (no spaces).


%----------------------------------------------------------------------
\end{document}
%======================================================================



%%% Local Variables: 
%%% mode: latex
%%% TeX-master: t
%%% End: 
